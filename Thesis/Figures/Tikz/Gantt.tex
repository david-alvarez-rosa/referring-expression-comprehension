% -*- TeX-master: "../../Thesis.tex" -*-


\begin{tikzpicture}
  \begin{ganttchart}[
    % Cuadrícula.
    vgrid = {{black!25, loosely dotted}, {black, loosely dotted}},
    hgrid = {*1{black, loosely dotted}},
    y unit chart = 1.03cm,
    % Título (semanas).
    title/.append style = {fill = yellowGantt},
    title height = .75,
    % Fondo.
    canvas/.append style = {fill = yellowGantt!15},
    expand chart = \textwidth,
    % Estilo día (en vertical).
    today = 8,
    today offset = .7,
    today label = Presente,
    today rule/.style = {draw = black!70, dashed, very thick},
    % Estilo progreso.
    progress = today,
    progress label text = {\pgfmathprintnumber[precision=0, verbatim]{#1}\% completado},
    bar progress label anchor = west,
    group progress label anchor = west,
    % Estilos grupos.
    group/.append style = {fill = greenGantt, draw = black, thick},
    group incomplete/.append style = {fill = greenGantt!30, draw = black, thick},
    group height = .5,
    group top shift = .25,
    group right shift = 0,
    group left shift = 0,
    group peaks height = .175,
    group peaks width = .65,
    group peaks tip position = .4,
    group label node/.append style= {align = right}, % Importante.
    % Estilo barras.
    bar/.append style = {fill = blueGantt},
    bar height = .5,
    bar incomplete/.append style = {fill = blueGantt!30, draw = black},
    bar label node/.append style= {align = right},
    % Barras verticales auxiliares.
    vrule/.style = {draw = none}
    ]{1}{28} % En medias semanas (dos días de clase/semana).

    % Semanas.
    \gantttitlelist{1,...,14}{2} \\

    % Entrega I.
    \ganttgroup{A. Propuesta\ganttalignnewline proyecto}{1}{8} \\
    \ganttbar{A1. Definición\ganttalignnewline general}{1}{4} \\
    \ganttbar{A2. Objetivos y \ganttalignnewline especificaciones}{3}{6} \\
    \ganttbar{A3. Entrega I}{5}{8} \\

    % Entrega II.
    \ganttgroup[progress = 0]{B. Análisis\ganttalignnewline usuarios}{8}{11} \\
    \ganttbar[progress = 0]{B1. Análisis \ganttalignnewline sistemas-usuario}{8}{9} \\
    \ganttbar[progress = 0]{B2. Análisis de\ganttalignnewline funciones}{8}{10} \\
    \ganttbar{B3. Entrega II}{9}{11} \\

    % Entrega III.
    \ganttgroup{C. Diseño\ganttalignnewline conceptual}{9}{14} \\
    \ganttbar{C1. Ergonomia y \ganttalignnewline seguridad}{9}{11} \\
    \ganttbar{C2. Normativa\ganttalignnewline vigente}{11}{13} \\
    \ganttbar{C3. Entrega III}{11}{14} \\

    % Entrega IV.
    \ganttgroup{D. Estudio\ganttalignnewline alternativas}{13}{20} \\
    \ganttbar{D1. Descripción\ganttalignnewline alternativas}{13}{16} \\
    \ganttbar{D2. Compración\ganttalignnewline y selección}{15}{18} \\
    \ganttbar{D3. Entrega IV}{15}{20} \\

    % Entrega V.
    \ganttgroup{E. Propuesta\ganttalignnewline solución}{19}{28} \\
    \ganttbar{E1. Memoria}{19}{26} \\
    \ganttbar{E2. Póster}{23}{26} \\
    \ganttbar{E3. Entrega V}{25}{28}

    % Barras verticales auxiliares.
    \ganttvrule{Inicio}{1} \ganttvrule{Fin}{27}
  \end{ganttchart}
\end{tikzpicture}