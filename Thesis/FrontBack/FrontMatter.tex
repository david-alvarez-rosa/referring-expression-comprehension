% -*- TeX-master: "../Thesis.tex" -*-


\begin{titlepage}
  \hypersetup{
    urlcolor = black
  }
  \centering

  \begin{minipage}[c]{.25\textwidth}
    \centering
    \href{https://www.utoronto.ca/}{\includesvg[height=.8\textwidth]{Logos/UofT.svg}}\\[.5ex]
    \normalsize
    \textsl{Supervisor}\large\\
    \href{https://www.cs.utoronto.ca/~fidler/}{Sanja \textsc{Fidler}}
  \end{minipage}\hfill
  \begin{minipage}[c]{.5\textwidth}
    \scshape\Large
    \centering
    \href{https://www.utoronto.ca/}{University of Toronto} \\[2ex]
    \href{https://www.upc.edu/en}{Politechnical University\\
      of Catalonia}
  \end{minipage}\hfill
  \begin{minipage}[c]{.25\textwidth}
    \centering
    \href{https://www.upc.edu/en}{\includesvg[height=.8\textwidth]{Logos/UPC.svg}}\\[.5ex]
    \normalsize
    \textsl{Co-Supervisor}\large\\
    \href{https://imatge.upc.edu/web/people/xavier-giro}{Xavier \textsc{Giró}}
  \end{minipage}\\

  \vfill

  \LARGE
  \resizebox{\textwidth}{4pt}{\pgfornament[width=\textwidth]{88}}\vspace{.75ex}
  \href{https://recomprehension.com}{\textbf{Exploring and Visualizing\\
      Referring Expression Comprehension}}\\[-.5ex]
  \resizebox{\textwidth}{4pt}{\pgfornament[width=\textwidth]{88}}\\

  \vfill

  \large
  \textsl{A senior Bachelor's Degree Thesis written by}\Large\\
  \href{https://david.alvarezrosa.com/}{David \textsc{Álvarez Rosa}}\\[1ex]

  \large
  \textsl{in partial fullfilment of the requirements for the bachelors' degrees
    in}\Large\\[-.25ex]
  \textsc{mathematics}\\[-.5ex]
  \textsc{industrial technology engineering}\\[1ex]

  \large
  \textsl{and submitted to}\Large\\[-.25ex]
  \scshape
  \href{https://cfis.upc.edu/en}{interdisciplinary higher education centre}\\[-.5ex]
  \href{https://etseib.upc.edu/en}{barcelona school of industrial engineering}\\[-.5ex]
  \href{https://fme.upc.edu/en}{school of mathematics and statistics}


  \vfill
  \href{https://cfis.upc.edu/en}{\includegraphics[height=10ex]{Logos/CFIS.jpeg}}
  \hspace{1em}
  \href{https://etseib.upc.edu/en}{\includegraphics[height=10ex]{Logos/ETSEIB.png}}
  \hspace{1em}
  \href{https://fme.upc.edu/en}{\includegraphics[height=10ex]{Logos/FME.png}}\\[1ex]

  \vfill

  \normalsize
  Toronto, ON, Canada\hfill\today
\end{titlepage}


% \cleardoublepage
% \thispagestyle{empty}
% \vspace*{\stretch{1}}
% \begin{center}
%   \huge
%   Exploring and Visualizing\\
%   Referring Expression Comprehension
% \end{center}
% \vspace{\stretch{3}}


% \clearpage
\thispagestyle{empty}
\small
\null\vfill

\begin{center}
  \pgfornament[width=.25\textwidth]{156}
\end{center}

\vspace{1ex}
\noindent\href{https://david.alvarezrosa.com/}{David \textsc{Álvarez Rosa}}
\textcopyright\ \today\\
\makeatletter\href{https://recomprehension.com/}{\textsl{\@title}}\makeatother\\
\url{https://recomprehension.com}


\bigskip
\noindent Thesis typeset with
\href{http://tug.org/applications/pdftex/}{pdf\TeX{}} 3.14159265--2.6--1.40.21
(\href{https://www.tug.org/svn/texlive/}{\TeX{} Live} 2020) on
\href{https://archlinux.org/}{Arch Linux} using
\href{https://www.ctan.org/tex-archive/fonts/lm/}{Latin Modern} typefaces and
written with \href{https://www.gnu.org/software/emacs/}{GNU Emacs}. The
\href{https://www.ctan.org/pkg/biblatex}{Bib\LaTeX{}} package has been used for
bibliography management with
\href{http://biblatex-biber.sourceforge.net/}{Biber} as processing backend.

Vector graphics have been created by the author using
\href{https://www.ctan.org/pkg/pgf}{PGF/Ti\emph{k}Z}. Vectorian decorative
ornaments are from the \href{https://www.latex-project.org//}{\LaTeX} package
\href{https://ctan.org/pkg/pgfornament}{\texttt{pgfornament}}.

% Dear Robin Schneider, I am very sorry for this, but an underfull \hbox is
% making my mycrotype OCD worse for hours.
\bigskip\noindent
\begin{minipage}{.66\textwidth}
  This thesis is licensed under a Creative Commons
  \href{https://creativecommons.org/licenses/by-nc-sa/4.0/deed.en}{``At\-tri\-bu\-tion--NonCommercial--ShareAlike
    4.0 International''} license.%
  % \doclicenseLongText{}
\end{minipage}\hfill
\begin{minipage}{.265\textwidth}
  \doclicenseImage[imagewidth=\textwidth]%
\end{minipage}
% \doclicenseThis{}
\normalsize



\cleardoublepage{}
\thispagestyle{empty}
\vspace*{\stretch{1}}
\begin{flushright}
  \itshape{}
  To my mother, for her love and patience. \\
  To all my friends.
\end{flushright}
\vspace{\stretch{2}}



\cleardoublepage{}
\thispagestyle{plain}
\null\vfill

\begin{center}
  \Large
  \textbf{Exploring and Visualizing\\
    Referring Expression Comprehension}

  \vspace{2ex}
  \large
  \textit{by} David \textsc{Álvarez Rosa}

  \vspace{3ex}
  \textbf{\abstractname}
\end{center}

\vspace{-2ex}
\noindent Human-machine interaction is one of the main objectives currently in
the field of Artificial Intelligence. This work will contribute to enhance this
interaction by exploring the new task of Referring Expression Comprehension
(REC), consisting of: given a referring expression---which can be a linguistic
phrase or human speech---and an image, detect the object to which the
expression refers (i.e., achieve a binary segmentation of the referred
object). The multimodal nature of this task will require the use of different
deep learning architectures, among them: convolutional neural networks
(computer vision); and recurrent neural networks and the Transformer model
(natural language processing).

This thesis is presented as a self-contained document that can be understood by
a reader with no prior knowledge of machine learning. The bulk of the work
consists of an exhaustive study of the REC task: from the applications; until
the study, comparison and implementation of models; going through a complete
description of the current state of the art. Likewise, a functional, free and
public web page is presented in which interaction is allowed in a simple way
with the model described in this work.

\begin{center}
  \pgfornament[width=.5\textwidth]{89}


  \bigskip\smallskip
  \textbf{Keywords}

  Referring Expression Comprehension\\
  Artificial Intelligence \textbullet{} Machine Learning \textbullet{} Deep
  Learning\\
  Computer Vision \textbullet{} Natural Language Processing\\
  Multimodal Learning\\

  \bigskip
  \href{https://mathscinet.ams.org/msc/msc2010.html}{\textbf{Mathematics
      Subject Classification}}

  \href{https://mathscinet.ams.org/msc/msc2010.html?t=68Txx}{68T45}
\end{center}

\vfill\null{}


\clearpage
\thispagestyle{plain}
\null\vfill

\begin{otherlanguage}{spanish}
  \begin{center}
    \Large
    \href{https://recomprehension.com/}{\textbf{Explorando y Visualizando\\
        Comprensión de la Expresión Referente}}

    \vspace{2ex} \large \textit{por}
    \href{https://david.alvarezrosa.com/}{David \textsc{Álvarez Rosa}}

    \vspace{3ex} \textbf{\abstractname}
  \end{center}

  \vspace{-2ex}
  \noindent La interacción humano-máquina es uno de los objetivos principales
  actualmente en el ámbito de la Inteligencia Artifcial. En este trabajo se
  contribuirá a facilitar esta interacción explorando la novedosa tarea de
  Comprensión de la Expresión Referente (CER), consistente en: dada una
  expresión referente ---que puede ser una frase lingüística o habla humana---
  y una imagen, detectar el objeto al que la expresión se refiere (i.e.,
  conseguir una segmentación binaria del objeto referido). El caracter
  multimodal de este cometido hará necesario el uso de diferentes arquitecturas
  de aprendizaje profundo, entre ellas: redes neuronales convolucionales
  (visión artificial); y redes neuronales recurrentes y el modelo
  \textit{Transformer} (procesamiento del lenguaje natural).

  Esta tesis se presenta como un documento autosuficiente que puede ser
  entendido por un lector sin conocimientos previos en aprendizaje
  automático. El grueso del trabajo consiste en un estudio exhaustivo de la
  tarea de CER: desde las aplicaciones; hasta el estudio, comparación e
  implementación de modelos; pasando por una descripción completa del estado
  del arte actual. Así mismo, se presenta una página web funcional, gratuita y
  pública en la que se permite la interacción de una manera sencilla con el
  modelo descrito en este trabajo.

  \begin{center}
    \pgfornament[width=.5\textwidth]{89}


    \bigskip\smallskip \textbf{Palabras clave}

    Comprensión de la Expresión Referente\\
    Inteligencia Artificial \textbullet{} Aprendizaje Automático \textbullet{}
    Aprendizaje Profundo\\
    Visión Artificial \textbullet{} Porcesamiento del Lenguaje Natural\\
    Aprendizaje Multimodal\\

    \bigskip
    \href{https://mathscinet.ams.org/msc/msc2010.html}{\textbf{Clasificación
        Matemática por Temas}}

    \href{https://mathscinet.ams.org/msc/msc2010.html?t=68Txx}{68T45}
  \end{center}
\end{otherlanguage}
\vfill\null{}


\clearpage
\thispagestyle{plain}
\null\vfill

\begin{otherlanguage}{catalan}
  \begin{center}
    \Large
    \href{https://recomprehension.com/}{\textbf{Explorant i Visualitzant\\
        Comprensió de l'Expressió Referent}}

    \vspace{2ex} \large \textit{per}
    \href{https://david.alvarezrosa.com/}{David \textsc{Álvarez Rosa}}

    \vspace{3ex} \textbf{\abstractname}
  \end{center}

  \vspace{-2ex}
  \noindent La interacció humà-màquina és un dels objectius principals
  actualment en l'àmbit de la Inte\lgem{}igència Artifcial. En aquest treball
  es contribuirà a facilitar aquesta interacció explorant la nova tasca de
  Comprensió de l'Expressió Referent (CER), que consisteix en: donada una
  expressió referent ---que pot ser una frase lingüística o parla humana--- i
  una imatge, detectar l'objecte a què l'expressió es refereix (i.e.,
  aconseguir una segmentació binària de l'objecte referit). El caràcter
  multimodal d'aquesta comesa farà necessari l'ús de diferents arquitectures
  d'aprenentatge profund, entre elles: xarxes neuronals convolucionals (visió
  artificial); i xarxes neuronals recurrents i el model \textit{Transformer}
  (processament de el llenguatge natural).

  Aquesta tesi es presenta com un document autosuficient que pot ser entès per
  un lector sense coneixements previs en aprenentatge automàtic. El gruix de la
  feina consisteix en un estudi exhaustiu de la tasca de CER: des de les
  aplicacions; fins a l'estudi, comparació i implementació de models; passant
  per una descripció completa de l'estat de l'art actual. Així mateix, es
  presenta una pàgina web funcional, gratuïta i pública en la qual es permet la
  interacció d'una manera senzilla amb el model descrit en aquest treball.

  \begin{center}
    \pgfornament[width=.5\textwidth]{89}


    \bigskip\smallskip \textbf{Paraules Clau}

    Comprensió de l'Expressió Referent\\
    Inte\lgem{}igència Artificial \textbullet{} Aprenentatge Automàtic
    \textbullet{} Aprenentatge Profund\\
    Visió Artificial \textbullet{} Procesament del Llenguatge Natural\\
    Aprenentatge Multimodal\\

    \bigskip
    \href{https://mathscinet.ams.org/msc/msc2010.html}{\textbf{Classificació
        Matemàtica per Temes}}

    \href{https://mathscinet.ams.org/msc/msc2010.html?t=68Txx}{68T45}
  \end{center}
\end{otherlanguage}
\vfill\null{}



\chapter*{Acknowledgements}


I would like to express my gratitude to Prof.\
\href{https://www.cs.utoronto.ca/~fidler/}{Sanja \textsc{Fidler}} for giving me
the opportunity to carry out this project under her supervision and allowing me
to be part of her laboratory and connect with its members. I also want to
express my thanks to the entire \href{https://vectorinstitute.ai/}{Vector
  Institute} staff for allowing me to use their computational resources, as
well as for remotely assisting me with any technical problems that arose.

Likewise, I want to thank Prof.\
\href{https://imatge.upc.edu/web/people/xavier-giro}{Xavier \textsc{Giró}} for
his work as liaison co-supervisor between Canada and Barcelona and for being
part of the evaluation panel of this thesis.

Moreover, I want to express my gratitude to
\href{http://cellex-mpq.icfo.eu/about_2/}{\itshape Fundació Privada Cellex} and
the \href{https://cfis.upc.edu/en}{Interdisciplinary Higher Education
  Centre}. They have been the engines of my academic education and those that
have allowed me to be part of this adventure of studying two official
bachelors' degrees simultaneously in the
\href{https://www.upc.edu/en}{Politechnical University of Catalonia}. Within
this great team I want to give special thanks to
\href{https://spcom.upc.edu/en/people/antonio-pascual-iserte}{Toni
  \textsc{Pascual}} for his management of the mobility stay and the
complications arising from the
\href{https://en.wikipedia.org/wiki/Covid-19}{COVID-19} pandemic. Thanks also
to \href{https://mat.upc.edu/en/people/miguel.angel.barja}{Miguel Ángel
  \textsc{Barja}}---with whom I have been lucky to be his student---for his
role as director of the center.

Finally, I want to thank my family and friends for their unconditional moral
support without asking me too much \textit{``When will you graduate?''}---or at
least not very often.

\begin{flushright}
  \href{https://david.alvarezrosa.com/}{David \textsc{Álvarez Rosa}}\\
  \today
\end{flushright}

\begin{center}
  \pgfornament[width=.3\textwidth]{160}
\end{center}



\microtypesetup{protrusion=false}
\tableofcontents
\microtypesetup{protrusion=true}



\chapter*{\acronymname}\label{cha:acronyms}
\markboth{\MakeUppercase\acronymname}{}
\addcontentsline{toc}{chapter}{\acronymname}


The acronyms used in this thesis have been divided into two blocks: those
considered primary and those with the names for the models mentioned in this
document.


\setglossarypreamble[main]{\label{sec:ac-main} Acronyms that refer to the main
  topic of this thesis used in this work with the notation used, their
  corresponding description and page list. \medskip}

\printglossary[type=main]


\setglossarypreamble[models]{\label{sec:ac-model}Acronyms for model names used
  in this work with the notation used, their corresponding description and page
  list. \medskip}

\printglossary[type=models]



\microtypesetup{protrusion=false}
\listoffigures
\listoftables
\microtypesetup{protrusion=true}
