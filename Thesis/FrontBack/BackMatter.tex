% -*- TeX-master: "../Thesis.tex" -*-


\chapter*{\acronymname}\label{cha:acronyms}
\markboth{\MakeUppercase\acronymname}{}
\addcontentsline{toc}{chapter}{\acronymname}


The acronyms used in this thesis are shown below divided into 6 large blocks:
those considered basic, those related to \acl{ml}, the main ones within the
topic of the thesis, those related to loss functions, the names of the models
mentioned and those related to web development.



\setglossarypreamble[basic]{\label{sec:ac-basic} Basic acronyms used in this
  work with the notation used, their corresponding description and page
  list. \medskip}

\printglossary[type=basic]



\setglossarypreamble[dl]{\label{sec:ac-dl} Acronyms related to \acs{dl} used in
  this work with the notation used, their corresponding description and page
  list. \medskip}

\printglossary[type=dl]



\setglossarypreamble[main]{\label{sec:ac-main} Acronyms that refer to the main
  topic of this thesis used in this work with the notation used, their
  corresponding description and page list. \medskip}

\printglossary[type=main]



\setglossarypreamble[losses]{\label{sec:ac-loss}Acronyms denoting loss
  functions used in this work with the notation used, their corresponding
  description and page list. \medskip}

\printglossary[type=losses]



\setglossarypreamble[models]{\label{sec:ac-model}Acronyms for model names used
  in this work with the notation used, their corresponding description and page
  list. \medskip}

\printglossary[type=models]



\setglossarypreamble[web]{\label{sec:ac-web}Acronyms related to web development
  used in this work with the notation used, their corresponding description and
  page list. \medskip}

\printglossary[type=web]




\printbibheading[heading=bibintoc]\label{cha:references}
\noindent The complete bibliography has been divided into three: primary
sources (which contains all the citations that appear in the text), figure
sources and quotation sources. \bigskip\medskip

\printbibliography[heading=subbibintoc, category={text}, title={Primary Sources}]
\printbibliography[heading=subbibintoc, category={figure}, title={Figure Sources}]
\printbibliography[heading=subbibintoc, category={quote}, title={Quotation Sources}]




\indexprologue{\label{cha:index}\noindent List of topics and terms present in
  this thesis arranged alphabetically with their corresponding page numbers to
  locate them within this publication.}

\printindex




\cleardoublepage
\pagestyle{empty}
\null\vfill
\begin{center}
  This page intentionally left blank.
\end{center}
\vfill\null
\newpage\null