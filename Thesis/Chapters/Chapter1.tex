% -*- TeX-master: "../Thesis.tex" -*-


\chapter{Introduction} \label{cha:intro}

\epigraphhead[75]{
  \epigraph{\itshape Begin at the beginning, the King said
    gravely, ``and go on till you come to the end: then stop.''}
  {---\textsc{Lewis Carroll}\\ \textit{Alice in Wonderland}}
}

\lettrine{A}{rtificial} Intelligence es uno de los temas relacionados con
\gls{cs} que más están de actualidad\footnote{En \today, fecha actual del
  documento.} en los últimos años. Ha cobrado gran importancia debido
principalmente a sus aplicaciones en el mundo industrial y cotidiano, como
puede ser el de la condución autónoma.

Es un area de investigación en la que, curiosamente, no existe una definición
precisa y universalmente aceptada por la comunidad de investigadores y
desarrolladores que trabajan cada día en el ámbito de la
\gls{ai}. \myCite{nilsson09:quest_artif_intel} provee una definición útil:

\begin{quoteBox}
  Artificial intelligence is that activity devoted to making machines
  intelligent, and intelligence is that quality that enables an entity to
  function appropriately and with foresight in its environment.
  \tcblower
  ---Nils J. Nilsson
\end{quoteBox}

Dentro del amplio mundo de la \gls{ai} se encuentra \gls{dl} (que es parte de
un ámbito más amplio llamado \gls{ml}). \gls{dl} consiste en aprender de
datacon el uso de redes neuronales profundas. Estos datos pueden ser de
diferente índole: imágenes, texto, audio, video, etc. Esta tesis mezclará tres
de estos medios (audio \(\rightarrow\) texto e imagen), por lo que formará
parte del area conocida como \emph{multimodal learning}. A continuación en la
\vref{sec:description} se describirá el trabajo de esta tesis: \gls{rec}.


\section{Description and motivation} \label{sec:description}

\gls{rec} is the task of, given a linguistic phrase (referring expression) and
an image, generate binary masks for the object which the phrase refers to. Este
tipo de tareas quedan enmarcadas dentro del campo del aprendizaje multimodal:
en la intersección entre computer vision y \gls{nlp}.

\begin{figure}[ht]
  \centering
  \begin{subfigure}[t]{.32\textwidth}
    \centering
    \caption{Man with cap.}
    \includegraphics[width=\textwidth]{Images/Man with cap.jpg}
  \end{subfigure}\hfill
  \begin{subfigure}[t]{.32\textwidth}
    \centering
    \caption{Laptop on the right.}
    \includegraphics[width=\textwidth]{Images/Laptop on the right.jpg}
  \end{subfigure}\hfill
  \begin{subfigure}[t]{.32\textwidth}
    \centering
    \caption{Army officer white suit.}
    \includegraphics[width=\textwidth]{Images/Army officer.jpg}
  \end{subfigure}
  \caption[Examples of \acl*{rec}]{Examples of \acl{rec}. Como se puede ver,
    podemos referirnos a objetos de la imagen con \gls{re} en lenguaje natural
    y se produce la segmentación.}
  \label{fig:demo}
  \source{Created by the author}
\end{figure}

En la \vref{fig:demo} podemos ver unos ejemplos de este tipo de tareas. Como
vemos, la entrada consistirá en dos entidades: una \gls{re} y una imagen. El
modelo será el encargado de generar la segmentación del objeto al que la frase
se refiere. Vemos que las \gls{re} mostradas son diversos tipos: relación del
objeto a segmentar con \emph{otro objeto} (``man with a cap''), tipo de objeto
+ \emph{posicionamiento} relativo (``laptop on the right'') y descripción del
objeto + \emph{color} distintitvo (``army officer white suit'').

De manera opcional esta tarea puede ser explotada y ampliada de diversas
maneras. Entre ellas ampliando el conjunto de entrada y el de salida que el
modelo puede obtener.
\begin{itemize}
  \item \textbf{Entrada}. Podemos proponer modelos más generales que sean capaz
  de entender la \gls{re} desde audio, sin necesidad de tener que introducir la
  frase de manera manual. Así mismo, se puede ampliar el modelo aceptando
  además de imágenes, vídeos. En esta tesis se trabajará en la parte del
  \emph{audio}.
  \item \textbf{Salida}. La salida también puede ser ampliada generando, además
  de la máscara binaria de la segmentación, una bounding box.
\end{itemize}

Este trabajo añadirá una gran facilidad principalmente en la interacción
human-computer, por lo que presenta un gran interés práctico. Se discutirán
diferentes aplicaciones de este modelo en la \vref{sec:aplicaciones}.

\subsection{Objectives}

Esta tesis de investigación tiene diferentes objetivos diferenciados. El
primero de ellos es el de aprender el funcionamiento de la \gls{ai}, sin ningún
conocimiento previo. Principalmente en el ámbito del \gls{dl}, que es donde el
modelo presentado queda encajado. Es fundamental conocer los fundamentos de los
modelos neuronales para poder entender en profundidad cómo es posible resolver
el problema de \gls{rec}.

\begin{itemize}
  \item \textbf{Model}. Search for a model that performs well in \gls{rec},
  estudiar su funcionamiento y tratar de mejorarlo.
  \item \textbf{Visualization}. Presentar una interfaz web interactiva en la
  cual se presenten los resultados del trabajo. Aquí será de vital importancia
  que el modelo funcione suficientemente rápido.
  \item \textbf{Academic}. Escribir esta memoria y presentar el trabajo.
\end{itemize}

Para conseguir hacer realidad estos objetivos, muchos otros objetivos
intermedios han tenido que ser alcanzados. Entre ellos el de leer, comprender y
analizar los resultados \gls{sota} de la literatura reciente. Además de ello,
ha sido necesario mejorar las programming skills y aprender lenguajes de
programación nuevos, tanto para el desarrollo del modelo en Pytorch, como para
el desarrollo de la interfaz web: frontend con HTML, CSS, JS y backend con PHP
y Python.


\section{Aplicaciones} \label{sec:aplicaciones}

El uso de referring expression comprehension puede tener aplicaciones de
diversa índole. En los últimos años están cobrando una gran importancia la
robótica y la domótica. Este trabajo facilita la interacción entre humano y
robot/ordenador. Por ejemplo, podría facilitar la comprensión de órdenes de un
humano por parte de una máquina.

\subsubsection{Teóricas}
La creación y estudio de modelos en el ámbito del aprendizaje multimodal usando
deep learning puede terminar teniendo aplicaciones en ámbitos diferentes. La
transferencia de conocimiento entre ámbitos en la \gls{ai} es algo típico:
muchas veces \gls{cv} y \gls{nlp} terminan compartiendo técnicas similares.

En este caso concreto, tenemos precisamente la interacción entre modelos para
lenguaje y para visión. Esto podría ser útil en el desarrollo de nuevos modelos
en el futuro en el ámbito del aprendizaje multimodal.

\subsubsection{Industria}
En el ámbito industrial, el modelo aquí presentado podría tener aplicaciones en
diversos ámbitos. Podrían facilitar la interacción entre operario/máquina,
mejorando así la eficiencia de una determinada empresa y optimizando los
procesos.

\begin{figure}[ht]
  \centering
  \includegraphics[width=.75\textwidth]{Images/Robots.png}
  \caption[Robots en fábrica de automóviles]{Robots en funcionamiento dentro de
    una fábica de automóviles. Estos procesos suelen estar completamente
    automizados y controlados en tiempo real.}
  \label{fig:robots}
  \source{From \cite{online21:china_no}}
\end{figure}

Entre ellos podría ser el del mundo automovilístico, como se
muestra en la \vref{fig:robots}, donde varios brazos robóticos operan sobre un
vehículo en fabricación. Poder referirse a objetos/partes del vehículo usando
frases lingüisticas sería de gran utilidad. Por ejemplo, un operario podría ver
de manera visual que una de las soldaduras está mal realizada y ordenar al
brazo robótico rehacerla con una \gls{re} del tipo: ``soldadura inferior de la
puerta delantera derecha''.

\begin{remarkBox}
  Para algunos puede existir preocupación sobre si este tipo de aplicaciones de
  la inteligencia artificial pueden ser perjudicial para el futuro profesional
  de la sociedad en su conjunto. Es un debate abierto, pero, en palabras de
  \citeauthor*{contributor18:artif_intel_will_replac_tasks_not_jobs}
  \cite{contributor18:artif_intel_will_replac_tasks_not_jobs}, ``\itshape
  Artificial Intelligence Will Replace Tasks, Not Jobs''.
\end{remarkBox}

Este es tan solo un ejemplo de las muchas aplicaciones que podrían tener estos
modelos en la industria. La inmensa mayoría de los sectores se podrían
beneficiar con la implementación de sistemas de interacción entre sus operarios
y sus máquinas por medio de voz y usando \gls{re}.

\subsubsection{Domótica y \acs*{iot}}
En el mundo de la domótica y de \gls{iot}, tan de moda a día de hoy, también
podría ser de utilidad el tema que concierne esta tesis. Principalmente por la
facilidad que añade a la interacción entre máquinas y humanos.

\begin{figure}[ht]
  \centering
  \begin{subfigure}[t]{.55\textwidth}
    \centering
    \caption{Cubos de diferentes colores.}
    \includegraphics[height=4.5cm]{Images/Robot.jpg}
  \end{subfigure}\hfill
  \begin{subfigure}[t]{.4\textwidth}
    \centering
    \caption{Alimentos y medicamentos.}
    \includegraphics[height=4.5cm]{Images/Robot2.jpg}
  \end{subfigure}
  \caption[Ejemplos de aplicaciones en robótica]{Ejemplos de aplicaciones en
    robótica. Pueden facilitar muchas interacciones y tareas del día a día.}
  \label{fig:robot}
  \source{From \cite{limited21:china_arduin_robot_arm} and
    \cite{iam21:futur_mater_handl}}
\end{figure}

En la \vref{fig:robot} podemos ver algunos ejemplos en los que se podría
facilitar la interacción en tareas del día a día. En ella se ve un robot
cogiendo cubos de diferentes colores y otro brazo robótico cogiendo alimentos y
medicamentos de una estantería y apilando los seleccionados en una caja.

No tan solo por la facilidad que esto podría añadir a la vida de muchas
personas, sino también por el interés añadido que supondría para personas con
discapacidad. Por ejemplo, una persona con algún tipo de dicapacidad física (o
una persona mayor), podría necesitar ayuda para elegir productos en un
supermercado. En este caso, un brazo robótico podría ayudarle, y este trabajo
serviría de enlace entre los dos y facilitaría la interacción. La persona
podría referirse a los productos que desea adquirir usando simplemente la voz y
refiriendose a ello en un lenguaje \emph{natural}.

\subsubsection{Seguridad}
Otro posible aplicación sería la usado por la policía en el control de la
seguridad en las carreteras. En la \vref{fig:dgt} podemos ver uno de estos
drones.

\begin{figure}[ht]
  \centering
  \includegraphics[width=.75\textwidth]{Images/Drone.jpg}
  \caption[Drones empleados para la seguridad vial]{Imagen de uno de los drones
    que pueden ser empleados por las autoridades con el fin de garantizar la
    seguridad vial.}
  \label{fig:dgt}
  \source{TODO}
\end{figure}

Este tipo de drones podrían incorporar sistemas como el presentado en este
trabajo fin de grado para poder realizar seguimiento de vehículos mediante
comandos por voz. Por ejemplo se podrían usar estructuras sintácticas
compuestas por acción y \gls{re}, como podrían ser las siguientes.

\begin{itemize}
  \item \textbf{Acciones}. Diferentes acciones podrían ser deseadas para
  controlar y garantizar la seguridad vial como: seguir, grabar, tomar
  velocidad, etc.
  \item \textbf{\gls*{re}}. Se correspondería con frases linguisticas que
  identificaran al vehículo a estudiar o al infractor al que perseguir: coche
  azul/negro/rojo/\ldots, camión grande, coche deportivo, furgoneta de la
  derecha, etc.
\end{itemize}

De esta manera, combinaciones como: ``toma velocidad del coche negro'' o
``graba al coche deportivo'' podrían ayudar al control de la seguridad en las
carreteras. Este trabajo, como ya se ha comentado anteriormente, se centra en
\gls{rec} y no en el uso que se hace de esa comprehension a posteriori.


\section{Overview of thesis}

Aquí se presentará una descripción de los diferentes capítulos que componen
esta tesis.

\begin{description}
  \item[Chapter 1] Es este capítulo y se trata de un capítulo introductorio a
  nivel general de la materia que se tratará en esta tesis. Se tratarán la
  motivación detrás de este trabajo, sus objetivos y sus posibles
  aplicaciones. \chapRef{cha:intro}
  \item[Chapter 2] Este capítulo se centrará en los fundamentos teóricos
  generales en el ámbito de la \gls{ai}, de manera que los capítulos siguientes
  puedan entenderse. \chapRef{cha:theory}
  \item[Chapter 3] Tratará el tema central de esta tesis. Se formulará de
  manera concreta el problema, se presentarán el dataset que se usará y las
  técnicas de evaluación y se discutirán los modelos existentes
  actualmente. \chapRef{cha:rec}
  \item[Chapter 4] Este capítulo introducirá el modelo concreto usado, como se
  ha entrenado, las diferentes versiones que se han estudiado del mismo, cómo
  se ha comportado y qué resultados se han obtenido. \chapRef{cha:model}
  \item[Chapter 5] Presentará todo el código desarrollado para conseguir una
  aplicación web interactiva con la que se permitirá evaluar y validar el
  modelo de manera sencilla. \chapRef{cha:web}
  \item[Chapter 6] TODO. \chapRef{cha:analysis}
  \item[Chapter 7] Se resumirá de manera concisa la tesis, se discutirán los
  resultados obtenidos, se presentará una conclusión global y se aportarán
  futuras lineas de investigación. \chapRef{cha:concl}
\end{description}

También se incluirá material suplementario en los anexos.

\begin{description}
  \item[Appendix A] Hey. \chapRef{cha:introduction}
  \item[Appendix B] He. \chapRef{cha:license}
\end{description}

También comentar que se pueden encontrar al final las referencias, una lista de
acrónimos y un índice alfabético.

\subsubsection{How to read this thesis}
A lo largo de la tesis se usarán tres tipos de cajas para incluir contenido
extra: \emph{quote} box, \emph{example} box and \emph{remark} box.

\begin{quoteBox}
  This will be a quote box.
  \tcblower
  ---Author Name
\end{quoteBox}

\begin{exampleBox}
  This will be an example box.
\end{exampleBox}

\begin{remarkBox}
  This will be a remark/alert box.
\end{remarkBox}

Todas las figuras que aparecen en esta tesis muestran la fuente de la cuál se
han extraído o, en su defecto, han sido creadas por el autor.
