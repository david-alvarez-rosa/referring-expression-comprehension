% -*- TeX-master: "../Thesis.tex" -*-


\chapter{Introduction}

\epigraphhead[75]{
  \epigraph{\itshape Begin at the beginning, the King said
    gravely, ``and go on till you come to the end: then stop.''}
  {---Lewis Carroll\\ \textit{Alice in Wonderland}}
}


\lettrine{R}{eferring} expression comprehension is the task of, given a
linguistic phrase (referring expression) and an image, generate binary masks
for the object which the phrase refers to. Este tipo de tareas quedan
enmarcadas dentro del campo del aprendizaje multimodal: en la intersección
entre computer vision y \gls{nlp}.

\begin{figure}[ht]
  \centering
  \missingfigure{Hey there!}
  \caption[TODO]{TODO: ejemplos de comprehension}
  \label{fig:demo}
\end{figure}

En la \vref{fig:demo} podemos ver unos ejemplos de este tipo de tareas.
\todo{explicar aquí los ejemplos que se muestran}.


\section{Objectives}

Este trabajo fin de grado tiene varios objetivos diferenciados.


\section{Aplicaciones}

El uso de referring expression comprehension puede tener aplicaciones de
diversa índole. En los últimos años están cobrando una gran importancia la
robótica y la domótica. Este trabajo facilita la interacción entre humano y
robot/ordenador. Por ejemplo, podría facilitar la comprensión de órdenes de un
humano por parte de una máquina. Una posible aplicación sería la mostrada en la
\vref{fig:robot}

\begin{figure}[ht]
  \centering
  \missingfigure{Show a figure of a robot picking objects}
  \caption[TODO]{TODO: ejemplos de comprehension}
  \label{fig:robot}
\end{figure}

Otro posible aplicación sería la usado por la policía en el control de la
seguridad en las carreteras. En la \vref{fig:dgt} podemos ver uno de los drones
usados actualmente por\ldots\ldots

\begin{figure}[ht]
  \centering
  \missingfigure{Mostra una figura de un dron siguiendo a coches, de la DGT.}
  \caption[TODO]{TODO}
  \label{fig:dgt}
\end{figure}

\todo{Una vez puesta la imagen revisar este párrafo.}

Este tipo de drones podrían incorporar sistemas como el presentado en este
trabajo fin de grado para poder realizar seguimiento de vehículos mediante
comandos por voz. Por ejemplo se podrían usar estructuras sintácticas
compuestas por acción y \gls{re}, como podrían ser las siguientes.

\begin{itemize}
  \item \textbf{Acciones}: seguir, grabar, tomar velocidad, etc.
  \item \textbf{\gls*{re}}: coche azul/negro/rojo/\ldots, camión grande, coche
  deportivo, furgoneta de la derecha, etc.
\end{itemize}

De esta manera, combinaciones como: ``toma velocidad del coche negro'' o
``graba al coche deportivo'' podrían ayudar al control de la seguridad en las
carreteras. Este trabajo, como ya se ha comentado anteriormente, se centra en
comprender las \gls{re}.


\section{State of the Art}

Hay diferentes


\section{Contributions}
