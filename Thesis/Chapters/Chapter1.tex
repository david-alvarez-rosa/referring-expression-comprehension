% -*- TeX-master: "../Thesis.tex" -*-


\chapter{Introduction}\label{cha:intro}

\begin{textblock*}{.4cm}[1, .5](.95\paperwidth, .57721566490153\paperheight)
  \noindent\resizebox{.4cm}{!}{\(\gamma\)}
\end{textblock*}

\epigraphhead[75]{
  \epigraph{\itshape Begin at the beginning, the King said gravely, ``and go on
    till you come to the end: then stop.''}
  {---\textsc{Lewis Carroll} \\
    \textit{Alice in Wonderland}}}


\lettrine{A}{rtificial Intelligence} (\acs{ai})\index{Artificial intelligence}
is one of the topics related to \gls{cs} that are most current\footnote{On
  \today, current date of the document.} in recent years. It has gained great
importance mainly due to its applications in industry and everyday world, such
as autonomous driving.

It is an area of research in which, curiously, there is no precise definition
universally accepted by the community of researchers and developers who work
every day in the field of \gls{ai}. \myCite{nilsson09:quest_artif_intel},
however, provides a useful definition:
\begin{quoteBox}
  Artificial intelligence is that activity devoted to making machines
  intelligent, and intelligence is that quality that enables an entity to
  function appropriately and with foresight in its environment.
  \tcblower\quotecite{nilsson09:quest_artif_intel}
\end{quoteBox}

In other words, \gls{ai} is a very broad concept that encompasses all those
systems that perform tasks that can be considered intelligent. Within this
broad world of \gls{ai} is the scope of \gls{ml}\index{Machine
  learning}. \Gls{ml} is the set of computational algorithms that are
responsible for automatically improving models through experience and with the
use of \emph{data}. \Gls{ml} algorithms build models using sample data to be
able to make predictions or make decisions in future new situations without
being explicitly programmed for it.

Within this scope appear the \gls{ann}, which are the basis of a large family
of \gls{ml} methods called \gls{dl}\index{Deep learning}. The adjective
``deep'' arises from the fact that neural models make use of multiple layers in
the network. The models presented in this work will be part of precisely this
area, as will be seen later (see \vref{cha:models}).

Furthermore, this work can be classified according to the type of data used, as
a model of \emph{multimodal learning}\index{Multimodal!learning}. Data can be
of a different nature: images, text, audio, video, etc. This thesis will mix
three of these types (audio \(\rightarrow\) text and image). Next in
\vref{sec:description} the work of this thesis will be described.



\section{Description and Motivation}\label{sec:description}

\gls{rec} is the task of, given a \gls{re} (is a linguistic phrase) and an
image, generate binary masks for the object which the phrase refers to. This
type of task is framed within the field of multimodal learning: at the
intersection between computer vision and \gls{nlp}.

\begin{figure}[ht]
  \centering
  \begin{subfigure}[t]{.32\textwidth}
    \centering
    \caption{Man with cap.}
    \includegraphics[width=\textwidth]{Images/Man with cap.jpg}
  \end{subfigure}\hfill
  \begin{subfigure}[t]{.32\textwidth}
    \centering
    \caption{Laptop on the right.}
    \includegraphics[width=\textwidth]{Images/Laptop on the right.jpg}
  \end{subfigure}\hfill
  \begin{subfigure}[t]{.32\textwidth}
    \centering
    \caption{Army officer white suit.}
    \includegraphics[width=\textwidth]{Images/Army officer.jpg}
  \end{subfigure}
  \caption[Examples of \glsentrylong{rec}]{Examples of \acl{rec}. Como se puede
    ver, podemos referirnos a objetos de la imagen con \gls{re} en lenguaje
    natural y se produce la segmentación. Figures created by the author
    (all).}\label{fig:demo}
\end{figure}

In \vref{fig:demo} we can see some examples of this type of task. As we can
see, the input will consist of two entities: one \gls{re} and an image. The
model will be in charge of generating the segmentation of the object to which
the phrase refers. We see that the \gls{re} shown are different types: relation
of the object to segment with \emph{other object} (\re{man with a cap}), type
of object + relative \emph{positioning} (\re{laptop on the right}) and object
description + \emph{color} distinctive (\re{army officer white suit}).

Optionally this task can be exploited and expanded in various ways. Among them,
expanding the input and output set that the model can obtain.
\begin{itemize}
  \item \textbf{Input.} We can propose more general models that are capable
  of understanding \gls{re} from audio, without having to enter the phrase
  manually. Likewise, the model can be expanded accepting in addition to
  images, videos. In this thesis we will work on the part of \emph{audio}.
  \item \textbf{Output.} The output can also be extended by generating, in
  addition to the binary mask of the segmentation, a bounding box.
\end{itemize}

This work will add a great facility mainly in the human-computer interaction,
so it is of great practical interest. Different applications of this model will
be discussed in \vref{sec:applications}.


\subsection{Objectives}

This research thesis has different different objectives. The first one is to
\emph{learn} the operation of \gls{ai}, without any prior knowledge. Mainly in
the area of \gls{dl}, which is where the presented model fits. It is essential
to know the fundamentals of neural models to be able to understand in depth how
it is possible to solve the problem of \gls{rec}. This \emph{learning}
objective can be divided into learning the \gls{dl} fundamentals and
understanding the state-of-the-art papers for solving the task of \gls{rec}.

With this knowledge, we will proceed to \emph{modeling}, the search for models
that work well in \gls{rec} and in \gls{stt}. Its way of operation and how to
improve it will be studied. For this, the Python programming language has been
used with the Pytorch framework. All these results will be collected on an
interactive website for \emph{visualization}. Before working on web
development, it will be necessary to learn about frontend web programming
languages (HTML, CSS, JS), languages to develop the \gls{api} in the back end
(PHP) and use of web servers (Apache).

Finally, there are the \emph{academic} objectives related to this bachelor's
thesis. These include the writing of this report, and the creation and
preparation of the presentation.


\section{Applications}\label{sec:applications}

The use of \gls{rec} can have applications of various kinds. In recent years,
robotics and home automation are gaining great importance. This work enhances
the interaction between human and robot/computer. For example, it could make it
easier for a machine to understand commands from a human. The possible
applications of this work have been divided into four large groups:
theoretical, industry, domotics and \acs{iot}, and security.

\subsubsection{Theoretical}

The creation and study of models in the field of multimodal
learning\index{Multimodal!learning} using deep learning can end up having
applications in different fields. Knowledge transfer between fields in \gls{ai}
is typical: many times \gls{cv} and \gls{nlp} end up sharing similar
techniques.

In this specific case, we have precisely the interaction between models for
language and for vision. This could be useful in the development of new models
in the future in the field of multimodal
learning\index{Multimodal!learning}. In addition, the creation of a website
with which to interact with the different versions of the models, constitutes a
tool for the evaluation of models. The general public is provided with a simple
tool with which to perform multiple qualitative evaluations of the functioning
of complex neural models.

\subsubsection{Industry}

In the industrial field, the model presented here could have applications in
various fields. They could facilitate the interaction between operator/machine,
thus improving the efficiency of a certain company and optimizing processes.

\begin{figure}[ht]
  \centering
  \includegraphics[width=.75\textwidth]{Images/Robots.png}
  \caption[Robots in automobile factory]{Robots in operation within an
    automobile factory. These processes are usually completely automated and
    controlled in real time. From
    \figcite{online21:china_no}.}\label{fig:robots}
\end{figure}

Among them could be that of the automotive world, as shown in
\vref{fig:robots}, where several robotic arms operate on a vehicle being
manufactured. Being able to refer to objects/parts of the vehicle using
linguistic phrases would be very useful. For example, an operator could
visually see that one of the welds is badly made and order the robotic arm to
redo it with a \gls{re} of the type: \re{lower right front door weld}.

\begin{remarkBox}
  For some, there may be concern about whether this type of artificial
  intelligence applications could be detrimental to the professional future of
  society as a whole. It's an open debate, but, in the words of
  \citeauthor*{contributor18:artif_intel_will_replac_tasks_not_jobs}
  \cite{contributor18:artif_intel_will_replac_tasks_not_jobs},
  ``\textit{Artificial Intelligence Will Replace Tasks, Not Jobs}''.
\end{remarkBox}

This is just one example of the many applications that these models could have
in the industry. The vast majority of sectors could benefit from the
implementation of interaction systems between their operators and their
machines by means of voice and using \gls{re}.

\subsubsection{Domotics and \glsentryshort{iot}}

In the world of domotics and \gls{iot}, so fashionable today, the topic that
concerns this thesis could also be useful. Mainly for the ease it adds to the
interaction between machines and humans.

\begin{figure}[ht]
  \centering
  \begin{subfigure}[t]{.55\textwidth}
    \centering
    \caption{Cubes of different colors.}
    \includegraphics[height=4.5cm]{Images/Robot.jpg}
  \end{subfigure}\hfill
  \begin{subfigure}[t]{.4\textwidth}
    \centering
    \caption{Food and medications.}
    \includegraphics[height=4.5cm]{Images/Robot2.jpg}
  \end{subfigure}
  \caption[Examples of applications in robotics]{Examples of applications in
    robotics. They can facilitate many interactions and tasks of day to
    day. From \figcite{limited21:china_arduin_robot_arm} (left) and from
    \figcite{iam21:futur_mater_handl} (right).}\label{fig:robot}
\end{figure}

In \vref{fig:robot} we can see some examples in which interaction in day-to-day
tasks could be facilitated. In this figure, a robot is seen taking cubes of
different colors and another robotic arm taking food and medicine from a shelf
and stacking the selected ones in a box.

Not only because of the ease that this could add to the lives of many people,
but also because of the added interest it would bring to people with
disabilities. For example, a person with some type of physical disability (or
an elderly person), might need help choosing products in a supermarket. In this
case, a robotic arm could help him, and this work would serve as a link between
the two and facilitate interaction. The person could refer to the products they
want to purchase by simply using their voice and referring to it in
\emph{natural} language.

\subsubsection{Security}

Another possible application would be that used by the police to control road
safety. In \vref{fig:dgt} we can see one of these drones.

\begin{figure}[ht]
  \centering
  \includegraphics[width=.75\textwidth]{Images/Drone.jpg}
  \caption[Drones employed for road safety]{Image of one of the drones that can
    be employed by the authorities in order to guarantee road safety. From
    TODO.}\label{fig:dgt}
\end{figure}

This type of drones could incorporate systems such as the one presented in this
final degree project to be able to track vehicles using voice commands. For
example, syntactic structures composed of action and \gls{re} could be used,
such as the following.

\begin{itemize}
  \item \textbf{Actions.} Different actions could be desired to control and
  guarantee road safety such as: follow, record, speed up, etc.
  \item \textbf{\gls{re}.} It would correspond to linguistic phrases that
  identify the vehicle to be studied or the offender to be pursued: blue/
  black/red/\ldots car, large truck, sports car, van on the right, etc.
\end{itemize}

In this way, combinations such as: ``speed up the black car'' or ``record the
sports car'' could help control road safety. This work, as has already been
commented previously, focuses on \gls{rec} and not on the use made of this
comprehension after.



\section{Thesis Overview}

Here a description of the different chapters that make up this thesis will be
presented.

\begin{description}
  \item [Chapter 1] It is this chapter and it is an introductory chapter at a
  general level of the subject that will be dealt with in this thesis. The
  motivation behind this work, its objectives and its possible applications
  will be discussed. \chapRef{cha:intro}
  \item [Chapter 2] This chapter will focus on the general theoretical
  foundations in the field of \gls{ai}, so that the following chapters can be
  understood. \chapRef{cha:theory}
  \item [Chapter 3] It will deal with the central theme of this thesis. The
  problem will be formulated in a concrete way, the existing datasets and the
  evaluation techniques for this task will be presented and the existing
  state-of-the-art models will be discussed. \chapRef{cha:rec}
  \item [Chapter 4] This chapter will introduce the concrete models used both
  for the task of \gls{rec} and for \gls{stt}. It will be discussed how it has
  been trained, the different versions that have been studied (model
  iterations) and how it has behaved. \chapRef{cha:models}
  \item [Chapter 5] For the selected model, an evaluation will be
  made---quantitative and qualitative---of the results. Likewise, this model
  will be compared with current state-of-the-art works. \chapRef{cha:results}
  \item [Chapter 6] It will present all the code developed to achieve an
  interactive web application with which it will be possible to evaluate and
  validate the model in a simple way. Both frontend and backend will be
  discussed. \chapRef{cha:web}
  \item [Chapter 7] In this chapter the project will be analyzed from a
  management point of view. The different activities carried out and its
  programming will be summarized. Finally, an analysis of economic cost and
  environmental impact will be made. \chapRef{cha:analysis}
  \item [Chapter 8] The thesis will be briefly summarized, the results obtained
  will be discussed, a global conclusion will be presented and future lines of
  research will be provided. \chapRef{cha:concl}
\end{description}

Supplementary material will also be included in the appendices.

\begin{description}
  \item [Appendix A] TODO Hey. \chapRef{cha:introduction}
  \item [Appendix B] TODO b. \chapRef{cha:license}
  \item [Appendix C] TODO c. \chapRef{cha:showcase}
\end{description}

Finally, after the appendices, there is a \textsl{List of Acronyms}, the full
\textsl{Bibliography} and an \textsl{Alphabetical Index} to facilitate the
search for terms and topics.

\subsubsection{How to Read this Thesis}

Throughout the thesis, three types of boxes will be used to include extra
content: \emph{quote} box, \emph{example} box and \emph{remark} box.

\begin{quoteBox}
  This will be a quote box.
  \tcblower\textsc{Author Name}
\end{quoteBox}

\begin{exampleBox}
  This will be an example box.
\end{exampleBox}

\begin{remarkBox}
  This will be a remark/alert box.
\end{remarkBox}

All the figures that appear in this thesis show the source from which they have
been extracted or, failing that, they have been created by the author.
