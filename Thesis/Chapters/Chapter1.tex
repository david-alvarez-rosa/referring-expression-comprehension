% -*- TeX-master: "../Thesis.tex" -*-


\chapter{Introduction} \label{cha:intro}

\epigraphhead [75]{
  \epigraph{\itshape Begin at the beginning, the King said
    gravely, ``and go on till you come to the end: then stop.''}
  {---\textsc{Lewis Carroll} \ \ \textit{Alice in Wonderland}}
}

\lettrine{A}{rtificial} Intelligence is one of the topics related to
\gls{cs} most current \footnote{In \today, current date of
  document.} in recent years. It has become very important due
mainly to its applications in the industrial and everyday world, such as
it may be that of autonomous driving.

It is an area of   research in which, curiously, there is no definition
accurately and universally accepted by the research community and
developers who work every day in the field of
\gls{ai}. \myCite{nilsson09:quest_artif_intel} provides a useful definition:

\begin{quoteBox}
  Artificial intelligence is that activity devoted to making machines
  intelligent, and intelligence is that quality that enables an entity to
  function appropriately and with foresight in its environment.
  \tcblower
  --- Nils J. Nilsson
\end{quoteBox}

Within the broad world of \gls{ai} is \gls{dl} (which is part of
a broader scope called \gls{ml}). \gls{dl} consists of learning from
data with the use of deep neural networks. These data can be from
different nature: images, text, audio, video, etc. This thesis will mix three
of these media (audio \(\rightarrow\) text and image), so it will form
part of the area known as \emph{multimodal learning}. Next in the
\vref{sec:description} the work of this thesis will be described: \gls{rec}.


\section{Description and motivation} \label{sec:description}

\gls{rec} is the task of, given a linguistic phrase (\gls{re}) and an image,
generate binary masks for the object which the phrase refers to. This type of
tasks are framed within the field of multimodal learning: in the
intersection between computer vision and \gls{nlp}.

\begin{figure}[ht]
  \centering
  \begin{subfigure}[t]{.32\textwidth}
    \centering
    \caption{Man with cap.}
    \includegraphics[width=\textwidth]{Images/Man with cap.jpg}
  \end{subfigure}\hfill
  \begin{subfigure}[t]{.32\textwidth}
    \centering
    \caption{Laptop on the right.}
    \includegraphics[width=\textwidth]{Images/Laptop on the right.jpg}
  \end{subfigure}\hfill
  \begin{subfigure}[t]{.32\textwidth}
    \centering
    \caption{Army officer white suit.}
    \includegraphics[width=\textwidth]{Images/Army officer.jpg}
  \end{subfigure}
  \caption[Examples of \acl*{rec}]{Examples of \acl{rec}. Como se puede ver,
    podemos referirnos a objetos de la imagen con \gls{re} en lenguaje natural
    y se produce la segmentación.}
  \label{fig:demo}
  \source{Created by the author}
\end{figure}

In \vref{fig:demo} we can see some examples of this type of task. What
We see, the input will consist of two entities: a \gls{re} and an image. The
The model will be in charge of generating the segmentation of the object to which the phrase
it means. We see that the \gls{re} shown are different types: relation of the
object to segment with \emph{otro objeto} (`` man with a cap ''), type of object
+ \emph{posicionamiento} relative (`` laptop on the right '') and description of the
object + \emph{color} distinctive (`` army officer white suit '').

Optionally, this task can be exploited and expanded in various ways.
ways. Among them by expanding the input and output set that the
model can get.
\begin{itemize}
  \item \textbf{Entrada}. We can propose more general models that are capable
  to understand \gls{re} from audio, without having to enter the
  phrase manually. Likewise, the model can be extended by accepting
  in addition to images, videos. In this thesis we will work on the part of
  \emph{audio}.
  \item \textbf{Salida}. The output can also be extended by generating, in addition
  from the binary mask of the segmentation, a bounding box.
\end{itemize}

This work will add a great facility mainly in the interaction
human-computer, so it is of great practical interest. Will be discussed
different applications of this model in \vref{sec:aplicaciones}.

\subsection{Objectives}

This research thesis has different different objectives. The
The first of these is to learn the operation of \gls{ai}, without any
prior knowledge. Mainly in the field of \gls{dl}, which is where the
model presented is fitted. It is essential to know the fundamentals of
neural models to be able to understand in depth how it is possible to solve
the problem of \gls{rec}.

\begin{itemize}
  \item \textbf{Model}. Search for a model that performs well in \gls{rec},
  study its operation and try to improve it.
  \item \textbf{Visualization}. Present an interactive web interface on the
  which will present the results of the work. Here it will be of vital importance
  that the model works fast enough.
  \item \textbf{Academic}. Write this report and present the work.
\end{itemize}

To achieve these goals, many other goals
intermediates have had to be reached. Among them that of reading, understanding and
analyze the results \gls{sota} of the recent literature. Besides that,
it has been necessary to improve programming skills and learn languages   of
new programming, both for the development of the model in Pytorch, and for
the development of the web interface: frontend with HTML, CSS, JS and backend with PHP
and Python.


\section{Applications} \label{sec:aplicaciones}

The use of referring expression comprehension can have applications of
diverse nature. In recent years, the
robotics and home automation. This work facilitates the interaction between human and
robot / computer. For example, it could make it easier to understand orders from a
human on the part of a machine.

\subsubsection{Theoretical}
The creation and study of models in the field of multimodal learning using
deep learning can end up having applications in different domains. The
knowledge transfer between fields in \gls{ai} is typical:
many times \gls{cv} and \gls{nlp} end up sharing similar techniques.

In this specific case, we have precisely the interaction between models for
language and vision. This could be useful in the development of new models
in the future in the field of multimodal learning.

\subsubsection{Industry}
In the industrial field, the model presented here could have applications in
various fields. They could facilitate the interaction between operator / machine,
thus improving the efficiency of a certain company and optimizing the
processes.

\begin{figure}[ht]
  \centering
  \includegraphics[width=.75\textwidth]{Images/Robots.png}
  \caption[Robots en fábrica de automóviles]{Robots en funcionamiento dentro de
    una fábica de automóviles. Estos procesos suelen estar completamente
    automizados y controlados en tiempo real.}
  \label{fig:robots}
  \source{From \cite{online21:china_no}}
\end{figure}

Among them it could be that of the automotive world, as
shown in \vref{fig:robots}, where several robotic arms operate on a
vehicle in manufacture. Be able to refer to objects / parts of the vehicle using
Linguistic phrases would be of great use. For example, an operator might see
visually that one of the welds is incorrectly performed and order the
robotic arm remake it with a \gls{re} of the type: `` lower weld of the
right front door ''.

\begin{remarkBox}
  For some, there may be concern about whether this type of application
  artificial intelligence can be detrimental to the future professional
  of society as a whole. It is an open debate, but, in the words of
  \citeauthor *{contributor18: artif_intel_will_replac_tasks_not_jobs}
  \cite{contributor18:artif_intel_will_replac_tasks_not_jobs}, `` \itshape
  Artificial Intelligence Will Replace Tasks, Not Jobs. ''
\end{remarkBox}

This is just one example of the many applications that these could have.
models in the industry. The vast majority of sectors could be
benefit from the implementation of interaction systems between your operators
and their machines by voice and using \gls{re}.

\subsubsection{Domotics and \acs *{iot}}
In the world of home automation and \gls{iot}, so fashionable today, also
the subject that concerns this thesis could be useful. Mainly because of the
ease that adds to the interaction between machines and humans.

\begin{figure}[ht]
  \centering
  \begin{subfigure}[t]{.55\textwidth}
    \centering
    \caption{Cubos de diferentes colores.}
    \includegraphics[height=4.5cm]{Images/Robot.jpg}
  \end{subfigure}\hfill
  \begin{subfigure}[t]{.4\textwidth}
    \centering
    \caption{Alimentos y medicamentos.}
    \includegraphics[height=4.5cm]{Images/Robot2.jpg}
  \end{subfigure}
  \caption[Ejemplos de aplicaciones en robótica]{Ejemplos de aplicaciones en
    robótica. Pueden facilitar muchas interacciones y tareas del día a día.}
  \label{fig:robot}
  \source{From \cite{limited21:china_arduin_robot_arm} and
    \cite{iam21:futur_mater_handl}}
\end{figure}

In \vref{fig:robot} we can see some examples in which one could
facilitate interaction in day-to-day tasks. In it you see a robot
picking up different colored cubes and another robotic arm picking up food and
medications from a shelf and stacking the selected ones in a box.

Not only because of the ease that this could add to the lives of many
people, but also because of the added interest it would bring to people with
disability. For example, a person with some type of physical disability (or
an older person), you may need help choosing products in a
Supermarket. In this case, a robotic arm could help you, and this job
It would serve as a link between the two and facilitate interaction. Person
you could refer to the products you want to purchase by simply using your voice and
referring to it in language \emph{natural}.

\subsubsection{Security}
Another possible application would be that used by the police in the control of the
road safety. In \vref{fig:dgt} we can see one of these
drones.

\begin{figure}[ht]
  \centering
  \includegraphics[width=.75\textwidth]{Images/Drone.jpg}
  \caption[Drones empleados para la seguridad vial]{Imagen de uno de los drones
    que pueden ser empleados por las autoridades con el fin de garantizar la
    seguridad vial.}
  \label{fig:dgt}
  \source{TODO}
\end{figure}

This type of drones could incorporate systems like the one presented in this
final degree project to be able to track vehicles through
voice commands. For example syntactic structures could be used
composed of share and \gls{re}, such as the following.

\begin{itemize}
  \item \textbf{Acciones}. Different actions could be desired for
  control and guarantee road safety such as: follow, record, take
  speed, etc.
  \item \textbf{\gls*{re}}. It would correspond to linguistic phrases that
  identify the vehicle to study or the offender to pursue: car
  blue / black / red / \ldots, big truck, sports car, van
  right, etc.
\end{itemize}

In this way, combinations such as: `` take the speed of the black car '' or
`` record the sports car '' could help control security in
roads. This work, as mentioned above, focuses on
\gls{rec} and not in the use made of that comprehension a posteriori.


\section{Overview of thesis}

Here will be presented a description of the different chapters that make up
this thesis.

\begin{description}
  \item [Chapter 1] It is this chapter and it is an introductory chapter to
  general level of the subject that will be treated in this thesis. The
  motivation behind this work, its objectives and its possible
  Applications. \chapRef{cha:intro}
  \item [Chapter 2] This chapter will focus on the theoretical foundations
  in the field of \gls{ai}, so that the following chapters
  can be understood. \chapRef{cha:theory}
  \item [Chapter 3] It will deal with the central theme of this thesis. It will be formulated from
  concretely the problem, the dataset that will be used and the
  evaluation techniques and existing models will be discussed
  actually. \chapRef{cha:rec}
  \item [Chapter 4] This chapter will introduce the concrete model used, as
  has trained, the different versions that have been studied of it, how
  has behaved and what results have been obtained. \chapRef{cha:model}
  \item [Chapter 5] Will present all the code developed to achieve a
  interactive web application with which it will be possible to evaluate and validate the
  model in a simple way. \chapRef{cha:web}
  \item [Chapter 6] EVERYTHING. \chapRef{cha:analysis}
  \item [Chapter 7] The thesis will be summarized concisely, the
  results obtained, a global conclusion will be presented and
  future lines of research. \chapRef{cha:concl}
\end{description}

Supplementary material will also be included in the annexes.

\begin{description}
  \item [Appendix A] Hey. \chapRef{cha:introduction}
  \item [Appendix B] He. \chapRef{cha:license}
\end{description}

Also comment that you can find at the end the references, a list of
acronyms and an alphabetical index.

\subsubsection{How to read this thesis}
Throughout the thesis, three types of boxes will be used to include content
extra: \emph{quote} box, \emph{example} box and \emph{remark} box.

\begin{quoteBox}
  This will be a quote box.
  \tcblower
  --- Author Name
\end{quoteBox}

\begin{exampleBox}
  This will be an example box.
\end{exampleBox}

\begin{remarkBox}
  This will be a remark / alert box.
\end{remarkBox}

All the figures that appear in this thesis show the source from which
have been extracted or, failing that, have been created by the author.
