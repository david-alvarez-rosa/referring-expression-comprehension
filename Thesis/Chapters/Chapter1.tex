% -*- TeX-master: "../Thesis.tex" -*-


\chapter{Introduction}

\epigraphhead[75]{
  \epigraph{\itshape Begin at the beginning, the King said
    gravely, ``and go on till you come to the end: then stop.''}
  {---\textsc{Lewis Carroll}\\ \textit{Alice in Wonderland}}
}


\lettrine{R}{eferring} expression comprehension is the task of, given a
linguistic phrase (referring expression) and an image, generate binary masks
for the object which the phrase refers to. Este tipo de tareas quedan
enmarcadas dentro del campo del aprendizaje multimodal: en la intersección
entre computer vision y \gls{nlp}.

\begin{figure}[ht]
  \centering
  \missingfigure{Hey there!}
  \caption[TODO]{TODO: ejemplos de comprehension}
  \label{fig:demo}
\end{figure}

En la \vref{fig:demo} podemos ver unos ejemplos de este tipo de tareas.
\todo{explicar aquí los ejemplos que se muestran}.


\section{Description and motivation}

\subsection{Objectives}

Este trabajo fin de grado tiene varios objetivos diferenciados.

\todo{Objetivos de conocimiento de lo aprendido, objetivos de mejorar modelos
  existentes, objetivos de añadir herramientas de visualización, objetivos de
  eficiencia de hacer más rápido el procesado de imágenes, \ldots}


\section{Aplicaciones}

El uso de referring expression comprehension puede tener aplicaciones de
diversa índole. En los últimos años están cobrando una gran importancia la
robótica y la domótica. Este trabajo facilita la interacción entre humano y
robot/ordenador. Por ejemplo, podría facilitar la comprensión de órdenes de un
humano por parte de una máquina.

\subsubsection{Teóricas}
La creación y estudio de modelos en el ámbito del aprendizaje multimodal usando
deep learning puede terminar teniendo aplicaciones en ámbitos diferentes. La
transferencia de conocimiento entre ámbitos en la \gls{ai} es algo típico:
muchas veces \gls{cv} y \gls{nlp} terminan compartiendo técnicas similares.

En este caso concreto, tenemos precisamente la interacción entre modelos para
lenguaje y para visión. Esto podría ser útil en el desarrollo de nuevos modelos
en el futuro en el ámbito del aprendizaje multimodal.

\subsubsection{Industria}
En el ámbito industrial, el modelo aquí presentado podría tener aplicaciones en
diversos ámbitos. Podrían facilitar la interacción entre operario/máquina,
mejorando así la eficiencia de una determinada empresa y optimizando los
procesos.

\begin{figure}[ht]
  \centering
  \includegraphics[width=.75\textwidth]{Images/Robots.png}
  \caption[Robots en fábrica de automóviles]{Robots en funcionamiento dentro de
  una fábica de automóviles. Estos procesos suelen estar completamente
  automizados y controlados en tiempo real.}
  \label{fig:robots}
\end{figure}

Entre ellos podría ser el del mundo automovilístico, como se
muestra en la \vref{fig:robots}, donde varios brazos robóticos operan sobre un
vehículo en fabricación. Poder referirse a objetos/partes del vehículo usando
frases lingüisticas sería de gran utilidad.

\subsubsection{Domótica y \acs*{iot}}
En el mundo de la domótica y de \gls{iot}, tan de moda a día de hoy.

\begin{figure}[ht]
  \centering
  \begin{subfigure}[b]{.55\textwidth}
    \centering
    \includegraphics[height=4.5cm]{Images/Robot.jpg}
    \caption{Cubos de diferentes colores.}
  \end{subfigure}\hfill
  \begin{subfigure}[b]{.4\textwidth}
    \centering
    \includegraphics[height=4.5cm]{Images/Robot2.jpg}
    \caption{Alimentos y medicamentos.}
  \end{subfigure}
  \caption[Ejemplos de aplicaciones en robótica]{Ejemplos de aplicaciones en
    robótica. Pueden facilitar muchas interacciones y tareas del día a día.}
  \label{fig:robot}
\end{figure}

Robot cogiendo alimentos y medicamentos de una estantería y apilando los
seleccionados en una caja.


\subsubsection{Seguridad}

Otro posible aplicación sería la usado por la policía en el control de la
seguridad en las carreteras. En la \vref{fig:dgt} podemos ver uno de estos
drones.

\begin{figure}[ht]
  \centering
  \includegraphics[width=.75\textwidth]{Images/Drone.jpg}
  \caption[TODO]{TODO}
  \label{fig:dgt}
\end{figure}

Este tipo de drones podrían incorporar sistemas como el presentado en este
trabajo fin de grado para poder realizar seguimiento de vehículos mediante
comandos por voz. Por ejemplo se podrían usar estructuras sintácticas
compuestas por acción y \gls{re}, como podrían ser las siguientes.

\begin{itemize}
  \item \textbf{Acciones}: seguir, grabar, tomar velocidad, etc.
  \item \textbf{\gls*{re}}: coche azul/negro/rojo/\ldots, camión grande, coche
  deportivo, furgoneta de la derecha, etc.
\end{itemize}

De esta manera, combinaciones como: ``toma velocidad del coche negro'' o
``graba al coche deportivo'' podrían ayudar al control de la seguridad en las
carreteras. Este trabajo, como ya se ha comentado anteriormente, se centra en
comprender las \gls{re}.


\section{Overview of chapters}

Aquí se presentará una descripción de los diferentes capítulos que componen
esta tesis.

\begin{description}
  \item[Chapter 1] Hey
  \item[Chapter 2] Hey
  \item[Chapter 3] More hey.
  \item[Chapter 4] Hey
  \item[Chapter 5] More hey.
  \item[Chapter 6] Hey
  \item[Chapter 7] More hey.
\end{description}

También se incluirá material suplementario en los anexos.

\begin{description}
  \item[Appendix A] Hey
  \item[Appendix B] Hey
\end{description}

También comentar que se pueden encontrar al final las referencias, una lista de
acrónimos y un índice alfabético.
