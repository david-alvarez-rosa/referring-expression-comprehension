% -*- TeX-master: "../Thesis.tex" -*-


\chapter{Introduction} \label{cha:intro}

\epigraphhead[75]{
  \epigraph{\itshape Begin at the beginning, the King said
    gravely, ``and go on till you come to the end: then stop.''}
  {---\textsc{Lewis Carroll}\\ \textit{Alice in Wonderland}}
}


\lettrine{R}{eferring} expression comprehension is the task of, given a
linguistic phrase (referring expression) and an image, generate binary masks
for the object which the phrase refers to. Este tipo de tareas quedan
enmarcadas dentro del campo del aprendizaje multimodal: en la intersección
entre computer vision y \gls{nlp}.

\begin{figure}[ht]
  \centering
  \missingfigure{Hey there!}
  \caption[Examples of \acl*{rec}]{TODO: ejemplos de comprehension}
  \label{fig:demo}
\end{figure}

En la \vref{fig:demo} podemos ver unos ejemplos de este tipo de tareas.


\section{Description and motivation}

Mirar el capítulo de motivation del Jordi Fortuny.

\subsection{Objectives}

Este trabajo fin de grado tiene varios objetivos diferenciados.

Objetivos de conocimiento de lo aprendido, objetivos de mejorar modelos
existentes, objetivos de añadir herramientas de visualización, objetivos de
eficiencia de hacer más rápido el procesado de imágenes.


\section{Aplicaciones}

El uso de referring expression comprehension puede tener aplicaciones de
diversa índole. En los últimos años están cobrando una gran importancia la
robótica y la domótica. Este trabajo facilita la interacción entre humano y
robot/ordenador. Por ejemplo, podría facilitar la comprensión de órdenes de un
humano por parte de una máquina.

\subsubsection{Teóricas}
La creación y estudio de modelos en el ámbito del aprendizaje multimodal usando
deep learning puede terminar teniendo aplicaciones en ámbitos diferentes. La
transferencia de conocimiento entre ámbitos en la \gls{ai} es algo típico:
muchas veces \gls{cv} y \gls{nlp} terminan compartiendo técnicas similares.

En este caso concreto, tenemos precisamente la interacción entre modelos para
lenguaje y para visión. Esto podría ser útil en el desarrollo de nuevos modelos
en el futuro en el ámbito del aprendizaje multimodal.

\subsubsection{Industria}
En el ámbito industrial, el modelo aquí presentado podría tener aplicaciones en
diversos ámbitos. Podrían facilitar la interacción entre operario/máquina,
mejorando así la eficiencia de una determinada empresa y optimizando los
procesos.

\begin{figure}[ht]
  \centering
  \includegraphics[width=.75\textwidth]{Images/Robots.png}
  \caption[Robots en fábrica de automóviles]{Robots en funcionamiento dentro de
    una fábica de automóviles. Estos procesos suelen estar completamente
    automizados y controlados en tiempo real.}
  \label{fig:robots}
  \source{From \cite{online21:china_no}}
\end{figure}

Entre ellos podría ser el del mundo automovilístico, como se
muestra en la \vref{fig:robots}, donde varios brazos robóticos operan sobre un
vehículo en fabricación. Poder referirse a objetos/partes del vehículo usando
frases lingüisticas sería de gran utilidad. Por ejemplo, un operario podría ver
de manera visual que una de las soldaduras está mal realizada y ordenar al
brazo robótico rehacerla con una \gls{re} del tipo: ``soldadura inferior de la
puerta delantera derecha''.

\subsubsection{Domótica y \acs*{iot}}
En el mundo de la domótica y de \gls{iot}, tan de moda a día de hoy, también
podría ser de utilidad el tema que concierne esta tesis. Principalmente por la
facilidad que añade a la interacción entre máquinas y humanos.

\begin{figure}[ht]
  \centering
  \begin{subfigure}[b]{.55\textwidth}
    \centering
    \includegraphics[height=4.5cm]{Images/Robot.jpg}
    \caption{Cubos de diferentes colores.}
  \end{subfigure}\hfill
  \begin{subfigure}[b]{.4\textwidth}
    \centering
    \includegraphics[height=4.5cm]{Images/Robot2.jpg}
    \caption{Alimentos y medicamentos.}
  \end{subfigure}
  \caption[Ejemplos de aplicaciones en robótica]{Ejemplos de aplicaciones en
    robótica. Pueden facilitar muchas interacciones y tareas del día a día.}
  \label{fig:robot}
  \source{From \cite{limited21:china_arduin_robot_arm} and
    \cite{iam21:futur_mater_handl}}
\end{figure}

En la \vref{fig:robot} podemos ver algunos ejemplos en los que se podría
facilitar la interacción en tareas del día a día. En ella se ve un robot
cogiendo cubos de diferentes colores y otro brazo robótico cogiendo alimentos y
medicamentos de una estantería y apilando los seleccionados en una caja.

No tan solo por la facilidad que esto podría añadir a la vida de muchas
personas, sino también por el interés añadido que supondría para personas con
discapacidad. Por ejemplo, una persona con algún tipo de dicapacidad física (o
una persona mayor), podría necesitar ayuda para elegir productos en un
supermercado. En este caso, un brazo robótico podría ayudarle, y este trabajo
serviría de enlace entre los dos y facilitaría la interacción. La persona
podría referirse a los productos que desea adquirir usando simplemente la voz y
refiriendose a ello en un lenguaje \emph{natural}.

\subsubsection{Seguridad}
Otro posible aplicación sería la usado por la policía en el control de la
seguridad en las carreteras. En la \vref{fig:dgt} podemos ver uno de estos
drones.

\begin{figure}[ht]
  \centering
  \includegraphics[width=.75\textwidth]{Images/Drone.jpg}
  \caption[Drones empleados para la seguridad vial]{Imagen de uno de los drones
    que pueden ser empleados por las autoridades con el fin de garantizar la
    seguridad vial.}
  \label{fig:dgt}
  \source{TODO}
\end{figure}

Este tipo de drones podrían incorporar sistemas como el presentado en este
trabajo fin de grado para poder realizar seguimiento de vehículos mediante
comandos por voz. Por ejemplo se podrían usar estructuras sintácticas
compuestas por acción y \gls{re}, como podrían ser las siguientes.

\begin{itemize}
  \item \textbf{Acciones}. Diferentes acciones podrían ser deseadas para
  controlar y garantizar la seguridad vial como: seguir, grabar, tomar
  velocidad, etc.
  \item \textbf{\gls*{re}}. Se correspondería con frases linguisticas que
  identificaran al vehículo a estudiar o al infractor al que perseguir: coche
  azul/negro/rojo/\ldots, camión grande, coche deportivo, furgoneta de la
  derecha, etc.
\end{itemize}

De esta manera, combinaciones como: ``toma velocidad del coche negro'' o
``graba al coche deportivo'' podrían ayudar al control de la seguridad en las
carreteras. Este trabajo, como ya se ha comentado anteriormente, se centra en
\gls{rec} y no en el uso que se hace de esa comprehension a posteriori.


\section{Overview of thesis}

Aquí se presentará una descripción de los diferentes capítulos que componen
esta tesis.

\begin{description}
  \item[Chapter 1] Es este capítulo y se trata de un capítulo introductorio a
  nivel general de la materia que se tratará en esta tesis. Se tratarán la
  motivación detrás de este trabajo, sus objetivos y sus posibles
  aplicaciones. \chapRef{cha:intro}
  \item[Chapter 2] Este capítulo se centrará en los fundamentos teóricos
  generales en el ámbito de la \gls{ai}, de manera que los capítulos siguientes
  puedan entenderse. \chapRef{cha:theory}
  \item[Chapter 3] Tratará el tema central de esta tesis. Se formulará de
  manera concreta el problema, se presentarán el dataset que se usará y las
  técnicas de evaluación y se discutirán los modelos existentes
  actualmente. \chapRef{cha:rec}
  \item[Chapter 4] Este capítulo introducirá el modelo concreto usado, como se
  ha entrenado, las diferentes versiones que se han estudiado del mismo, cómo
  se ha comportado y qué resultados se han obtenido. \chapRef{cha:model}
  \item[Chapter 5] Presentará todo el código desarrollado para conseguir una
  aplicación web interactiva con la que se permitirá evaluar y validar el
  modelo de manera sencilla. \chapRef{cha:web}
  \item[Chapter 6] Se resumirá de manera concisa la tesis, se discutirán los
  resultados obtenidos, se presentará una conclusión global y se aportarán
  futuras lineas de investigación. \chapRef{cha:concl}
\end{description}

También se incluirá material suplementario en los anexos.

\begin{description}
  \item[Appendix A] Hey
  \item[Appendix B] Hey
\end{description}

También comentar que se pueden encontrar al final las referencias, una lista de
acrónimos y un índice alfabético.
