% -*- TeX-master: "../Thesis.tex" -*-


\chapter{Results and Comparison}\label{cha:results}

\epigraphhead[75]{
  \epigraph{\itshape However beautiful the strategy, \\
    you should occasionally look at the results.}
  {---Winston \textsc{Churchill}}
}


\lettrine{A}{quí mostraremos los resultados} del modelo final seleccionado (en
nuestro caso RefVOS). \textbf{No se mostrarán resultados entre diferentes
  iteracciones del modelo}. Lo dividiremos en varias partes, la parte
cualitativa y la cuantitativa. Compararemos con modelos SOTA siempre y con
diferentes datasets. La información la puedo sacar del paper de REfvOS y del de
RelatedWorks.



\section{Quantitative Evaluation}\label{sec:quantitative-eval}

Respecto a la evaluación cuantitativa del modelo (como ya se ha discutido en la
\vref{sec:quantitative-measures}) existen diferentes métricas a usar. Entre
ellas destacan tres: el mean y overall \gls{iou}, y precisión at threshold. En
el caso del modelo creado en este trabajo podemos evaluarlo con cualquier
métrica que consideremos oportuna y en cualquier dataset, ya que disponemos de
la implementación del mismo. Ahora bien, lo que realmente es interesante, es
poder compararlo con otros modelos state-of-the-art existentes actualmente. La
literatura consultada en este trabajo típicamente usa dos métricas
fundamentales: overall \gls{iou} y precisión at 0.5. La Prec@0.5 es usada como
una medida de accuracy, i.e., se computa el número de porcentaje de samples
donde la segmentación predecida overlaps with the ground trhut region by at
least 50\%.

En este apartado mostraremos tablas comparativas de la evaluación cuantitativa
del modelo de este trabajo y de otros modelos actuales. Para ello se hará un
estudio del overall \gls{iou} (see \vref{sec:overall-iou}) y un estudio del
accuracy o Prec@0.5 (see \vref{sec:accuracy}).


\subsection{Overall \glsentrylong{iou}}\label{sec:overall-iou}

Respecto al overall \gls{iou}, se han recogido datos de múltiples modelos, los
cuales se muestran de manera resumida en la \vref{tab:overall-iou}. Se ha
realizado la evaluación en los datasets RefCOCO y RefCOCO+ con los splits
\code{val/testA/testB}. Como ya se ha comentado anteriormente, el dataset
RefCOCO+ presenta \glspl{re} de mayor complejidad, por lo que presenta valores
de overall \gls{iou} más bajos que en el dataset RefCOCO para todos los
modelos.

\begin{table}[p]
  \centering
  \caption[Overall \glsentrylong{iou} model comparison]{Overall \acl{iou} model
    comparison. For each of the models the overall \acs{iou} is shown for the
    splits \code{val/testA/testB} in the datasets RefCOCO and RefCOCO+. The
    state of the art in each category is shown in bold. Full names for model
    acronyms can be found in section \textsl{Model Acronyms} on
    page~\pageref{sec:ac-model}. Table created by the author using data from
    second column references.}\label{tab:overall-iou}
  \rowcolors{5}{rowColor}{}
  \begin{tabular}{lc*6c}
    \toprule
    & & \multicolumn{3}{c}{\textbf{RefCOCO}} & \multicolumn{3}{c}{\textbf{RefCOCO+}} \\
    \cmidrule(lr){3-5}\cmidrule(lr){6-8}
    \textbf{Method} & \textbf{Paper}                                               & \code{val}     & \code{testA}   & \code{testB}   & \code{val}     & \code{testA}   & \code{testB}   \\
    \midrule
    \acs{asgn}      & \cite{qiu20:refer_image_segmen_gener_adver_learn}            & 50.46          & 51.20          & 49.27          & 38.41          & 39.79          & 35.97          \\
    \acs{brinet}    & \cite{hu20:bi_direc_relat_infer_networ}                      & 61.35          & 63.37          & 59.57          & 48.57          & 52.87          & 42.13          \\
    \acs{cac}       & \cite{chen19:refer_expres_objec_segmen_caption_aware_consis} & 58.90          & 61.77          & 53.81          & -              & -              & -              \\
    \acs{cmpc}      & \cite{huang20:refer_image_segmen_cross_modal_progr_compr}    & \textbf{61.36} & \textbf{64.53} & \textbf{59.64} & \textbf{49.56} & \textbf{53.44} & \textbf{43.23} \\
    \acs{cmsa}      & \cite{ye21:refer_segmen_images_videos_cross}                 & 58.32          & 60.61          & 55.09          & 43.76          & 47.60          & 37.89          \\
    \acs{dmn}       & \cite{margffoy-tuay18:dynam_multim_instan_segmen}            & 49.78          & 54.83          & 45.13          & 38.88          & 44.22          & 32.29          \\
    \acs{mattnet}   & \cite{yu18:mattn}                                            & 56.51          & 62.37          & 51.70          & 46.67          & 52.39          & 40.08          \\
    \acs{refvos}    & \cite{bellver20:refvos}                                      & 59.45          & 63.19          & 54.17          & 44.71          & 49.73          & 36.17          \\
    \acs{rmi}       & \cite{liu17:recur_multim_inter_refer_image_segmen}           & 45.18          & 45.69          & 45.57          & 29.86          & 30.48          & 29.50          \\
    \acs{rrn}       & \cite{li18:refer_image_segmen_recur_refin_networ}            & 55.33          & 57.26          & 53.95          & 39.75          & 42.15          & 36.11          \\
    \acs{step}      & \cite{chen19:see_throug_text_group_refer_image_segmen}       & 60.04          & 63.46          & 58.97          & 48.18          & 52.33          & 40.41          \\
    \bottomrule
  \end{tabular}\\[1.25ex]
  {\small\textbf{Note}. Models arranged in alphabetical order.}
\end{table}

Podemos destacar como modelos interesantes en este estudio el: TODO. hablar
aquí de diferentes modelos y qué son y cómo lo hacen.

El ganador absoluto respecto a esta métrica es el modelo creado por
\myCite{huang20:refer_image_segmen_cross_modal_progr_compr} y llamado
\gls{cmpc}. Presenta los valores más altos de overall \gls{iou} en todas las
categorías. Este modelo está basado en... TODO.


Aquí TODO tiene que haber mucho más texto para que se pueda llenar toda esta
página. Debería haber más TODO más texto para que se complete en un principio
hasta el finla. Aquí TODO tiene que haber mucho más texto para que se pueda
llenar toda esta página. Debería haber más TODO más texto para que se complete
en un principio hasta el finla.  Aquí TODO tiene que haber mucho más texto para
que se pueda llenar toda esta página. Debería haber más TODO más texto para que
se complete en un principio hasta el finla. Aquí TODO tiene que haber mucho más
texto para que se pueda llenar toda esta página.

Aquí TODO tiene que haber mucho más texto para que se pueda llenar toda esta
página. Debería haber más TODO más texto para que se complete en un principio
hasta el finla. Aquí TODO tiene que haber mucho más texto para que se pueda
llenar toda esta página. Debería haber más TODO más texto para que se complete
en un principio hasta el finla.


\subsection{Accuracy or Prec@0.5}\label{sec:accuracy}

TODO.
Ahora lo que falta añadir es la tabla de Prec@0.5 que está en el paper de
related works. Añadir los modelos más representativos, y hacer una estimación
para la precisión.

- Añadir aquí también algún comentario relacionado con lo de que la precisión
se puede hacer en diferntes escalas.














\begin{table}[p]
  \centering
  \caption[Accuracy or Prec@0.5 model comparison]{Accuracy or Prec@0.5 model
    comparison. For each of the models the accuracy percentage or Prec@0.5 is
    shown for the splits \code{val/testA/testB} in the datasets RefCOCO and
    RefCOCO+. The state of the art in each category is shown in bold. Full
    names for model acronyms can be found in section \textsl{Model Acronyms} on
    page~\pageref{sec:ac-model}. Table created by the author using data from
    second column references.}\label{tab:accuracy}
  \rowcolors{5}{rowColor}{}
  \begin{tabular}{lc*6c}
    \toprule
    & & \multicolumn{3}{c}{\textbf{RefCOCO}} & \multicolumn{3}{c}{\textbf{RefCOCO+}} \\
    \cmidrule(lr){3-5}\cmidrule(lr){6-8}
    \textbf{Method}  & \textbf{Paper}                                               & \code{val} & \code{testA} & \code{testB} & \code{val} & \code{testA} & \code{testB} \\
    \midrule
    \acs{brinet}     & \cite{hu20:bi_direc_relat_infer_networ}                      & 71.83      & 75.09        & 68.38        & -          & -            & -            \\
    \acs{cac}        & \cite{chen19:refer_expres_objec_segmen_caption_aware_consis} & 77.08      & 80.34        & 70.62        & -          & -            & -            \\
    \acs{cmpc}       & \cite{huang20:refer_image_segmen_cross_modal_progr_compr}    & 71.27      & -            & -            & -          & -            & -            \\
    \acs{cmsa}       & \cite{ye21:refer_segmen_images_videos_cross}                 & 69.24      & 73.87        & 64.55        & 45.48      & 51.41        & 37.57        \\
    \acs{mattnet}    & \cite{yu18:mattn}                                            & 76.65      & 81.14        & 69.99        & 65.33      & 71.62        & 56.02        \\
    \acs{refvos}     & \cite{bellver20:refvos}                                      &            &              &              &            &              &              \\
    \acs{rmi}        & \cite{liu17:recur_multim_inter_refer_image_segmen}           & 42.99      & 42.99        & 44.99        & 20.52      & 21.22        & 20.78        \\
    \acs{rrn}        & \cite{li18:refer_image_segmen_recur_refin_networ}            & 61.66      & 64.13        & 59.35        & 37.32      & 40.80        & 32.42        \\
    \acs{step}       & \cite{chen19:see_throug_text_group_refer_image_segmen}       & 70.15      & -            & -            & -          & -            & -            \\
    \acs{mmi}        & \cite{mao16:gener}                                           & -          & 64.90        & 54.51        & -          & 54.03        & 42.81        \\
    \acs{cmatterase} & \cite{liu19:improv_refer_expres_groun_cross_atten_erasin}    & 78.35      & 83.14        & 71.32        & 68.09      & 73.65        & 58.03        \\
    \acs{lgran}      & \cite{wang19:neigh}                                          & -          & 76.6         & 66.4         & -          & 64.00        & 53.40        \\
    \acs{faoa}       & \cite{yang19:fast_accur_one_stage_approac_visual_groun}      & 71.15      & 74.88        & 66.32        & 56.88      & 61.89        & 49.46        \\
    \acs{nmtree}     & \cite{liu19:learn_assem_neural_modul_tree}                   & 74.71      & 79.71        & 68.93        & 65.06      & 70.24        & 56.15        \\
    \acs{vilbert}    & \cite{lu19:vilber}                                           & -          & -            & -            & 72.34      & 78.52        & 62.61        \\
    \bottomrule
  \end{tabular}\\[1.25ex]
  {\small\textbf{Note}. Models arranged in alphabetical order.}
\end{table}


















\section{Qualitative Evaluation}

Aquí mostraremos de una manera visual diferentes ejemplos y patologías
encontradas en el procesado del sistema.


Aquí mostrar diferentes ejemplos con:
- Donde funciona bien.
- Donde funciona mal.
- Variando el tamaño.
- Variando la clase de objeto: animal, persona, objeto, fruta, comida,...
