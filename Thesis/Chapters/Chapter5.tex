% -*- TeX-master: "../Thesis.tex" -*-


\chapter{Results} \label{cha:results}

\epigraphhead[75]{
  \epigraph{\itshape TODO \\ All models are wrong, but some are useful.}
  {---\scshape George E. P. Box}
}


\lettrine{A}{quí} mosraremos los resultados del modelo final seleccionado (en
nuestro caso RefVOS). \textbf{No se mostrarán resultados entre diferentes
  iteracciones del modelo}. Lo dividiremos en varias partes, la parte
cualitativa y la cuantitativa. Compararemos con modelos SOTA siempre y con
diferentes datasets. La información la puedo sacar del paper de REfvOS y del de
RelatedWorks.


\section{Quantitative Evaluation}

Aquí mostraremos tablas comparativas de la evaluación cuantitativa.


\section{Qualitative Evaluation}

Aquí mostraremos de una manera visual diferentes ejemplos y patologías
encontradas en el procesado del sistema.
