% -*- TeX-master: "../Thesis.tex" -*-


\chapter{Results} \label{cha:results}

\epigraphhead[75]{
  \epigraph{\itshape TODO \\ All models are wrong, but some are useful.}
  {---\scshape George E. P. Box}
}


\lettrine{M}{odelizar} consiste en crear un modelo matemático que represente de
la manera más fiel posible una situación compleja. En este trabajo se usarán
dos modelos diferentes: uno de ellos para realizar el trabajo de \gls{rec}
partiendo de una \gls{re} en forma de texto y otro modelo para speech
recognition, de manera que también se pueda trabajar con lenguaje natural
hablado.


\section{Numerical Evaluation} \label{sec:evaluation}

Para la evaluación del modelo se usará el concepto de intersección y de unión
entre la segmentación predecida (que es una máscara \emph{binaria}) y el ground
truth (en la \vref{fig:sets} se muestran unos diagramas con estos
conceptos).

\begin{figure}[ht]
  \begin{subfigure}[t]{.5\textwidth}
    \centering
    \caption{Union of sets \(A\) and \(B\).}
    \includesvg[width=.55\textwidth]{Union_of_sets_A_and_B.svg}
  \end{subfigure}
  \begin{subfigure}[t]{.5\textwidth}
    \centering
    \caption{Intersection of sets \(A\) and \(B\).}
    \includesvg[width=.55\textwidth]{Intersection_of_sets_A_and_B.svg}
  \end{subfigure}
  \caption[Union and intersection of sets \(A\) and \(B\)]{Graphic
    representation of union and intersection of sets \(A\) and \(B\).}
  \label{fig:sets}
\end{figure}

De aquí surge el conocido Jaccard index o coeficiente \gls{iou},
\begin{equation}
  J(A,B) = \frac{|A \cap B|}{|A \cup B|}
  = \frac{|A \cap B|}{|A| + |B| - |A \cap B|},
\end{equation}
que es típicamente usado para medir el accuracy de un modelo, pero ---como
hemos comentado anteriormente--- no puede ser usado como función de pérdida al
no ser una aplicación diferenciable.

\begin{figure}[ht]
  \begin{subfigure}[t]{.45\textwidth}
    \centering
    \caption{Bounding boxes example.}
    \includegraphics[width=.8\textwidth]{Images/Object detection Bounding Boxes.jpg}
  \end{subfigure}\hfill
  \begin{subfigure}[t]{.45\textwidth}
    \centering
    \caption{\gls{iou} visual equation.}
    \includegraphics[width=.8\textwidth]{Images/Intersection over Union.png}
  \end{subfigure}
  \caption[Explicación del Jaccard Index]{Explicación y ejemplo del Jaccard
    Index en el caso de bounding boxes.}
\end{figure}

Este índice aporta información relevante sobre cómo de ajustada está una
bounding box.\footnote{Se estudia el caso de bounding box por simplicidad, pero
  el mismo concepto aplica en el caso de segmentación pixel a pixel.} Es
evidente que el Jaccard index toma un valor entre \(0\) y \(1\), siendo \(0\)
cuando no hay intersección entre las bounding boxes y tomando el valor de \(1\)
cuando la correspondencia es exacta.


\section{Visual Evaluation}

Aquí mostraremos de una manera visual diferentes ejemplos y patologías
encontradas en el procesado del sistema.
