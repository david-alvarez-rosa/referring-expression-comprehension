% -*- TeX-master: "../Thesis.tex" -*-


\chapter{Conclusions}\label{cha:concl}

\epigraphhead[75]{
  \epigraph{\itshape Now this is not the end.\\
    It is not even the beginning of the end.\\
    But it is, perhaps, the end of the beginning.}
  {---\scshape Winston Churchill}
}


\drop Carrying out this research work within the university framework of a
bachelor's thesis has allowed me, on a personal level, to initiate and delve
into topics related to \gls{ml} that are currently on everyone's lips. The
\gls{ai} has come to stay in our current society. The specific topic of this
work \gls{rec} has allowed me to work in the field of multimodal learning, so
that I have been able to explore at the same time the fields of \gls{cv} and
\gls{nlp}, which for me were a novelty. Being able to start from scratch and
finish training and modifying state-of-the-art models produces in me a
satisfying feeling on an academic level.

Likewise, it has allowed me to improve my skills in the field of programming:
both in the development of neural models using the Python PyTorch library, and
in the field of web development and the creation of \glspl{api} and management
of servers.

Likewise, I am satisfied with the results obtained, despite not having obtained
significant improvements in the RefVOS model. Having been able to train
professional models and being able to modify it and understand all the small
parts that make it up are already a source of joy for me. In addition, being
able to provide the general public with a website where they can easily
interact with models so complex that the one presented, I think is positive for
society in general. Tools similar to this one may be useful for future
researchers in this or similar field.

Also, this project has allowed me to grow as a person. It has been carried out
in a turbulent time within the COVID-19 pandemic, which has forced remote
work. Having to work remotely with a large research laboratory in \gls{ai} in
another time zone is not an easy task. I appreciate all the help received by
email and by videoconference from my thesis supervisors and the Vector
Institute members with whom I have been fortunate to discuss aspects of the
work.



\section{Future Work}

In order to advance in the task of \gls{rec} it is essential to first know the
main limitations that currently exist. One of the main problems that arise is
the difficulty of understanding what the model is doing. All the models present
in the current literature present some method in which the embeddings of
\gls{re} and the image are joined. Now, this process is currently a ``black
box'' for researchers: the reasoning process of the model cannot be
visualized. This greatly penalizes the possibilities of improving the models,
since it makes it very difficult to interpret the decisions of the model in the
understanding process.

In addition, this lack of interpretability of the reasoning method of the model
is enhanced by the evaluation process in which only the final prediction is
taken into account, so that a concrete evaluation of the reasoning process is
not made step by step. That is, the evaluation metrics used in current
state-of-the-art studies are not capable of extracting useful information about
the real reasoning capacity of the model and, therefore, do not provide a
vision about the deficiencies of the model.

Another important limitation in the \gls{rec} task is the lack of quality
dataset for training. It has been possible to manually observe samples from the
dataset that are not adequate (\gls{re} misspelled, containing bad words,
etc.). In addition, there is a clear imbalance in the samples of the current
datasets. Most of the \gls{re} present refer to the objects in the image using
attributes. This imbalance can lead to models where there is no deep reasoning
process, in which segmentation is only learned depending on the class to which
the object belongs. It could even be the case that the models ignore \gls{re}
and only make a random guess of the most representative object (that is, using
only the information present in the image).

Following the current jobs path may not lead to significant improvements in
model performance. That is, adding complexity (increasing the number of
basically trainable parameters) to current techniques \gls{rec} may not be the
way to go. Designing more sophisticated models but under the same current
principles will not necessarily lead to significant improvements in the task of
\gls{rec}. To achieve significant improvement, the next logical step is to try
to find models in which some sophisticated (and assessable) method of reasoning
can be exploited more effectively. I consider that the most successful ones in
the future would be multi-step reasoning models, in which relevant information
is actually extracted from \gls{re}. In addition, it would be very useful if
each of these reasoning steps could be visualized and validated with objective
metrics.

\begin{exampleBox}
  A simple example of these multi-step reasoning would be for example with the
  following \gls{re}, \re{woman in red dress sitting on the right} and an image
  with a large group of people. An ideal multi-step reasoning model would work
  as follows:
  \begin{enumerate}
    \item Find all the women present in the image, these objects will be the only
    solution candidates.
    \item From these women choose all those who wear a red dress.
    \item From this group select those that are seated.
    \item Finally, if there is more than one possibility, select the one on the
    right.
  \end{enumerate}

  In this type of model, or similar, the real reasoning would be guaranteed and
  it would be easy to evaluate step by step.
\end{exampleBox}

In the case of datasets, they could also be improved. It would be necessary to
collect more data, of higher quality and with different types of \gls{re}. In
addition, it would be very useful for the training and validation of the models
to have a metric to evaluate the difficulty of one \gls{re}. This dataset
expansion could be done, if necessary, making use of generative models, i.e.,
synthetic data could be used. This would be especially useful to correct
already detected imbalances.

Also add that you can see how to apply these models to video in addition to
image. The model presented in this thesis, obviously, could also be used for
video using it frame by frame. Now, the temporal relationship between different
frames would be neglected. Here it would be of vital importance to ensure the
efficiency and speed of the models used, to make real-speed comprehension
possible.
