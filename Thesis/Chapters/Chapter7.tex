% -*- TeX-master: "../Thesis.tex" -*-


\chapter{Project Analysis}\label{cha:analysis}

\epigraphhead[75]{
  \epigraph{\itshape Give me six hours to chop down a tree and I will spend the
    first four sharpening the axe.}
  {---\scshape Abraham Lincoln}
}


\drop The analysis of a project is fundamental from an engineering point of
view. This analysis falls within the scope of project management, which
constitutes the area in charge of managing the evolution of the project,
controlling and responding to problems that appear and facilitating its
completion and approval. Here we will analyze the work carried out from a
resource management point of view in terms of planning and scheduling the tasks
(see \vref{sec:planning}), an analysis of the cost of the project will be
carried out (see \vref{sec:cost}) and, finally, the environmental impact will
be studied (see \vref{sec:enviromental}).



\section{Planning and Scheduling}%
\label{sec:planning}\index{Planning}\index{Scheduling}

This project will (obviously) entail carrying out a series of activities for
its development. This time \emph{scheduling} of these activities and everything
related to the consideration of the necessary resources are the most important
functions to develop in project \emph{planning}.

The main objective of \emph{planning} is to obtain a distribution of activities
over time and tries to use resources in a way that minimizes the cost of the
project, always complying with the different conditions required: start/end
date, available technology, available resources, the maximum possible level of
occupation of these resources, etc. That is, project planning consists of a
scheduling of activities and a management of resources---which can be material
or human---to obtain a cost objective complying with the conditions
imposed/demanded by a particular client.


\subsection{Table of Activities}\label{sec:activities}

La programación de actividades nos permitirá disponer de un calendario de
ejecución del proyecto donde queden reflejadas las fechas de inicio y de fin de
las distintas actividades en las cuales se han descompuesto el proyecto. Para
facilitar la comprensión de las diferentes actividades, se han dividido en 5
grandes grupos:
\begin{itemize}
  \item \textbf{Learn basics of \gls{ml} and \gls{dl}.} Este conjunto de tareas
  ha consistido en adquirir los conocimientos básicos en los ámbitos de
  \gls{ml}/\gls{dl} que son los que han permitido poder continuar en el
  proyecto. Gran parte de lo aprendido aquí es precisamente lo descrito en el
  \vref{cha:theory}.
  \item \textbf{Learn thesis topic.} Una vez afianzados los conocimientos
  básicos en \gls{dl} se ha procedido a profundizar más en conocimientos
  avanzados y más relacionados con el ámbito concreto de la tesis. Para ello se
  han leído diferentes papers recomendados y se ha leído la literatura
  existente about state-of-the-art models.
  \item \textbf{Modeling.} Este conjunto de tareas coincide con el
  \vref{cha:models}. Aquí es donde se presentan los dos modelos usados en este
  trabajo.
  \item \textbf{Web development.} Aquí se recogen todas las actividades
  relacionadas con el desarrollo de la web (see \vref{cha:web}). Desde el
  aprendizaje de los lenguajes de front end hasta la publicación de la web con
  un dominio propio y pasando por la programación del back end.
  \item \textbf{Bachelor's thesis.} Estas tareas se corresponden con las
  solicitadas por la universidad: redacción de la memoria del trabajo, y la
  creación y preparación de la presentación final.
\end{itemize}

El conjunto de actividades desglosado se muestra en la
\vref{tab:activities}, donde se puede ver de manera detallada las tasks que
componen las main activities. Así mismo, se muestran fechas aproximadas de
comienzo y de final tanto de las tasks como de las main activities.

\begin{table}[p]
  \centering
  \caption[Main activities broken down into tasks]{Main activities broken down
    into tasks and with approximate start and end dates. Note that various
    tasks have been carried out in parallel.}\label{tab:activities}
  \begin{tabular}{cp{.55\textwidth}cc}
    \toprule
    \rowcolor{gray!37.5}
    \textbf{Code} & \textbf{Activity} & \textbf{Start} & \textbf{End} \\
    \midrule

    \rowcolor{rowColor}
    \textbf{A} & \textbf{Learn basics of \acs{ml}/\acs{dl}}                                      & Dec. 1 & Dec. 1 \\
    \rowcolor{rowColor}
    A1         & \Acs{ml} course \cite{ng20:machin_learn}                                        & 1      & 1      \\
    \rowcolor{rowColor}
    A2         & \Acs{dl} lectures from UPC \cite{giro-i-nieto20:all_deep_learn_upc_etset_telec} & 1      & 1      \\
    \rowcolor{rowColor}
    A3         & Stanford CS231n: \acsp{cnn} for Visual Recognition \cite{li20:cs231}            & 1      & 1      \\
    \rowcolor{rowColor}
    A4         & \Acs{dl} specialization \cite{ng20:deep_learn_special}                          & 1      & 1      \\
    \midrule

    \textbf{B} & \textbf{Learn thesis topic}                                                       & 10 & 5 \\
    B1         & Multimodal learning lectures \cite{giro-i-nieto20:all_deep_learn_upc_etset_telec} & 1  & 1 \\
    B2         & Publications                                                                      &    & 4 \\
    B3         & State-of-the-art papers on \acs{rec}: \cite{}                                     &    & 4 \\
    \midrule

    \rowcolor{rowColor}
    \textbf{C} & \textbf{Modeling}    & 8 & 1 \\
    \rowcolor{rowColor}
    C1         & Server usage         & 1 & 1 \\
    \rowcolor{rowColor}
    C2         & Multiple iterations  & 1 & 1 \\
    \rowcolor{rowColor}
    C3         & Generate test values & 1 & 1 \\
    \midrule

    \textbf{D} & \textbf{Web development}         & 10  & 5 \\
    D1         & Front end (HTML, CSS, JS)        &     & 4 \\
    D2         & \Acs{api} creation  (PHP)        &     & 4 \\
    D3         & Web server configuration         &     & 4 \\
    D4         & Publish website (domain, server) & 1,5 & 4 \\
    \midrule

    \rowcolor{rowColor}
    \textbf{E} & \textbf{Bachelor's thesis} & 8 & 1 \\
    \rowcolor{rowColor}
    E1         & Write thesis (\LaTeX)       & 1 & 1 \\
    \rowcolor{rowColor}
    \rowcolor{rowColor}
    E2         & Create presentation slides (\LaTeX) & 1 & 1 \\
    \rowcolor{rowColor}
    E4         & Prepare presentation       & 1 & 1 \\
    \bottomrule
  \end{tabular} \\[1.25ex]
  {\small\textbf{Note}. Table created by the author.}
\end{table}


\subsection{Gantt Chart}\label{sec:gantt}\index{Gantt chart}

La información recogida en en forma de tabla de actividades en la sección
anterior (\vref{sec:activities}), se puede mostrar de manera más gráfica con un
diagrama. La herramienta más conocida para representar la planificación de
tareas en el tiempo es la creada por
\myCite{gantt73:work_wages_profit_manag_histor_no}. Este diagrama, llamado en
honor a su creador como Gantt chart, es una herramienta gráfica cuyo objetivo
es exponer el tiempo de dedicación previsto para diferentes tareas o
actividades a lo largo de un tiempo total determinado.

Para este trabajo concreto, se muestra el Gantt chart correspondiente en la
\vref{fig:gantt}. Este chart es exactamente la representación gráfica de la
distribución de tareas de la \vref{tab:activities}.

\begin{figure}[ht]
  \centering
  % -*- TeX-master: "../../Thesis.tex" -*-


\begin{tikzpicture}
  \begin{ganttchart}[
    % Cuadrícula.
    vgrid = {{black!25, loosely dotted}, {black, loosely dotted}},
    hgrid = {*1{black, loosely dotted}},
    y unit chart = 1.03cm,
    % Título (semanas).
    title/.append style = {fill = yellowGantt},
    title height = .75,
    % Fondo.
    canvas/.append style = {fill = yellowGantt!15},
    expand chart = \textwidth,
    % Estilo día (en vertical).
    today = 8,
    today offset = .7,
    today label = Presente,
    today rule/.style = {draw = black!70, dashed, very thick},
    % Estilo progreso.
    progress = today,
    progress label text = {\pgfmathprintnumber[precision=0, verbatim]{#1}\% completado},
    bar progress label anchor = west,
    group progress label anchor = west,
    % Estilos grupos.
    group/.append style = {fill = greenGantt, draw = black, thick},
    group incomplete/.append style = {fill = greenGantt!30, draw = black, thick},
    group height = .5,
    group top shift = .25,
    group right shift = 0,
    group left shift = 0,
    group peaks height = .175,
    group peaks width = .65,
    group peaks tip position = .4,
    group label node/.append style= {align = right}, % Importante.
    % Estilo barras.
    bar/.append style = {fill = blueGantt},
    bar height = .5,
    bar incomplete/.append style = {fill = blueGantt!30, draw = black},
    bar label node/.append style= {align = right},
    % Barras verticales auxiliares.
    vrule/.style = {draw = none}
    ]{1}{28} % En medias semanas (dos días de clase/semana).

    % Semanas.
    \gantttitlelist{1,...,14}{2} \\

    % Entrega I.
    \ganttgroup{A. Propuesta\ganttalignnewline proyecto}{1}{8} \\
    \ganttbar{A1. Definición\ganttalignnewline general}{1}{4} \\
    \ganttbar{A2. Objetivos y \ganttalignnewline especificaciones}{3}{6} \\
    \ganttbar{A3. Entrega I}{5}{8} \\

    % Entrega II.
    \ganttgroup[progress = 0]{B. Análisis\ganttalignnewline usuarios}{8}{11} \\
    \ganttbar[progress = 0]{B1. Análisis \ganttalignnewline sistemas-usuario}{8}{9} \\
    \ganttbar[progress = 0]{B2. Análisis de\ganttalignnewline funciones}{8}{10} \\
    \ganttbar{B3. Entrega II}{9}{11} \\

    % Entrega III.
    \ganttgroup{C. Diseño\ganttalignnewline conceptual}{9}{14} \\
    \ganttbar{C1. Ergonomia y \ganttalignnewline seguridad}{9}{11} \\
    \ganttbar{C2. Normativa\ganttalignnewline vigente}{11}{13} \\
    \ganttbar{C3. Entrega III}{11}{14} \\

    % Entrega IV.
    \ganttgroup{D. Estudio\ganttalignnewline alternativas}{13}{20} \\
    \ganttbar{D1. Descripción\ganttalignnewline alternativas}{13}{16} \\
    \ganttbar{D2. Compración\ganttalignnewline y selección}{15}{18} \\
    \ganttbar{D3. Entrega IV}{15}{20} \\

    % Entrega V.
    \ganttgroup{E. Propuesta\ganttalignnewline solución}{19}{28} \\
    \ganttbar{E1. Memoria}{19}{26} \\
    \ganttbar{E2. Póster}{23}{26} \\
    \ganttbar{E3. Entrega V}{25}{28}

    % Barras verticales auxiliares.
    \ganttvrule{Inicio}{1} \ganttvrule{Fin}{27}
  \end{ganttchart}
\end{tikzpicture}
    \caption[TODO]{TODO.}\label{fig:gantt}
\end{figure}



\section{Cost Analysis}\label{sec:cost}

El coste total asociado a este proyecto se dividie en dos partes: el coste
personal y el coste de infraestructura.

\subsubsection{Coste Personal}

Respecto al coste personal, se refiere al número de horas dedicadas a la
realización de este trabajo, incluyendo todas sus partes. Esto es, aquí se
considerarán desde las horas dedicadas al aprendizaje, como las horas dedicadas
a la programación, como las horas dedicadas al diseño web y las horas dedicadas
a la redacción de la memoria y de la creación de la presentación de este
trabajo.

Para realizar la estimación de horas dedicadas, usaremos el \gls{ects}, que es
un estándar para comparar créditos académicos. Como es conocido, un crédito
\gls{ects} equivale a una dedicación de 25--30 horas. En este caso, al estar
realizando un bachelor thesis de dos titulaciones, sumaremos los créditos
destinados a cada titulación. Concretamente, para la titulación del grado en
\textsc{Enginyeria en Tecnologies Industrials}, son 12 créditos y para el grado
en \textsc{Matemàtiques} son 15 créditos. En total \SI{27}{\ectss}.

Por tanto, usando la equivalencia de 1 crédito \gls{ects} con 27,5 horas,
tenemos que el número de horas dedicado al trabajo será de,
\begin{equation}
  \SI{27}{\ectss} \times \frac{\SI{27.5}{\hour}}{\SI{1}{\ects}} =
  \SI{742.5}{\hour},
\end{equation}
lo cual, suponiendo un salario de \EUR{15}/hour, hace un total del coste
\emph{personal} de,
\begin{equation}
  \SI{742.5}{\hour} \times \frac{\text{\EUR{15}}}{\SI{1}{\hour}} =
  \text{\EUR{\num{11137.5}}}.
\end{equation}

\begin{remarkBox}
  Hay que tener en cuenta que esta estimación de \EUR{15}/hour es una
  \emph{aproximación}, para poder obtener un dato del dinero del coste
  personal. Claro está, que el número de horas dedicadas a formación tendrán
  una remuneración mucho más baja que la de las horas dedicadas a la creación
  del modelo y a la remuneración de las horas dedicadas al diseño web.
\end{remarkBox}

\subsubsection{Coste de Infraestructura}

Respecto al coste de \emph{infraesctructura}, solo será necesario incluir el
gasto realizado en servidores, ya que el resto de herramientas usadas son free
(as in \emph{freedom}) software, pero también free in terms of
price.\footnote{Aquí entran el uso de Python, PyTorch para la modelización; los
  lenguajes HTML, CSS, JS, PHP para la creación de la interfaz web; y \LaTeX\
  para la redacción de la memoria y la creacion dé la presentación.}

\subsubsection{Coste Total}

Por tanto, en total, se estima el coste de este trabajo en los \EUR{1221}.



\section{Enviromental Impact}\label{sec:enviromental}

El impacto ambiental que este trabajo ha producido es mínimo, ya que se ha
tratado de un desarrollo de software. El único elemento que tiene sentido
considerar en este aspecto es el de uso de energía eléctrica para la
alimentación del ordenador y de los servidores, ya que la genración de esta
energía eléctrica llevará ciertas emisiones de \ch{CO2}.

Suponiendo un consumo medio aproximado del ordenador de \SI{150}{\watt}, y que
ha sido usado durante el total de las \SI{742.5}{\hour} que ha durado el
proyecto (see \vref{sec:cost}), tenemos que, a nivel energético, se han
consumido,
\begin{equation}
  \SI{150}{\watt} \times \SI{742.5}{\hour} = \SI{111.375}{\kWh}.
\end{equation}

Podemos ahora, usando la calculadora online de emisiones de \ch{CO2}, creada
por \myCite{aragon21:emiss_calcul}, concluir que las emisiones de \ch{CO2} son
de \SI{39}{\kg} de \ch{CO2}.

\begin{exampleBox}
  Estas emisiones de \ch{CO2} son las que emitiría un único coche gasolina
  durante un trayecto de \SI{200}{\km} (from \cite{aragon21:emiss_calcul}).
\end{exampleBox}

También podríamos considerar las emisiones de \ch{CO2} debidas al uso de
servidores durante el entrenamiento y del servidor web. Ahora bien, en el
primer caso, es difícil de cuantificar, ya que es un servidor multi-nodo con
usuarios. Y, en el segundo caso, es difícil cuantificar el uso del servidor
web, ya que está abierto al publico y depende del número de usuarios que entren
en la web.

En cualquier caso, como era de esperar, el impacto medioambiental de este
proyecto es mínimo.