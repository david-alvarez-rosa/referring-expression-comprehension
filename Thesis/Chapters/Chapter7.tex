% -*- TeX-master: "../Thesis.tex" -*-


\chapter{Project Analysis}\label{cha:analysis}

\epigraphhead[75]{
  \epigraph{\itshape Give me six hours to chop down a tree and I will spend the
    first four sharpening the axe.}
  {---Abraham \textsc{Lincoln}}
}


\lettrine{T}{he analysis of a project} is fundamental from an engineering point
of view. This analysis falls within the scope of project management, which
constitutes the area in charge of managing the evolution of the project,
controlling and responding to problems that appear and facilitating its
completion and approval. Here we will analyze the work carried out from a
resource management point of view in terms of planning and scheduling the tasks
(see \vref{sec:planning}), an analysis of the cost of the project will be
carried out (see \vref{sec:cost}) and, finally, the environmental impact will
be studied (see \vref{sec:enviromental}).



\section{Planning and Scheduling}%
\label{sec:planning}\index{Planning}\index{Scheduling}

This project will (obviously) entail carrying out a series of activities for
its development. This time \emph{scheduling} of these activities and everything
related to the consideration of the necessary resources are the most important
functions to develop in project \emph{planning}.

The main objective of \emph{planning} is to obtain a distribution of activities
over time and tries to use resources in a way that minimizes the cost of the
project, always complying with the different conditions required: start/end
date, available technology, available resources, the maximum possible level of
occupation of these resources, etc. That is, project planning consists of a
scheduling of activities and a management of resources---which can be material
or human---to obtain a cost objective complying with the conditions
imposed/demanded by a particular client.


\subsection{Table of Activities}\label{sec:activities}

The scheduling of activities will allow us to have a project execution calendar
where the start and end dates of the different activities in which the project
has been decomposed are reflected. To facilitate understanding of the different
activities, they have been divided into 5 large groups:
\begin{itemize}
  \item \textbf{Learn basics of \gls{ml} and \gls{dl}.} This set of tasks has
  consisted of acquiring basic knowledge in the areas of \gls{ml}/\gls{dl} that
  have allowed us to continue in the project. Much of what is learned here is
  precisely what is described in \vref{cha:theory}.
  \item \textbf{Learn thesis topic.} Once the basic knowledge was established
  in \gls{dl}, we have proceeded to go deeper into advanced knowledge and more
  related to the specific field of the thesis. For this, different recommended
  papers have been read and the existing literature about state-of-the-art
  models has been read.
  \item \textbf{Models creation.} This set of tasks coincides with
  \vref{cha:models}. This is where the two models used in this work are
  presented.
  \item \textbf{Web development.} Here all the activities related to the
  development of the web are collected (see \vref{cha:web}). From learning
  front end languages to publishing the web with your own domain and going
  through back end programming.
  \item \textbf{Bachelor's thesis.} These tasks correspond to those requested
  by the university: writing the work report, and creating and preparing the
  final presentation.
\end{itemize}

The set of activities broken down is shown in \vref{tab:activities}, where you
can see in detail the tasks that make up the main activities. Likewise,
approximate start and end dates are shown for both the tasks and the main
activities.

\begin{table}[p]
  \centering
  \caption[Main activities broken down into tasks]{Main activities broken down
    into tasks and with approximate start and end dates. Note that various
    tasks have been carried out in parallel.}\label{tab:activities}
  \begin{tabular}{cp{.55\textwidth}cc}
    \toprule

    \rowcolor{gray!37.5}
    \textbf{Code} & \textbf{Activity} & \textbf{Start} & \textbf{End} \\
    \midrule

    \rowcolor{rowColor}
    \textbf{A} & \textbf{Learn basics of \acs{ml}/\acs{dl}} & Dec. 1 & Dec. 1 \\
    \rowcolor{rowColor}
    A1 & \Acs{ml} course \cite{ng20:machin_learn} &  &  \\
    \rowcolor{rowColor} A2 & \Acs{dl} lectures from UPC
                             \cite{giro-i-nieto20:all_deep_learn_upc_etset_telec}
                                      &  &  \\
    \rowcolor{rowColor}
    A3 & Stanford CS231n: \acsp{cnn} for Visual Recognition \cite{li20:cs231} &  &  \\
    \rowcolor{rowColor}
    A4 & \Acs{dl} specialization \cite{ng20:deep_learn_special} &  &  \\
    \midrule

    \textbf{B} & \textbf{Learn thesis topic} &  &  \\
    B1 & Multimodal learning lectures \cite{giro-i-nieto20:all_deep_learn_upc_etset_telec} &  &  \\
    B2 & Publications & &  \\
    B3 & State-of-the-art papers on \acs{rec}: \cite{} & &  \\
    \midrule

    \rowcolor{rowColor}
    \textbf{C} & \textbf{Models creation} &  &  \\
    \rowcolor{rowColor} C1 & Server usage &  &  \\
    \rowcolor{rowColor} C2 & Multiple iterations &  &  \\
    \rowcolor{rowColor} C3 & Generate test values &  &  \\
    \midrule

    \textbf{D} & \textbf{Web development} & &  \\
    D1 & Front end (HTML, CSS, JS) & &  \\
    D2 & \Acs{api} creation (PHP) & &  \\
    D3 & Web server configuration & &  \\
    D4 & Publish website (domain, server) &  &  \\
    \midrule

    \rowcolor{rowColor}
    \textbf{E} & \textbf{Bachelor's thesis} &  &  \\
    \rowcolor{rowColor}
    E1 & Write thesis (\LaTeX) &  &  \\
    \rowcolor{rowColor}
    E2 & Create presentation slides (\LaTeX) &  &  \\
    \rowcolor{rowColor} E4 & Prepare presentation &  &  \\
    \bottomrule
  \end{tabular} \\[1.25ex]
  {\small\textbf{Note}. Table created by the author.}
\end{table}


\subsection{Gantt Chart}\label{sec:gantt}\index{Gantt chart}

The information collected in the form of an activity table in the previous
section (\vref{sec:activities}) can be shown more graphically with a
diagram. The best known tool to represent the planning of tasks over time is
the one created by \myCite{gantt73:work_wages_profit_manag_histor_no}. This
diagram, named in honor of its creator as the Gantt chart, is a graphical tool
whose objective is to expose the time of dedication planned for different tasks
or activities over a given total time.

For this specific work, the corresponding Gantt chart is shown in
\vref{fig:gantt}. This chart is exactly the graphical representation of the
distribution of tasks in \vref{tab:activities}.

\begin{figure}[ht]
  \centering
  % -*- TeX-master: "../../Thesis.tex" -*-


\begin{tikzpicture}
  \begin{ganttchart}[
    % Cuadrícula.
    vgrid = {{black!25, loosely dotted}, {black, loosely dotted}},
    hgrid = {*1{black, loosely dotted}},
    y unit chart = 1.03cm,
    % Título (semanas).
    title/.append style = {fill = yellowGantt},
    title height = .75,
    % Fondo.
    canvas/.append style = {fill = yellowGantt!15},
    expand chart = \textwidth,
    % Estilo día (en vertical).
    today = 8,
    today offset = .7,
    today label = Presente,
    today rule/.style = {draw = black!70, dashed, very thick},
    % Estilo progreso.
    progress = today,
    progress label text = {\pgfmathprintnumber[precision=0, verbatim]{#1}\% completado},
    bar progress label anchor = west,
    group progress label anchor = west,
    % Estilos grupos.
    group/.append style = {fill = greenGantt, draw = black, thick},
    group incomplete/.append style = {fill = greenGantt!30, draw = black, thick},
    group height = .5,
    group top shift = .25,
    group right shift = 0,
    group left shift = 0,
    group peaks height = .175,
    group peaks width = .65,
    group peaks tip position = .4,
    group label node/.append style= {align = right}, % Importante.
    % Estilo barras.
    bar/.append style = {fill = blueGantt},
    bar height = .5,
    bar incomplete/.append style = {fill = blueGantt!30, draw = black},
    bar label node/.append style= {align = right},
    % Barras verticales auxiliares.
    vrule/.style = {draw = none}
    ]{1}{28} % En medias semanas (dos días de clase/semana).

    % Semanas.
    \gantttitlelist{1,...,14}{2} \\

    % Entrega I.
    \ganttgroup{A. Propuesta\ganttalignnewline proyecto}{1}{8} \\
    \ganttbar{A1. Definición\ganttalignnewline general}{1}{4} \\
    \ganttbar{A2. Objetivos y \ganttalignnewline especificaciones}{3}{6} \\
    \ganttbar{A3. Entrega I}{5}{8} \\

    % Entrega II.
    \ganttgroup[progress = 0]{B. Análisis\ganttalignnewline usuarios}{8}{11} \\
    \ganttbar[progress = 0]{B1. Análisis \ganttalignnewline sistemas-usuario}{8}{9} \\
    \ganttbar[progress = 0]{B2. Análisis de\ganttalignnewline funciones}{8}{10} \\
    \ganttbar{B3. Entrega II}{9}{11} \\

    % Entrega III.
    \ganttgroup{C. Diseño\ganttalignnewline conceptual}{9}{14} \\
    \ganttbar{C1. Ergonomia y \ganttalignnewline seguridad}{9}{11} \\
    \ganttbar{C2. Normativa\ganttalignnewline vigente}{11}{13} \\
    \ganttbar{C3. Entrega III}{11}{14} \\

    % Entrega IV.
    \ganttgroup{D. Estudio\ganttalignnewline alternativas}{13}{20} \\
    \ganttbar{D1. Descripción\ganttalignnewline alternativas}{13}{16} \\
    \ganttbar{D2. Compración\ganttalignnewline y selección}{15}{18} \\
    \ganttbar{D3. Entrega IV}{15}{20} \\

    % Entrega V.
    \ganttgroup{E. Propuesta\ganttalignnewline solución}{19}{28} \\
    \ganttbar{E1. Memoria}{19}{26} \\
    \ganttbar{E2. Póster}{23}{26} \\
    \ganttbar{E3. Entrega V}{25}{28}

    % Barras verticales auxiliares.
    \ganttvrule{Inicio}{1} \ganttvrule{Fin}{27}
  \end{ganttchart}
\end{tikzpicture}
  \vspace{-.6cm} % Close you eyes and never look here again.
  \caption[Gantt chart of main activities]{Gantt chart of main activities. The
    duration and relationship between main activities is shown
    graphically. Figure created by the author.}\label{fig:gantt}
\end{figure}



\section{Cost Analysis} \label{sec:cost}

The total cost associated with this project is divided into two parts: the
personal cost and the infrastructure cost.

\subsubsection{Personal Cost}

Regarding personal cost, it refers to the number of hours dedicated to carrying
out this work, including all its parts. That is, here they will be considered
from the hours dedicated to learning, such as the hours dedicated to
programming, such as the hours dedicated to web design and the hours dedicated
to the writing of the memory and the creation of the presentation of this work.

To estimate the hours dedicated, we will use \gls{ects}, which is a standard
for comparing academic credits. As is known, one credit \gls{ects} is
equivalent to a dedication of 25-30 hours. In this case, as we are doing a
bachelor thesis of two degrees, we will add the credits allocated to each
degree. Specifically, for the degree in \textsc{Industrial Technology
  Engineering}, there are 12 credits and for the degree in \textsc{Mathematics}
there are 15 credits. In total \SI{27}{\ectss}.

Therefore, using the equivalence of 1 credit \gls{ects} with 27.5 hours, we
have that the number of hours dedicated to work will be,
\begin{equation}
  \SI{27}{\ectss} \times \frac{\SI{27.5}{\hour}}{\SI{1}{\ects}} =
  \SI{742.5}{\hour},
\end{equation}
which, assuming a wage of \EUR{12}/hour, makes a total cost \emph{personal} of,
\begin{equation}
  \SI{742.5}{\hour} \times \frac{\text{\EUR{12}}}{\SI{1}{\hour}} =
  \text{\EUR{\num{8910}}}.
\end{equation}

\begin{remarkBox}
  It must be taken into account that this estimate of \EUR{12}/hour is a
  \emph{aproximación}, in order to obtain data on the personal cost money. Of
  course, the number of hours dedicated to training will have a much lower
  remuneration than that of the hours dedicated to the creation of the model
  and to the remuneration of the hours dedicated to web design.
\end{remarkBox}

\subsubsection{Infrastructure Cost}

Regarding the \emph{infraesctructura} cost, it will only be necessary to
include the expenditure made on servers, since the rest of the tools used are
free (as in \emph{freedom}) software, but also free in terms of
price.\footnote{Here they enter the use of Python, PyTorch for modeling; the
  HTML, CSS and JS languages for the creation of the web interface; PHP for
  \gls{api}; Apache as a web server; and \LaTeX \ for the writing of the report
  and the creation of the presentation.} The servers used for training have
been assigned by \fhref{https://vectorinstitute.ai/}{Vector Institute}. The
server used for the web has an approximate cost of \EUR{20} / month and has
been rented for a total of 2 months. In total \EUR{40}.

\subsubsection{Total Cost}

Therefore, the total cost can be calculated by adding the two cost sources,
personnel and infrastructure. Clearly, the personal cost far exceeds the
infrastructure cost (mainly because the cost of the training servers with
\gls{gpu} has been zero as they have been provided free of charge). The total
cost was \text{\EUR{8950}}.



\section{Enviromental Impact} \label{sec:enviromental}

The environmental impact that this work has produced is minimal, since it has
been a software development. The only element that makes sense to consider in
this regard is the use of electrical energy to power the computer and servers,
since the generation of this electrical energy will lead to certain emissions
of \ch{CO2}.

Assuming an approximate average consumption of the computer of \SI{150}{\watt},
and that it has been used during the total of \SI{742.5}{\hour} that the
project has lasted (see \vref{sec:cost}), we have that, at an energy level,
they have been consumed,
\begin{equation}
  \SI{150}{\watt} \times \SI{742.5}{\hour} = \SI{111.375}{\kWh}.
\end{equation}

We can now, using the online emission calculator of \ch{CO2}, created by
\myCite{aragon21:emiss_calcul}, conclude that the emissions of \ch{CO2} are
\SI{39}{\kg} of \ch{CO2}.

\begin{exampleBox}
  These emissions of \ch{CO2} are those that a single gasoline car would emit
  during a journey of \SI{200}{\km} (from \cite{aragon21:emiss_calcul}).
\end{exampleBox}

We could also consider the emissions of \ch{CO2} due to the use of servers
during training and the web server. Now, in the first case, it is difficult to
quantify, since it is a multi-node server with users. And, in the second case,
it is difficult to quantify the use of the web server, since it is open to the
public and depends on the number of users entering the web.

In any case, unsurprisingly, the environmental impact of this project is
minimal.
