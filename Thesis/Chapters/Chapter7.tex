% -*- TeX-master: "../Thesis.tex" -*-


\chapter{Project Analysis}\label{cha:analysis}

\epigraphhead[75]{
  \epigraph{\itshape Give me six hours to chop down a tree and I will spend the
    first four sharpening the axe.}
  {---\scshape Abraham Lincoln}
}


\drop The analysis of a project is fundamental from an engineering point of
view. This analysis falls within the scope of project management, which
constitutes the area in charge of managing the evolution of the project,
controlling and responding to problems that appear and facilitating its
completion and approval. Here we will analyze the work carried out from a
resource management point of view in terms of planning and scheduling the tasks
(see \vref{sec:planning}), an analysis of the cost of the project will be
carried out (see \vref{sec:cost}) and, finally, the environmental impact will
be studied (see \vref{sec:enviromental}).



\section{Planning and Scheduling}%
\label{sec:planning}\index{Planning}\index{Scheduling}

This project will (obviously) entail carrying out a series of activities for
its development. This time \emph{scheduling} of these activities and everything
related to the consideration of the necessary resources are the most important
functions to develop in project \emph{planning}.

The main objective of \emph{planning} is to obtain a distribution of activities
over time and tries to use resources in a way that minimizes the cost of the
project, always complying with the different conditions required: start/end
date, available technology, available resources, the maximum possible level of
occupation of these resources, etc. That is, project planning consists of a
scheduling of activities and a management of resources---which can be material
or human---to obtain a cost objective complying with the conditions
imposed/demanded by a particular client.


\subsection{Table of Activities}\label{sec:activities}




\begin{comment}
  Aquí diferentes cosas:

  - En un inicio aprendizaje de lo básico sobre Deep Learning. Realicé varios
  cursos: el de Stanford, los de Github de UPC Telecos, uno de Machine Learning
  en Coursera y otro de Deep Learning Specialization en Coursera (este es el
  que era de 5 cursos pero que no terminé).

  - Después de esto aprendí sobre temas más profundos del ámbito en el que iba
  a trabajar leyendo publicaciones sobre el tema. Aquí podría hacer un listado
  de los diferentes papers que leí (los que tengo impresos más los que me envió
  Sanja por Slack).

  - Después de esto comencé con el tema concreto de la tesis. Para ello ya leí
  SOTA de exactamente la tesis. Aquí podría citar todos los papers que aparecen
  en la sección de ``related works''.

  - Aquí faltaría que tengo que aprender a usar el servidor.

  - Después de esto comenzó el proceso iterativo de mejorar el modelo de
  RefVOS.

  - Paralelamente con esto estuvo todo el trabajo de formarme sobre desarrollo
  web (es decir, aprender a usar los diferentes lenguajes típicos web).

  - Y, por último, llegó la creación de la web.

  - Después de esto llegó el escribir la tesis.

  - Por último, crear la presentación en pdf y prepararla.

  Esto se agrupa en:
  - Learning:
      - Aprendizaje fundamentos ML/DL.
      - Aprendizaje más produndo sobre ámbito relacionado con la tesis (aprendizaje
  multimodal, etc.)
      - Aprendizaje tema concreto de la tesis (básicamente leer SOTA).

  - Entranimiento modelos:
      - Aprender a usar servidor (los programas raros estos que son unas
      colas).
      - Probar diferentes iteracciones del modelo.
      - Generación de los datos de evaluación.

  - Desarrollo web:
     - Aprender lenguajes front-end (HTML, CSS, JS) quizá decir también que uso
     Bootstrap.
     - Aprender lenguaje de backend para poder realizar la API que conecta el
     frontend con el modelo en Python (es decir, aprender a usar PHP).
     - Alquilar servidor, ser capaz de configurar Apache como web server y
     comprar dominio público para hacer la web accesible desde cualquier
     lugar.

  - Bachelor thesis
      - Escribir todo este trabajo.
      - Maquetarlo con Latex.
      - Crear una presentación para presentarlo.
      - Preparar la presentación.
\end{comment}







La programación de actividades nos permitirá disponer de un calendario de
ejecución del proyecto donde queden reflejadas las fechas de inicio y de fin
de las distintas actividades en las cuales se han descompuesto el proyecto.

Para poder definir dicho calendario será necesario conocer las duraciones de
las diversas actividades a realizar y su orden relativo. Asimismo, tendrán que
ser fijadas las fechas de inicio y fin del proyecto en su conjunto. De esta
manera, en la \vref{tab:activities} se muestran las actividades principales a
desarrollar en este proyecto.

\begin{table}[ht]
  \centering
  \rowcolors{1}{}{rowColor}
  \caption[Main activities of the project]{Main activities of the project. The
    temporal relationship between them and the start/end dates are shown (note
    that there are activities that have been carried out in
    parallel).}\label{tab:activities}
  \begin{tabular}{cl*3c}
    \toprule
    \textbf{Code} & \textbf{Activity name} & \textbf{Next} & \textbf{Start} & \textbf{End} \\
    \midrule
    A & Learning                       &   & Oct. &    \\
    B & Model trainning                &   &      &    \\
    C & Web development                &   &      &    \\
    D & Bachelor's thesis                &   &      &    \\
    \bottomrule
  \end{tabular} \\[1.25ex]
  {\small\textbf{Note}. Table created by the author.}
\end{table}

Para poder hacernos una idea más concreta de las diferentes etapas/fases del
proyecto, se muestra en la tabla \ref{tab:activities-desglosadas} el desglose
en subpartes de las etapas. De las fases \textbf{A}-\textbf{D} se destacan dos
partes principales (A1 y A2 en el caso de \textbf{A}, por ejemplo), y la tercera
y última parte se corresponde a la preparación de la entrega correspondiente
(preparación del informe + preparación de la presentación) (A3 en el caso de
\textbf{A}). \\

De esta manera estamos dividiendo cada etapa (entrega) del proyecto en dos
partes más de tipo informativo (recabar información y aprendizaje nuestro) y
otra parte más de presentación de esta información adquirida. Análogamente a la
tabla resumida anterior, a información que se muestra en la tabla se encuentra
expresada en horas \textbf{por alumno} y las celdas de Inicio y Fin en semanas. \\

\begin{table}[p]
  \centering
  \caption[TODO]{TODO. Actividades principales desglosadas.}
  \label{tab:activities-desglosadas}
  \begin{tabular}{clcccc}
    \toprule
    \rowcolor{gray!37.5}
    \bf Código & \bf Denominación actividad & \bf Sig. & \bf Duración & \bf Inicio & \bf Fin \\
    \midrule

    \rowcolor{gray!7.5}
    \textbf{A} & \bf Propuesta del proyecto      & A1    & 8 & 1 & 4 \\
    \rowcolor{gray!7.5}
    A1         & Definición general del proyecto & A2    & 1 & 1 & 2 \\
    \rowcolor{gray!7.5}
    A2         & Objetivos y especificaciones    & A3    & 2 & 2 & 3 \\
    \rowcolor{gray!7.5}
    A3         & Preparación de la entrega I     & \bf B & 4 & 3 & 4 \\
    \midrule

    \textbf{B} & \bf Análisis de usuarios y sistemas & B1    & 10        & 4 & 5 \\
    B1         & Análisis de sistemas y del usuario  & B2    & 1,5 + 1,5 & 4 & 4 \\
    B2         & Análisis de funciones               & B3    & 3         & 4 & 4 \\
    B3         & Preparación de la entrega II        & \bf C & 4         & 5 & 5 \\
    \midrule

    \rowcolor{gray!7.5}
    \textbf{C} & \bf Diseño conceptual             & C1    & 16    & 5 & 7 \\
    \rowcolor{gray!7.5}
    C1         & Estudio ergonómico y de seguridad & C2    & 3 + 3 & 5 & 5 \\
    \rowcolor{gray!7.5}
    C2         & Normativa vigente                 & C3    & 5     & 6 & 6 \\
    \rowcolor{gray!7.5}
    C3         & Preparación de la entrega III     & \bf D & 5     & 6 & 7 \\
    \midrule

    \textbf{D} & \bf Estudio de alternativas       & D1    & 20    & 7 & 10 \\
    D1         & Descripción alternativas          & D2    & 6     & 7 & 8  \\
    D2         & Análisis comparativos y selección & D3    & 6 + 2 & 8 & 9  \\
    D3         & Preparación de la entrega IV      & \bf E & 6     & 8 & 10 \\
    \midrule

    \rowcolor{gray!7.5}
    \textbf{E} & \bf Propuesta de solución. Memoria & E1    & 35 & 10 & 14 \\
    \rowcolor{gray!7.5}
    E1         & Memoria. Informe final.            & E2    & 10 & 10 & 13 \\
    \rowcolor{gray!7.5}
    E2         & Póster.                            & E3    & 10 & 12 & 13 \\
    \rowcolor{gray!7.5}
    E3         & Preparación de la entrega V        & \bf - & 5  & 13 & 14 \\
    \bottomrule
  \end{tabular}                                                            \\[1.25ex]
  \textbf{Nota}. Duración expresada en horas. Inicio y fin en semanas.
\end{table}










\subsection{Gantt Chart}\label{sec:gantt}\index{Gantt chart}


Hey.



La información recogida en la sección en forma de tabla de actividades en la
sección anterior (sección \ref{sec:activities}), se puede mostrar de manera más
gráfica con el conocido diagrama de Gantt \cite{gantt}. El \textbf{diagrama de
  Gant} es una herramienta gráfica cuyo objetivo es exponer el tiempo de
dedicación previsto para diferentes tareas o actividades a lo largo de un tiempo
total determinado.

De esta manera, se \textbf{representar} la información de la tabla
\ref{tab:activities} en un diagrama de Gantt (figura \ref{fig:gantt})
resumido. Es importante destacar que la distribución temporal es en
\textbf{semanas} y que no hay margen en ninguna actividad puesto que todas son secuenciales. Eso es, no podemos adelantar ni retrasar trabajo porque solo hay un camino que sería el crítico en este caso.

\begin{figure}[htbp]
    \centering
    \includegraphics[width=.5\textwidth]{example-image-a}
    \caption[TODO]{TODO.}\label{fig:gantt}
\end{figure}

Estas serían las actividades que ser irán desarrollando en las próximas 12
semanas. Claro esta, que al referirnos a actividades futuras, siempre existe una
\textbf{incertidumbre} sobre las fechas y duración de las mismos. Por tanto,
este diagrama tiene que considerarse como una guía, la cual no obliga al
cumplimiento inmediato de las actividades mostradas. Siempre hay que trabajar
con cierto margen, hay que ser previsor, nunca se sabe qué problemas nos podemos
encontrar en el futuro.



\section{Cost Analysis}\label{sec:cost}

El coste total asociado a este proyecto se dividie en dos partes: el coste
personal y el coste de infraestructura.

\subsubsection{Coste Personal}

Respecto al coste personal, se refiere al número de horas dedicadas a la
realización de este trabajo, incluyendo todas sus partes. Esto es, aquí se
considerarán desde las horas dedicadas al aprendizaje, como las horas dedicadas
a la programación, como las horas dedicadas al diseño web y las horas dedicadas
a la redacción de la memoria y de la creación de la presentación de este
trabajo.

Para realizar la estimación de horas dedicadas, usaremos el \gls{ects}, que es
un estándar para comparar créditos académicos. Como es conocido, un crédito
\gls{ects} equivale a una dedicación de 25--30 horas. En este caso, al estar
realizando un bachelor thesis de dos titulaciones, sumaremos los créditos
destinados a cada titulación. Concretamente, para la titulación del grado en
\textsc{Enginyeria en Tecnologies Industrials}, son 12 créditos y para el grado
en \textsc{Matemàtiques} son 15 créditos. En total \SI{27}{\ectss}.

Por tanto, usando la equivalencia de 1 crédito \gls{ects} con 27,5 horas,
tenemos que el número de horas dedicado al trabajo será de,
\begin{equation}
  \SI{27}{\ectss} \times \frac{\SI{27.5}{\hour}}{\SI{1}{\ects}} =
  \SI{742.5}{\hour},
\end{equation}
lo cual, suponiendo un salario de \EUR{15}/hour, hace un total del coste
\emph{personal} de,
\begin{equation}
  \SI{742.5}{\hour} \times \frac{\text{\EUR{15}}}{\SI{1}{\hour}} =
  \text{\EUR{\num{11137.5}}}.
\end{equation}

\begin{remarkBox}
  Hay que tener en cuenta que esta estimación de \EUR{15}/hour es una
  \emph{aproximación}, para poder obtener un dato del dinero del coste
  personal. Claro está, que el número de horas dedicadas a formación tendrán
  una remuneración mucho más baja que la de las horas dedicadas a la creación
  del modelo y a la remuneración de las horas dedicadas al diseño web.
\end{remarkBox}

\subsubsection{Coste de Infraestructura}

Respecto al coste de \emph{infraesctructura}, solo será necesario incluir el
gasto realizado en servidores, ya que el resto de herramientas usadas son free
(as in \emph{freedom}) software, pero también free in terms of
price.\footnote{Aquí entran el uso de Python, PyTorch para la modelización; los
  lenguajes HTML, CSS, JS, PHP para la creación de la interfaz web; y \LaTeX\
  para la redacción de la memoria y la creacion dé la presentación.}

\subsubsection{Coste Total}

Por tanto, en total, se estima el coste de este trabajo en los \EUR{1221}.



\section{Enviromental Impact}\label{sec:enviromental}

El impacto ambiental que este trabajo ha producido es mínimo, ya que se ha
tratado de un desarrollo de software. El único elemento que tiene sentido
considerar en este aspecto es el de uso de energía eléctrica para la
alimentación del ordenador y de los servidores, ya que la genración de esta
energía eléctrica llevará ciertas emisiones de \ch{CO2}.

Suponiendo un consumo medio aproximado del ordenador de \SI{150}{\watt}, y que
ha sido usado durante el total de las \SI{742.5}{\hour} que ha durado el
proyecto (see \vref{sec:cost}), tenemos que, a nivel energético, se han
consumido,
\begin{equation}
  \SI{150}{\watt} \times \SI{742.5}{\hour} = \SI{111.375}{\kWh}.
\end{equation}

Podemos ahora, usando la calculadora online de emisiones de \ch{CO2}, creada
por \myCite{aragon21:emiss_calcul}, concluir que las emisiones de \ch{CO2} son
de \SI{39}{\kg} de \ch{CO2}.

\begin{exampleBox}
  Estas emisiones de \ch{CO2} son las que emitiría un único coche gasolina
  durante un trayecto de \SI{200}{\km} (from \cite{aragon21:emiss_calcul}).
\end{exampleBox}

También podríamos considerar las emisiones de \ch{CO2} debidas al uso de
servidores durante el entrenamiento y del servidor web. Ahora bien, en el
primer caso, es difícil de cuantificar, ya que es un servidor multi-nodo con
usuarios. Y, en el segundo caso, es difícil cuantificar el uso del servidor
web, ya que está abierto al publico y depende del número de usuarios que entren
en la web.

En cualquier caso, como era de esperar, el impacto medioambiental de este
proyecto es mínimo.