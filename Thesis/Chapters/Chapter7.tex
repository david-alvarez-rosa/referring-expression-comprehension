% -*- TeX-master: "../Thesis.tex" -*-


\chapter{Project Analysis}\label{cha:analysis}

\epigraphhead[75]{
  \epigraph{\itshape Give me six hours to chop down a tree and I will spend the
    first four sharpening the axe.}
  {---\scshape Abraham Lincoln}
}


\drop Lorem ipsum dolor sit amet, consectetur adipiscing elit, sed do eiusmod
tempor incididunt ut labore et dolore magna aliqua. Ut enim ad minim veniam,
quis nostrud exercitation ullamco laboris nisi ut aliquip ex ea commodo
consequat. Duis aute irure dolor in reprehenderit in voluptate velit esse
cillum dolore eu fugiat nulla pariatur. Excepteur sint occaecat cupidatat non
proident, sunt in culpa qui officia deserunt mollit anim id est laborum.



\section{Planning and Programming}

Hey.


\subsection{Table of Activities}

Aquí diferentes cosas:
- En un inicio aprendizaje de lo básico sobre Deep Learning. Realicé varios
cursos: el de Stanford, los de Github de UPC Telecos, uno de Machine Learning
en Coursera y otro de Deep Learning Specialization en Coursera (este es el que
era de 5 cursos pero que no terminé).
- Después de esto aprendí sobre temas más profundos del ámbito en el que iba a
trabajar leyendo publicaciones sobre el tema. Aquí podría hacer un listado de
los diferentes papers que leí (los que tengo impresos más los que me envió
Sanja por Slack).
- Después de esto comencé con el tema concreto de la tesis. Para ello ya leí
SOTA de exactamente la tesis. Aquí podría citar todos los papers que aparecen
en la sección de ``related works''.
- Después de esto comenzó el proceso iterativo de mejorar el modelo de RefVOS.
- Paralelamente con esto estuvo todo el trabajo de formarme sobre desarrollo
web (es decir, aprender a usar los diferentes lenguajes típicos web).
- Y, por último, llegó la creación de la web.
- Después de esto llegó el escribir la tesis.
- Por último, crear la presentación en pdf y prepararla.


\subsection{Gantt Chart}

Hey.


\section{Cost Analysis}\label{sec:cost}

El coste total asociado a este proyecto se dividie en dos partes: el coste
personal y el coste de infraestructura.

\subsubsection{Coste Personal}

Respecto al coste personal, se refiere al número de horas dedicadas a la
realización de este trabajo, incluyendo todas sus partes. Esto es, aquí se
considerarán desde las horas dedicadas al aprendizaje, como las horas dedicadas
a la programación, como las horas dedicadas al diseño web y las horas dedicadas
a la redacción de la memoria y de la creación de la presentación de este
trabajo.

Para realizar la estimación de horas dedicadas, usaremos el \gls{ects}, que es
un estándar para comparar créditos académicos. Como es conocido, un crédito
\gls{ects} equivale a una dedicación de 25--30 horas. En este caso, al estar
realizando un bachelor thesis de dos titulaciones, sumaremos los créditos
destinados a cada titulación. Concretamente, para la titulación del grado en
\textsc{Enginyeria en Tecnologies Industrials}, son 12 créditos y para el grado
en \textsc{Matemàtiques} son 15 créditos. En total \SI{27}{\ectss}.

Por tanto, usando la equivalencia de 1 crédito \gls{ects} con 27,5 horas,
tenemos que el número de horas dedicado al trabajo será de,
\begin{equation}
  \SI{27}{\ectss} \times \frac{\SI{27.5}{\hour}}{\SI{1}{\ects}} =
  \SI{742.5}{\hour},
\end{equation}
lo cual, suponiendo un salario de \EUR{15}/hour, hace un total del coste
\emph{personal} de,
\begin{equation}
  \SI{742.5}{\hour} \times \frac{\text{\EUR{15}}}{\SI{1}{\hour}} =
  \text{\EUR{\num{11137.5}}}.
\end{equation}

\begin{remarkBox}
  Hay que tener en cuenta que esta estimación de \EUR{15}/hour es una
  \emph{aproximación}, para poder obtener un dato del dinero del coste
  personal. Claro está, que el número de horas dedicadas a formación tendrán
  una remuneración mucho más baja que la de las horas dedicadas a la creación
  del modelo y a la remuneración de las horas dedicadas al diseño web.
\end{remarkBox}

\subsubsection{Coste de Infraestructura}

Respecto al coste de \emph{infraesctructura}, solo será necesario incluir el
gasto realizado en servidores, ya que el resto de herramientas usadas son free
(as in \emph{freedom}) software, pero también free in terms of
price.\footnote{Aquí entran el uso de Python, PyTorch para la modelización; los
  lenguajes HTML, CSS, JS, PHP para la creación de la interfaz web; y \LaTeX\
  para la redacción de la memoria y la creacion dé la presentación.}

\subsubsection{Coste Total}

Por tanto, en total, se estima el coste de este trabajo en los \EUR{1221}.



\section{Enviromental Impact}

El impacto ambiental que este trabajo ha producido es mínimo, ya que se ha
tratado de un desarrollo de software. El único elemento que tiene sentido
considerar en este aspecto es el de uso de energía eléctrica para la
alimentación del ordenador y de los servidores, ya que la genración de esta
energía eléctrica llevará ciertas emisiones de \ch{CO2}.

Suponiendo un consumo medio aproximado del ordenador de \SI{150}{\watt}, y que
ha sido usado durante el total de las \SI{742.5}{\hour} que ha durado el
proyecto (see \vref{sec:cost}), tenemos que, a nivel energético, se han
consumido,
\begin{equation}
  \SI{150}{\watt} \times \SI{742.5}{\hour} = \SI{111.375}{\kWh}.
\end{equation}

Podemos ahora, usando la calculadora online de emisiones de \ch{CO2}, creada
por \myCite{aragon21:emiss_calcul}, concluir que las emisiones de \ch{CO2} son
de \SI{39}{\kg} de \ch{CO2}.

\begin{exampleBox}
  Estas emisiones de \ch{CO2} son las que emitiría un único coche gasolina
  durante un trayecto de \SI{200}{\km} (from \cite{aragon21:emiss_calcul}).
\end{exampleBox}

También podríamos considerar las emisiones de \ch{CO2} debidas al uso de
servidores durante el entrenamiento y del servidor web. Ahora bien, en el
primer caso, es difícil de cuantificar, ya que es un servidor multi-nodo con
usuarios. Y, en el segundo caso, es difícil cuantificar el uso del servidor
web, ya que está abierto al publico y depende del número de usuarios que entren
en la web.

En cualquier caso, como era de esperar, el impacto medioambiental de este
proyecto es mínimo.