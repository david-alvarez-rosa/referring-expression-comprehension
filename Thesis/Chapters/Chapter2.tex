% -*- TeX-master: "../Thesis.tex" -*-


\chapter{Theoretical Background}

\epigraphhead[75]{
  \epigraph{\itshape Begin at the beginning, the King said
    gravely, ``and go on till you come to the end: then stop.''}
  {---Lewis Carroll\\ \textit{Alice in Wonderland}}
}


\lettrine{A}{neural} network, more properly referred to as an \gls{ann} are
computing systems vaguely inspired by the biological neural networks that
constitute animal brains. Dr
\fhref{https://en.wikipedia.org/wiki/Robert_Hecht-Nielsen}{Robert
  Hecht-Nielsen} (inventor of one of the first neurocomputers) defines a neural
network as:

\begin{quoteBox}
  \itshape
  \ldots a computing system made up of a number of simple, highly interconnected
  processing elements, which process information by their dynamic state response
  to external inputs.
  \tcblower
  \hfill \upshape
  ---\href{https://en.wikipedia.org/wiki/Robert_Hecht-Nielsen}
  {Robert Hecht-Nielsen}
\end{quoteBox}

\subsubsection{Mathematical perspective}
Although the analogy made above of an \gls{ann} with a biological brain, there
is no need for this, we can just think of a neural network as a mathematical
optimization problem. We can think of the whole network to be a function that
takes some inputs to some outputs, and this function dependent on
parameters. The idea is to adjust this parameters to get a function that works
well with some known dataset, and we will trust that it will generalize
well. If the network is big enough and we carefully adjust the parameters, we
will be able to learn and calculate very complex functions.

\subsubsection{Tipos de redes neuronales}
Durante los últimos años, se han producido avances significativos en el ámbito
de la inteligencia artificial y han aparecido múltiples modelos nuevos de redes
neuronales. En una clasificación muy general de modelos neuronales podemos
encontrar diferentes.

\begin{itemize}
  \item \textbf{Fully connected layered neural networks}. Estas redes son el
  tipo de modelo neuronales maś sencillo que existe y sobre el que se
  construyen modelos más complejos. Una descripción más detallada sobre este
  tipo de redes puede ser encontrado en la sección~\ref{sec:fully-conn-layer}
  (página~\pageref{sec:fully-conn-layer}).
  \item \textbf{\acrlong*{cnn}}. Este tipo de redes neuronales son
  especialmente útiles en el procesado de imágenes, ya que permiten preservar
  la información espacial de relación entre píxeles. Serán estudiadas en la
  sección~\ref{sec:conv-neur-netw}
  (página~sección~\pageref{sec:conv-neur-netw}).
  \item \textbf{\acrlong*{rnn}}. Modelos neuronales ideales para el análisis
  de series temporales (como puede ser texto o audio). Serán estudiadas en la
  sección~\ref{sec:recurr-neur-netw}
  (página~sección~\pageref{sec:recurr-neur-netw}).
  \item \textbf{Transformers}. Los modelos basados en transformadores forman
  parte del \acrlong{sota} actual\footnote{En 2021, fecha actual del
    documento.}. Están siendo usados de manera extensiva en \gls{nlp} y
  estudiando sus posibles aplicaciones en visión por computadores y procesado
  de imágen. Serán descritos en la sección~\ref{sec:transformers}
  (página~\pageref{sec:transformers}).
\end{itemize}


\section{Fully Connected Layered Neural networks}
\label{sec:fully-conn-layer}

Las fully connected layered neural networks son el tipo de modelo neuronal más
sencillo que existe y que consitutye la base para estructuras neuronales más
complejas.

\subsection{Topology}
\label{sec:topology}

Este tipo de redes neuronales se caracteriza por una estructura de neuronales
organizada en diferentes capas, de manera que las neuronas entre capas
adyacentes están conectadas entre ellas por arcos. En la
figura~\ref{fig:fully-conn-network} se muestra un ejemplo de este tipo de
arquitectura.

\begin{figure}[ht]
  \centering
  % -*- TeX-master: "../../Thesis.tex" -*-


\begin{tikzpicture}[
  scale = 1.25,
  neuron/.style = {
    circle,
    draw,
    thick,
    fill = white,
    blur shadow
  },
  connection/.style = {
    -latex,
    black!60,
    thick
  },
  box/.style = {
    draw = black!80,
    thick,
    rounded corners
  }
  ]

  % Bounding box input layer.
  \draw[box, fill=red!20] (-.5, -1) rectangle (.5, -4) node[below, red,
  xshift=-16.5] {input layer};
  % Bounding box first hidden layer.
  \draw[box, fill=RoyalBlue!20] (2, -.5) rectangle (3, -4.5) node[below,
  RoyalBlue, xshift=-17.5] {hidden layer 1};
  % Bounding box second hidden layer.
  \draw[box, fill=RoyalPurple!20] (4.5, -.5) rectangle (5.5, -4.5)
  node[below, RoyalPurple, xshift=-17.5] {hidden layer 2};
  % Bounding box output layer.
  \draw[box, fill=OliveGreen!20] (7, -1.5) rectangle (8, -3.5) node[below,
  OliveGreen, xshift=-17.5] {output};

  % Neurons input layer.
  \begin{scope}[shift={(0, -.5)}]
    \foreach \name in {1,...,3} {
      \node[neuron] (I-\name) at (0, -\name) {$\mathbf{x}_\name^{1}$};
    }
  \end{scope}
  % Neurons first hidden layer.
  \foreach \name in {1,...,4} {
    \node[neuron] (H1-\name) at (2.5, -\name) {$\mathbf{x}_\name^{2}$};
  }
  % Neurons second hidden layer.
  \foreach \name in {1,...,4} {
    \node[neuron] (H2-\name) at (5, -\name) {$\mathbf{x}_\name^{3}$};
  }

  % Node output layer.
  \begin{scope}[shift={(0, -1)}]
    \foreach \name in {1,...,2} \node[neuron] (O-\name) at (7.5, -\name)
    {$\mathbf{x}_\name^{4}$};
  \end{scope}

  % Connections between input and first hidden layers.
  \foreach \source in {1,...,3} {
    \foreach \dest in {1,...,4} {
      \path[connection] (I-\source) edge (H1-\dest);
    }
  }
  % Connections between first and second hidden layers.
  \foreach \source in {1,...,4} {
    \foreach \dest in {1,...,4} {
      \path[connection] (H1-\source) edge (H2-\dest);
    }
  }
  % Connection between second hidden layer and output.
  \foreach \source in {1,...,4} {
    \foreach \dest in {1,...,2} {
      \path[connection] (H2-\source) edge (O-\dest);
    }
  }
\end{tikzpicture}

  \caption[Fully Connected Neural Network topology]{Ejemplo de fully connected
    layered neural network. Consta de 4 capas diferenciadas de neuronas y se
    muestran las conexiones entre ellas de manera gráfica con flechas.}
  \label{fig:fully-conn-network}
\end{figure}

A continuación discutiremos las partes principales constituyentes de este tipo
de redes neuronales: layers, neurons and connections.

\subsubsection{Layers}
The layers are just a collection of neurons, we will distinguish between three
types depending on its position in the network.

\begin{itemize}
  \item Input layer: patterns are presented to the network via this layer.
  \item Hidden layers: all the inner layers.
  \item Output layer: is the last layer, where the answer is obtained.
\end{itemize}

We will denote with \(L\) the number of layers and with \(n_l\) the size of the
\(l\)-th layer.

To gain some intuition on how this work, let's think about the handwritten
recognition problem. Suppose we have a set of images with handwritten digits in
it and that we will like to implement an ANN that is capable of recognizing
that digits. In this case, if the digits images are of \(28 \times 28\) pixels,
the input layer will consists of 784 (\(28 \times 28\)) neurons and each neuron
will hold the grayscale value of a pixel. For the output layer we will need 10
neurons (one for each number between 0 and 9), and we will like that when we
feed our network with an image holding a handwritten number 3, then the ouput
is one 1 in the position corresponding to the number 3 and the rest of zeros.
\todo{quizá borrar este párrafo que es innecesario}.

\subsubsection{Neurons}
Neurons are the core component of any neural network. Basically there are three
subparts that form a neuron.

\begin{itemize}
  \item Value: each neuron holds a value, it will be denoted by \(x_i^l \in \R\)
  for the \(i\)-th neuron in the \(l\)-th layer. Of course, it should be satisfied
  \(1 \leq i \leq n_l\). We will use use the notation \(\mathbf{x}^l\) for the
  vector of all the values in the \(l\)-th level. When we speak of the input
  vector, we may ommit the superindex, i.e.\ we will use \(\mathbf{x}\) to denote
  \(\mathbf{x}^0\). Similarly, for the output layer, we will use
  \(\mathbf{\hat{y}}\) to refer to \(\mathbf{x}^L\).
  \item Bias: also each neuron has a bias, denoted as \(b_i^l\) for the
  \(i\)-th neuron in the \(l\)-th layer. Is then true that
  \(1 \leq i \leq n_l\). The vector of all biases in the \(l\)-th layer will be
  denoted by \(\mathbf{b}^l\).
  \item Activation function: all neurons have an activation function
  \(f_i^l \in \mathcal{C}^1(\R, \R)\) for the \(i\)-th neuron in the \(l\)-th
  layer\footnote{Usually all the activation functions are neuron-independent
    (i.e.\ \(f_i^l\) does not really depend on \(i\) or \(l\)).}. Of course, it is
  needed \(1 \leq i \leq n_l\). The regularity assummed for this functions is
  important, since we will be optimizing in the future by taking derivatives.
\end{itemize}

\subsubsection{Connections}
As we discussed in the topology section (\ref{sec:topology}), all neurons
between adjacent layers are required to be connected, this are the connections,
that should have associated a \emph{weight}. For the connection between the
\(i\)-th neuron in the \(l\)-th layer and the \(j\)-th neuron in the
\(l + 1\)-th layer, we will denote this weight by \(w_{ij}^l \in \R\). The set of
all this weights is as follows,
\begin{equation}
   \{w_{ij}^l \mid 1 \leq i \leq n_{l}, \, 1 \leq j \leq n_{l+1}, \, 1 \leq l <
   L \} \subset \R.
 \end{equation}

The \emph{matrix} of all weights in the \(l\)-th layer will be denoted by
\(\mathbf{W}^l\). This is, \((\mathbf{W}^l)_{ij} = w_{ij}^l\).

\subsection{Forward}
\label{subsec:forward}

From now on let's suppose we are working with a fully-connected neural network,
with different layers and we will be using the same notation used before. The
values of the neurons can be computed with
\begin{equation}
  x_i^l = f_i^l \left( \sum_{k=1}^{n_{l-1}} w_{ik}^{l-1} x_{k}^{l-1} + b_i^l \right).
\end{equation}

This formula is sometimes referred to as the feed-forward formula or forward
propagation. It's important to note that it's a recursive formula, once the
values the neurons in the the input layer are known, we can iterate computing
the values of the neurons in the the next adjacent layer, until we reach the
output layer. In this network we can think of information travelling in one
direction, forward, from the input layer, through the hidden layers to the
output neurons.

\subsection{Backward}
\label{sec:backward}

In order to being able to train our neural network, is mandatory to define an
error function (also known as loss function) that quantifies how good or bad
the neural network is performing when feeded with a particular dataset. We will
you the below notation.
\begin{itemize}
  \item \textbf{Dataset}: will be denoted by and consists in input-output pairs
  \((\mathbf{x}, \mathbf{y})\), where \(\mathbf{x}\) represents the input and
  \(\mathbf{y}\) the \emph{desired} output. We shall denote the size (cardinal)
  of the dataset by \(N\). Of course, in terms of our ANN \(x\) corresponds to
  the values of the first (or input) layer and \(\mathbf{y}\) to the output (or
  last) layer.
  \item \textbf{Parameters}: the parameters of our ANN are both the connection
  weights \(w_{ij}^l\) and the biases \(b_i^l\), we will denote the set of all
  parameters by \(\bm{\theta}\). Keep in mind that \(\bm{\theta}\) is just a
  set of real vectors of \(\R^D\) where \(D\) denotes the number of weigths and
  biases.
\end{itemize}

The error function quantifies how different is the desired output
\(\mathbf{y}\) and the calculated (\emph{predicted}) output
\(\mathbf{\hat{y}}\) of the neural network on input \(\mathbf{x}\) for a set of
input-output pairs \((\mathbf{x}, \mathbf{y}) \in \Omega\) and a particular
value of the parameters \(\bm{\theta}\). We will denote the error funcion by
\(E(\Omega, \theta)\) and we will assume that is continuosly differentiable
(i.e.\ \(\mathcal{C}^1\)).


\section{Convolutional Neural Networks}
\label{sec:conv-neur-netw}

\glspl{cnn} son una clase de \glspl{dnn} que son ampliamente utilizadas en
computer vision.


\section{Recurrent Neural Networks}
\label{sec:recurr-neur-netw}


\section{Transformers}
\label{sec:transformers}


\section{Deep Neural Network Architectures}
