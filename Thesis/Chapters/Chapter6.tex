% -*- TeX-master: "../Thesis.tex" -*-


\chapter{Visualization}\label{cha:web}

\epigraphhead[75]{
  \epigraph{\itshape By visualizing information, we turn it into a landscape
    that you can explore with your eyes.}
  {---\scshape David McCandless}
}


\lettrine{L}{orem} ipsum dolor sit amet, consectetur
adipiscing elit, sed do eiusmod tempor incididunt ut labore et dolore magna
aliqua. Ut enim ad minim veniam, quis nostrud exercitation ullamco laboris nisi
ut aliquip ex ea commodo consequat. Duis aute irure dolor in reprehenderit in
voluptate velit esse cillum dolore eu fugiat nulla pariatur. Excepteur sint
occaecat cupidatat non proident, sunt in culpa qui officia deserunt mollit anim
id est laborum.


\section{Frontend: web interface}\label{sec:frontend}\index{Frontend}

In \vref{fig:web}.

\begin{figure}[ht]
  \centering
  \includegraphics[width=\textwidth]{Images/Web.png}
  \caption[TODO]{TODO.}
  \label{fig:web}
\end{figure}


\subsection{Accesibility}


\subsection{Options}


\section{Backend}\label{sec:backend}\index{Backend}

Aquí veremos la estructura del backend. Un gráfico explicativo se encuentra en
la \vref{fig:server}.

\begin{figure}[ht]
  \centering
  % -*- TeX-master: "../../Thesis.tex" -*-


\newcommand{\connectV}[2]{
  \draw[con] ([xshift = -10pt]#1.south) to ([xshift = -10pt]#2.north);
  \draw[con] ([xshift = 10pt]#2.north) to ([xshift = 10pt]#1.south);
}

\newcommand{\connectH}[2]{
  \draw[con] ([yshift = -8pt]#1.east) to ([yshift = -8pt]#2.west);
  \draw[con] ([yshift = 8pt]#2.west) to ([yshift = 8pt]#1.east);
}

\begin{tikzpicture}[
  box/.style = {
    rectangle,
    draw = #1!75,
    fill = #1!30,
    inner sep = 5pt,
    very thick
  },
  boxp/.style = {
    rectangle,
    draw = #1,
    fill = #1!60,
    inner sep = 5pt,
    very thick
  },
  cont/.style = {
    shape = rectangle,
    align = center,
    draw  = #1,
    fill  = #1!10,
    rounded corners,
    inner sep = 8pt
  },
  con/.style = {
    -stealth,
    very thick
  }
  ]
  % Client.
  \node[boxp=magenta] (UA) {User};
  \node[box=magenta] (UB) [right=of UA] {Browser};
  \begin{scope}[on background layer]
    \node[fit=(UA)(UB), cont=magenta, label={Client}] (U) {};
  \end{scope}

  % Web server.
  \node[boxp=gray] (WA) [below=1.25 of U] {Index};
  \node[box=gray] (WB) [left=.75 of WA] {Bootstrap};
  \node[box=gray] (WC) [right=.75 of WA] {\phantom{a}Navbar\phantom{a}};
  \node[box=gray] (WD) [below=.5 of WB] {JQuery};
  \node[box=gray] (WE) [below=.5 of WA] {Audio};
  \node[box=gray] (WF) [below=.5 of WC] {Footer};
  \node[box=gray] (WG) [below=.5 of WE] {Others\ldots};
  \begin{scope}[on background layer]
    \node[fit=(WA)(WB)(WC)(WD)(WE)(WF)(WG), cont=gray,
    label=above right:{Web Server}] (W) {};
  \end{scope}

  % API.
  \node[boxp=orange] (AA) [below=1.5 of W] {\Acs{api}};
  \node[box=orange] (AB) [below=.5 of AA] {Comprehend};
  \node[box=orange] (AC) [below=.5 of AB] {Audio};
  \begin{scope}[on background layer]
    \node[fit=(AA)(AB)(AC), cont=orange, label=below:{\Acs{api}}] (A) {};
  \end{scope}

  % Silero.
  \node[boxp=red] (SA) [right=2.5 of A, yshift=-.5cm] {\Acs{stt}};
  \node[box=red] (SB) [below=.5 of SA] {Silero};
  \begin{scope}[on background layer]
    \node[fit=(SA)(SB), cont=red, label={\Acl{stt}}] (S) {};
  \end{scope}

  % Comprehension.
  \node[boxp=blue] (CA) [left=2.5 of A, yshift=-.5cm] {\Acs{rec}};
  \node[box=blue] (CB) [below=.5 of CA] {Comprehend};
  \begin{scope}[on background layer]
    \node[fit=(CA)(CB), cont=blue, label={Comprehension}] (C) {};
  \end{scope}

  % Global connects.
  \connectV{U}{W};
  \connectV{W}{A};
  \connectH{C}{A};
  \connectH{A}{S};
\end{tikzpicture}

  \caption[TODO]{TODO.}
  \label{fig:server}
\end{figure}
