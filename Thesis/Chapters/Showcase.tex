% -*- TeX-master: "../Thesis.tex" -*-


\chapter{Showcase}

\epigraphhead[75]{
  \epigraph{\itshape Begin at the beginning, the King said
    gravely, ``and go on till you come to the end: then stop.''}
  {---Lewis Carroll\\ \textit{Alice in Wonderland}}
}


\lettrine{L}{orem} ipsum dolor sit amet, consectetur adipiscing elit, sed do
eiusmod tempor incididunt ut labore et dolore magna aliqua. Ut enim ad minim
veniam, quis nostrud exercitation ullamco laboris nisi ut aliquip ex ea commodo
consequat. Duis aute irure dolor in reprehenderit in voluptate velit esse cillum
dolore eu fugiat nulla pariatur. Excepteur sint occaecat cupidatat non proident,
sunt in culpa qui officia deserunt mollit anim id est laborum.


\section{Including code}

\codeLst{python}{../Code/refer.py}{Refer.py name}{label}


\section{Neural Network and Figure}

On the other hand, we denounce with righteous indignation and dislike men who
are so beguiled and demoralized by the charms of pleasure of the moment, so
blinded by desire, that they cannot foresee the pain and trouble that are bound
to ensue; and equal blame belongs to those who fail in their duty through
weakness of will, which is the same as saying through shrinking from toil and
pain.

A prime number (or a prime) is a natural number greater than 1 that is not a
product of two smaller natural numbers. A natural number greater than 1 that is
not prime is called a composite number. For example, 5 is prime because the only
ways of writing it as a product, 1 × 5 or 5 × 1, involve 5 itself. However, 4 is
composite because it is a product (2 × 2) in which both numbers are smaller than
4. Primes are central in number theory because of the fundamental theorem of
arithmetic: every natural number greater than 1 is either a prime itself or can
be factorized as a product of primes that is unique up to their order.

The property of being prime is called primality. A simple but slow method of
checking the primality of a given number n, called trial
division, tests whether n is a multiple of any integer between
2 and { { {n}}}{ {n}}. Faster algorithms include the
Miller–Rabin primality test, which is fast but has a small chance of error, and
the AKS primality test, which always produces the correct answer in polynomial
time but is too slow to be practical. Particularly fast methods are available
for numbers of special forms, such as Mersenne numbers. As of December 2018 the
largest known prime number is a Mersenne prime with 24,862,048 decimal
digits[1].

There are infinitely many primes, as demonstrated by Euclid around 300 BC. No
known simple formula separates prime numbers from composite numbers. However,
the distribution of primes within the natural numbers in the large can be
statistically modelled. The first result in that direction is the prime number
theorem, proven at the end of the 19th century, which says that the probability
of a randomly chosen number being prime is inversely proportional to its number
of digits, that is, to its logarithm.

\begin{figure}[ht]
  \centering
  % -*- TeX-master: "../../Thesis.tex" -*-


\begin{tikzpicture}[
  scale = 1.25,
  neuron/.style = {
    circle,
    draw,
    thick,
    fill = white,
    blur shadow
  },
  connection/.style = {
    -latex,
    black!60,
    thick
  },
  box/.style = {
    draw = black!80,
    thick,
    rounded corners
  }
  ]

  % Bounding box input layer.
  \draw[box, fill=red!20] (-.5, -1) rectangle (.5, -4) node[below, red,
  xshift=-16.5] {input layer};
  % Bounding box first hidden layer.
  \draw[box, fill=RoyalBlue!20] (2, -.5) rectangle (3, -4.5) node[below,
  RoyalBlue, xshift=-17.5] {hidden layer 1};
  % Bounding box second hidden layer.
  \draw[box, fill=RoyalPurple!20] (4.5, -.5) rectangle (5.5, -4.5)
  node[below, RoyalPurple, xshift=-17.5] {hidden layer 2};
  % Bounding box output layer.
  \draw[box, fill=OliveGreen!20] (7, -1.5) rectangle (8, -3.5) node[below,
  OliveGreen, xshift=-17.5] {output};

  % Neurons input layer.
  \begin{scope}[shift={(0, -.5)}]
    \foreach \name in {1,...,3} {
      \node[neuron] (I-\name) at (0, -\name) {$\mathbf{x}_\name^{1}$};
    }
  \end{scope}
  % Neurons first hidden layer.
  \foreach \name in {1,...,4} {
    \node[neuron] (H1-\name) at (2.5, -\name) {$\mathbf{x}_\name^{2}$};
  }
  % Neurons second hidden layer.
  \foreach \name in {1,...,4} {
    \node[neuron] (H2-\name) at (5, -\name) {$\mathbf{x}_\name^{3}$};
  }

  % Node output layer.
  \begin{scope}[shift={(0, -1)}]
    \foreach \name in {1,...,2} \node[neuron] (O-\name) at (7.5, -\name)
    {$\mathbf{x}_\name^{4}$};
  \end{scope}

  % Connections between input and first hidden layers.
  \foreach \source in {1,...,3} {
    \foreach \dest in {1,...,4} {
      \path[connection] (I-\source) edge (H1-\dest);
    }
  }
  % Connections between first and second hidden layers.
  \foreach \source in {1,...,4} {
    \foreach \dest in {1,...,4} {
      \path[connection] (H1-\source) edge (H2-\dest);
    }
  }
  % Connection between second hidden layer and output.
  \foreach \source in {1,...,4} {
    \foreach \dest in {1,...,2} {
      \path[connection] (H2-\source) edge (O-\dest);
    }
  }
\end{tikzpicture}

  \caption[My first figure short caption]{My first figure long caption. Morbi
    bibendum est aliquam, hendrerit dolor ac, pretium sem. Nunc molestie, dui in
    euis-mod finibus, nunc enim viverra enim, eu mattis mi metus id libero. Cras
    sed accumsan justo,ut volut.}
  \label{fig:label}
\end{figure}

On the other hand, we denounce with righteous indignation and dislike men who
are so beguiled and demoralized by the charms of pleasure of the moment, so
blinded by desire, that they cannot foresee the pain and trouble that are bound
to ensue; and equal blame belongs to those who fail in their duty through
weakness of will, which is the same as saying through shrinking from toil and
pain.\todo{Rewrite this paragraph\ldots}

A prime number (or a prime) is a natural number greater than 1 that is not a
product of two smaller\footnote{This is short explanation of what is
  happening here for the prime numbers set inside the real line} numbers. A
natural number greater than 1 that is not prime is called a composite
number. For example, 5 is prime because the only ways of writing it as a
product, 1 × 5 or 5 × 1, involve 5 itself. However, 4 is composite because it is
a product (2 × 2) in which both numbers are smaller than 4. Primes are central
in number theory because of the fundamental theorem of arithmetic: every natural
number greater than 1 is either a prime itself or can be factorized as a product
of primes that is unique up to their order.

The property of being prime is called primality. A simple but slow method of
checking the primality of a given number n, called trial division, tests whether
n is a multiple of any integer between\footnote{I will add a shorter footnote
  here.}  2 and { { {n}}}{ {n}}. Faster algorithms include the Miller–Rabin
primality test, which is fast but has a small chance of error, and the AKS
primality test, which always produces the correct answer in polynomial time but
is too slow to be practical. Particularly fast methods are available for numbers
of special forms, such as Mersenne numbers. As of December 2018 the largest
known prime number is a Mersenne prime with 24,862,048 decimal digits[1].

There are infinitely many primes, as demonstrated by Euclid around 300 BC. No
known simple formula separates prime numbers from composite numbers. However,
the distribution of primes within the natural numbers in the large can be
statistically modelled. The first result in that direction is the prime number
theorem, proven at the end of the 19th century, which says that the probability
of a randomly chosen number being prime is inversely proportional to its number
of digits, that is, to its logarithm.


\section{Table}

\begin{table}[ht]
  \centering
  \renewcommand{\arraystretch}{1.25}
  \setlength{\tabcolsep}{1.5\tabcolsep}
  \rowcolors{2}{white}{gray!30}
  \caption{Información experimental y calculada. On the other hand, we denounce
    with righteous indignation and dislike men who are so beguiled and
    demoralized by the charms of pleasure of the moment, some.}
  \label{tab:info}
  \begin{tabular}{*7c} \toprule
    Medida & $n$ & $T_H$ & $T_L$ & $I$ & $U$ & $P_{el}$ \\
    \midrule
    1  & 515 & 170 & 55,6 & 19,2  & 6,88 & 132    \\
    2  & 520 & 169 & 58,2 & 21,2  & 6,73 & 142,7  \\
    3  & 530 & 169 & 60,1 & 23,9  & 6,7  & 160    \\
    4  & 530 & 190 & 62,9 & 27    & 6,5  & 175,5  \\
    5  & 518 & 186 & 63,6 & 30,5  & 6,2  & 189,1  \\
    6  & 500 & 178 & 64   & 34,2  & 5,7  & 194,9  \\
    7  & 484 & 187 & 64,4 & 38,4  & 5,1  & 207,36 \\
    \bottomrule
  \end{tabular}
\end{table}


\section{Comment}

\begin{comment}
  \begin{hola}[index=zapto!hoasd]{Principio de seguridad de los
      alimentos.}{articulo-8}
    % \begin{enumerate}[leftmargin = *]
    %   \item Conforme a lo requerido en el artículo 14 del Reglamento (CE) n.º
    %   178/2002, solo podrán comercializarse alimentos y piensos que en
    %   condiciones
    %   de uso normales, sean seguros.
    %   \item Para determinar que un alimento es seguro, además de lo previsto
    %   en el
    %   artículo 14.3 del referido Reglamento, se tendrán también en cuenta los
    %   posibles efectos, por la sensibilidad particular de una categoría
    %   específica
    %   de consumidores, cuando el alimento esté destinado a ella.
    %   [\,\textbf{\ldots}]
    % \end{enumerate}
    On the other hand, we denounce with righteous indignation and dislike men
    who are so beguiled and demoralized by the charms of pleasure of the moment,
    so blinded by desire, that they cannot foresee the pain and trouble that are
    bound to ensue; and equal blame belongs to those who fail in their duty
    through weakness of will, which is the same as saying through shrinking from
    toil and pain.

    \tcblower

    \hfill \texttt{Ley 17/2011, de 5 de julio, de Seguridad Alimentaria y
      Nutrición}

    \hfill \textit{Fuente}: \url{https://www.boe.es/}
  \end{hola}
\end{comment}


\section{Lot of text}

On the other hand, we denounce with righteous indignation and dislike men who
are so beguiled and demoralized by the charms of pleasure of the moment, so
blinded by desire, that they cannot foresee the pain and trouble that are bound
to ensue; and equal blame belongs to those who fail in their duty through
weakness of will, which is the same as saying through shrinking from toil and
pain.

A prime number (or a prime) is a natural number greater than 1 that is not a
product of two smaller natural numbers. A natural number greater than 1 that is
not prime is called a composite number. For example, 5 is prime because the only
ways of writing it as a product, 1 × 5 or 5 × 1, involve 5 itself. However, 4 is
composite because it is a product (2 × 2) in which both numbers are smaller than
4. Primes are central in number theory because of the fundamental theorem of
arithmetic: every natural number greater than 1 is either a prime itself or can
be factorized as a product of primes that is unique up to their order.

The property of being prime is called primality. A simple but slow method of
checking the primality of a given number n, called trial
division, tests whether n is a multiple of any integer between
2 and { { {n}}}{ {n}}. Faster algorithms include the
Miller–Rabin primality test, which is fast but has a small chance of error, and
the AKS primality test, which always produces the correct answer in polynomial
time but is too slow to be practical. Particularly fast methods are available
for numbers of special forms, such as Mersenne numbers. As of December 2018 the
largest known prime number is a Mersenne prime with 24,862,048 decimal
digits[1].

There are infinitely many primes, as demonstrated by Euclid around 300 BC. No
known simple formula separates prime numbers from composite numbers. However,
the distribution of primes within the natural numbers in the large can be
statistically modelled. The first result in that direction is the prime number
theorem, proven at the end of the 19th century, which says that the probability
of a randomly chosen number being prime is inversely proportional to its number
of digits, that is, to its logarithm.

Several historical questions regarding prime numbers are still unsolved. These
include Goldbach's conjecture, that every even integer greater than 2 can be
expressed as the sum of two primes, and the twin prime conjecture, that there
are infinitely many pairs of primes having just one even number between
them. Such questions spurred the development of various branches of number
theory, focusing on analytic or algebraic aspects of numbers. Primes are used in
several routines in information technology, such as public-key cryptography,
which relies on the difficulty of factoring large numbers into their prime
factors. In abstract algebra, objects that behave in a generalized way like
prime numbers include prime elements and prime ideals.

On the other hand, we denounce with righteous indignation and dislike men who
are so beguiled and demoralized by the charms of pleasure of the moment, so
blinded by desire, that they cannot foresee the pain and trouble that are bound
to ensue; and equal blame belongs to those who fail in their duty through
weakness of will, which is the same as saying through shrinking from toil and
pain.

A prime number (or a prime) is a natural number greater than 1 that is not a
product of two smaller natural numbers. A natural number greater than 1 that is
not prime is called a composite number. For example, 5 is prime because the only
ways of writing it as a product, 1 × 5 or 5 × 1, involve 5 itself. However, 4 is
composite because it is a product (2 × 2) in which both numbers are smaller than
4. Primes are central in number theory because of the fundamental theorem of
arithmetic: every natural number greater than 1 is either a prime itself or can
be factorized as a product of primes that is unique up to their order.

The property of being prime is called primality. A simple but slow method of
checking the primality of a given number n, called trial
division, tests whether n is a multiple of any integer between
2 and { { {n}}}{ {n}}. Faster algorithms include the
Miller–Rabin primality test, which is fast but has a small chance of error, and
the AKS primality test, which always produces the correct answer in polynomial
time but is too slow to be practical. Particularly fast methods are available
for numbers of special forms, such as Mersenne numbers. As of December 2018 the
largest known prime number is a Mersenne prime with 24,862,048 decimal
digits[1].

There are infinitely many primes, as demonstrated by Euclid around 300 BC. No
known simple formula separates prime numbers from composite numbers. However,
the distribution of primes within the natural numbers in the large can be
statistically modelled. The first result in that direction is the prime number
theorem, proven at the end of the 19th century, which says that the probability
of a randomly chosen number being prime is inversely proportional to its number
of digits, that is, to its logarithm.

Several historical questions regarding prime numbers are still unsolved. These
include Goldbach's conjecture, that every even integer greater than 2 can be
expressed as the sum of two primes, and the twin prime conjecture, that there
are infinitely many pairs of primes having just one even number between
them. Such questions spurred the development of various branches of number
theory, focusing on analytic or algebraic aspects of numbers. Primes are used in
several routines in information technology, such as public-key cryptography,
which relies on the difficulty of factoring large numbers into their prime
factors. In abstract algebra, objects that behave in a generalized way like
prime numbers include prime elements and prime ideals.

On the other hand, we denounce with righteous indignation and dislike men who
are so beguiled and demoralized by the charms of pleasure of the moment, so
blinded by desire, that they cannot foresee the pain and trouble that are bound
to ensue; and equal blame belongs to those who fail in their duty through
weakness of will, which is the same as saying through shrinking from toil and
pain.

A prime number (or a prime) is a natural number greater than 1 that is not a
product of two smaller natural numbers. A natural number greater than 1 that is
not prime is called a composite number. For example, 5 is prime because the only
ways of writing it as a product, 1 × 5 or 5 × 1, involve 5 itself. However, 4 is
composite because it is a product (2 × 2) in which both numbers are smaller than
4. Primes are central in number theory because of the fundamental theorem of
arithmetic: every natural number greater than 1 is either a prime itself or can
be factorized as a product of primes that is unique up to their order.

The property of being prime is called primality. A simple but slow method of
checking the primality of a given number n, called trial
division, tests whether n is a multiple of any integer between
2 and { { {n}}}{ {n}}. Faster algorithms include the
Miller–Rabin primality test, which is fast but has a small chance of error, and
the AKS primality test, which always produces the correct answer in polynomial
time but is too slow to be practical. Particularly fast methods are available
for numbers of special forms, such as Mersenne numbers. As of December 2018 the
largest known prime number is a Mersenne prime with 24,862,048 decimal
digits[1].

There are infinitely many primes, as demonstrated by Euclid around 300 BC. No
known simple formula separates prime numbers from composite numbers. However,
the distribution of primes within the natural numbers in the large can be
statistically modelled. The first result in that direction is the prime number
theorem, proven at the end of the 19th century, which says that the probability
of a randomly chosen number being prime is inversely proportional to its number
of digits, that is, to its logarithm.

Several historical questions regarding prime numbers are still unsolved. These
include Goldbach's conjecture, that every even integer greater than 2 can be
expressed as the sum of two primes, and the twin prime conjecture, that there
are infinitely many pairs of primes having just one even number between
them. Such questions spurred the development of various branches of number
theory, focusing on analytic or algebraic aspects of numbers. Primes are used in
several routines in information technology, such as public-key cryptography,
which relies on the difficulty of factoring large numbers into their prime
factors. In abstract algebra, objects that behave in a generalized way like
prime numbers include prime elements and prime ideals.

On the other hand, we denounce with righteous indignation and dislike men who
are so beguiled and demoralized by the charms of pleasure of the moment, so
blinded by desire, that they cannot foresee the pain and trouble that are bound
to ensue; and equal blame belongs to those who fail in their duty through
weakness of will, which is the same as saying through shrinking from toil and
pain.

A prime number (or a prime) is a natural number greater than 1 that is not a
product of two smaller natural numbers. A natural number greater than 1 that is
not prime is called a composite number. For example, 5 is prime because the only
ways of writing it as a product, 1 × 5 or 5 × 1, involve 5 itself. However, 4 is
composite because it is a product (2 × 2) in which both numbers are smaller than
4. Primes are central in number theory because of the fundamental theorem of
arithmetic: every natural number greater than 1 is either a prime itself or can
be factorized as a product of primes that is unique up to their order.

The property of being prime is called primality. A simple but slow method of
checking the primality of a given number n, called trial
division, tests whether n is a multiple of any integer between
2 and { { {n}}}{ {n}}. Faster algorithms include the
Miller–Rabin primality test, which is fast but has a small chance of error, and
the AKS primality test, which always produces the correct answer in polynomial
time but is too slow to be practical. Particularly fast methods are available
for numbers of special forms, such as Mersenne numbers. As of December 2018 the
largest known prime number is a Mersenne prime with 24,862,048 decimal
digits[1].

There are infinitely many primes, as demonstrated by Euclid around 300 BC. No
known simple formula separates prime numbers from composite numbers. However,
the distribution of primes within the natural numbers in the large can be
statistically modelled. The first result in that direction is the prime number
theorem, proven at the end of the 19th century, which says that the probability
of a randomly chosen number being prime is inversely proportional to its number
of digits, that is, to its logarithm.

Several historical questions regarding prime numbers are still unsolved. These
include Goldbach's conjecture, that every even integer greater than 2 can be
expressed as the sum of two primes, and the twin prime conjecture, that there
are infinitely many pairs of primes having just one even number between
them. Such questions spurred the development of various branches of number
theory, focusing on analytic or algebraic aspects of numbers. Primes are used in
several routines in information technology, such as public-key cryptography,
which relies on the difficulty of factoring large numbers into their prime
factors. In abstract algebra, objects that behave in a generalized way like
prime numbers include prime elements and prime ideals.

On the other hand, we denounce with righteous indignation and dislike men who
are so beguiled and demoralized by the charms of pleasure of the moment, so
blinded by desire, that they cannot foresee the pain and trouble that are bound
to ensue; and equal blame belongs to those who fail in their duty through
weakness of will, which is the same as saying through shrinking from toil and
pain.

A prime number (or a prime) is a natural number greater than 1 that is not a
product of two smaller natural numbers. A natural number greater than 1 that is
not prime is called a composite number. For example, 5 is prime because the only
ways of writing it as a product, 1 × 5 or 5 × 1, involve 5 itself. However, 4 is
composite because it is a product (2 × 2) in which both numbers are smaller than
4. Primes are central in number theory because of the fundamental theorem of
arithmetic: every natural number greater than 1 is either a prime itself or can
be factorized as a product of primes that is unique up to their order.

The property of being prime is called primality. A simple but slow method of
checking the primality of a given number n, called trial division, tests whether
n is a multiple of any integer between 2 and { { {n}}}{ {n}}. Faster algorithms
include the Miller–Rabin primality test, which is fast but has a small chance of
error, and the AKS primality test, which always produces the correct answer in
polynomial time but is too slow to be practical. Particularly fast methods are
available for numbers of special forms, such as Mersenne numbers. As of December
2018 the largest known prime number is a Mersenne prime with 24,862,048 decimal
digits[1].

There are infinitely many primes, as demonstrated by Euclid around 300 BC. No
known simple formula separates prime numbers from composite numbers. However,
the distribution of primes within the natural numbers in the large can be
statistically modelled. The first result in that direction is the prime number
theorem, proven at the end of the 19th century, which says that the probability
of a randomly chosen number being prime is inversely proportional to its number
of digits, that is, to its logarithm.

Several historical questions regarding prime numbers are still unsolved. These
include Goldbach's conjecture, that every even integer greater than 2 can be
expressed as the sum of two primes, and the twin prime conjecture, that there
are infinitely many pairs of primes having just one even number between
them. Such questions spurred the development of various branches of number
theory, focusing on analytic or algebraic aspects of numbers. Primes are used in
several routines in information technology, such as public-key cryptography,
which relies on the difficulty of factoring large numbers into their prime
factors. In abstract algebra, objects that behave in a generalized way like
prime numbers include prime elements and prime ideals.
