% -*- TeX-master: "../Thesis.tex" -*-


\chapter{Implementation Details}\label{cha:code}

\epigraphhead[75]{
  \epigraph{\itshape{}Talk is cheap. Show me the code.}
  {---Linus \textsc{Torvalds}}
}

\lettrine{T}{he most representative details} about the implementation and
programming of this project will be collected in this chapter. The code files
that collect the general ideas used in this work will be exposed verbatim, but
various existing auxiliary files will not be shown here to avoid being extended
too much. The curious reader can consult the entire bulk of the code used and
its evolution in the
\fhref{https://gitlab.com/david-alvarez-rosa/bachelor-thesis}{official
  repository} of this project (also consult
\fhref{https://recomprehension.com}{web} for more information). This chapter
will divide itself into the code files---mainly Python model implementation
---in\ \vref{sec:code-files}, the web server-related implementation (see\
\vref{sec:code-web}), and files related to using servers (see\
\vref{sec:code-server}).


\section{Code Files}\label{sec:code-files}

Among the different files used for the implementation, training and testing of
the model are those shown below. Many more than these files have been used,
since the model creation process has been an iterative process, in which
various modifications of the base model have been tested.

Below is the code used to carry out the testing of the model.

\codeInFull{python}{../Code/test.py}

Below is the code used for training the different versions of the model (keep
in mind that different parts of files are parameter dependent). Actually
multiple versions of this same file have been used, since different parts of it
have been modified.

\codeInFull{python}{../Code/train.py}

To access the datasets used during this work, it is necessary to use the
following file as \gls{api}.

\codeInFull{python}{../Code/refer.py}

For the use of the model, it has been useful to create a file in Python
containing the same model as an object. Attached below.

\codeInFull{python}{../Code/model.py}



\section{Website}\label{sec:code-web}

In relation to the website, we will show the most important files created. We
will separate between the front end (see\ \vref{sec:code-front}) and the back
end (see\ \vref{sec:code-back})

\subsection{Front End}\label{sec:code-front}

Within the front end the main file is obviously \code{index.html}, but due to
its extension it has not been included. It also doesn't bring too much extra
functionality to work. Yes, the stylesheet \acs{css} is included below.

\codeInFull{css}{../Website/css/main.css}

In addition, as a fundamental part of the interactivity of the web, the file
containing the code of \acs{js} is fundamental, which makes the requests to the
\gls{api} of the back end to collect the information.

\codeInFull{js}{../Website/js/main.js}


\subsection{Back End}\label{sec:code-back}

In the back end highlighting two files, which are the ones that really are
\gls{api}. The first is the one that deals with actually performing the main
task of this work, that is, segmentation.

\codeInFull{php}{../Website/api/comprehend.php}

The following file constitutes the part of the \gls{api} of the back end is the
one for converting audio to text, which is shown below.

\codeInFull{php}{../Website/api/uploadWav.php}

In addition, the Python files that are executed in the back end after being
called by the different functions of \gls{api} are added below. These are,
\code{comprehend.py} and \code{silero.py}.

\codeInFull{python}{../Code/comprehend.py}

\codeInFull{python}{../Code/silero.py}



\section{Server}\label{sec:code-server}

For the connection and use of the server, the main files that have been
necessary are shown below. The first one is the script used to synchronize
files between the local computer and the remote servers, attached below.

\codeInFull{bash}{../Utils/newServer}

In addition, the server makes use of a management system called Slurm, which is
an open-source job scheduler so that it is possible to make use of the
computational resources of multiple servers by numerous users and do all this
in an orderly manner. For this, this program has a very specific syntax with
which to execute the desired code. A typical script for this type of task is
shown below.

\codeInFull{bash}{../Utils/launch}
