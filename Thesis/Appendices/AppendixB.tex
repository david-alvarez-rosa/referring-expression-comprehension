% -*- TeX-master: "../Thesis.tex" -*-


\chapter{Implementation Details}\label{cha:code}

\epigraphhead[75]{
  \epigraph{\itshape{}Talk is cheap. Show me the code.}
  {---Linus \textsc{Torvalds}}
}


\lettrine{L}{orem ipsum dolor} sit amet, consectetur adipiscing elit, sed do
eiusmod tempor incididunt ut labore et dolore magna aliqua. Ut enim ad minim
veniam, quis nostrud exercitation ullamco laboris nisi ut aliquip ex ea commodo
consequat. Duis aute irure dolor in reprehenderit in voluptate velit esse
cillum dolore eu fugiat nulla pariatur. Excepteur sint occaecat cupidatat non
proident, sunt in culpa qui officia deserunt mollit anim id est laborum.

TODO. Añadir aquí los archivos de código más representativos.



\section{Code}

\codeInFull{python}{../Code/test.py}

\codeInFull{python}{../Code/train.py}

\codeInFull{python}{../Code/refer.py}

\codeInFull{python}{../Code/model.py}



\section{Website}

En relación a la website, mostraremos los archivos más importantes
creados. Separaremos entre el front end (see \vref{sec:code-front}) y el back
end (see \vref{sec:code-back})

\subsection{Front End}\label{sec:code-front}

Dentro del front end el archivo más importante es el de la página principal
(\acs{html}), que es el \code{index.html}.

\codeInFull{html}{../Website/index.html}

También añadimos la hoja de estilos \acs{css}, \code{css/main.css}.

\codeInFull{css}{../Website/css/main.css}

Además, como parte fundamental de la interactividad de la web, es fundamental
el archivo de \acs{js}, \code{js/main.js}

\codeInFull{js}{../Website/js/main.js}


\subsection{Back End}\label{sec:code-back}

En el back end destacand dos archivos, que son los que realmente son la
\gls{api}. Entre ellos


\codeInFull{php}{../Website/api/comprehend.php}

% \codeInFull{php}{../Website/api/upload\_wav.php}

Además, a continuación se añaden los archivos de Python que se ejecutan en el
back end tras ser llamados por las diferentes funciones de la \gls{api}. Estos
son, \code{comprehend.py} y \code{silero.py}.

\codeInFull{python}{../Code/comprehend.py}

\codeInFull{python}{../Code/silero.py}



\section{Server}
Comentar aquí qué servidor se ha escogido y cómo se ha configurado con PHP y
demás para funcionar de manera correcta en los servidores.


\codeInFull{bash}{../Utils/newServer}

\codeInFull{bash}{../Utils/launch}
