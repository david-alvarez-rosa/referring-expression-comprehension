% -*- TeX-master: "../Thesis.tex" -*-


\chapter{Introduction}\label{cha:introduction}

\epigraphhead[75]{
  \epigraph{\itshape Begin at the beginning, the King said
    gravely, ``and go on till you come to the end: then stop.''}
  {---\textsc{Lewis Carroll}\\ \textit{Alice in Wonderland}}
}


\lettrine{L}{orem ipsum dolor} sit amet, consectetur adipiscing elit, sed do
eiusmod tempor incididunt ut labore et dolore magna aliqua. Ut enim ad minim
veniam, quis nostrud exercitation ullamco laboris nisi ut aliquip ex ea commodo
consequat. Duis aute irure dolor in reprehenderit in voluptate velit esse
cillum dolore eu fugiat nulla pariatur. Excepteur sint occaecat cupidatat non
proident, sunt in culpa qui officia deserunt mollit anim id est laborum.

Here I present a complete list of all the papers I have read for developing this
thesis. It's similar to a commented bibliography reference.


\section{Activation functions}
Here a list of activation functions.


\begin{itemize}
  \item \textbf{\acf{relu}.}
  \item \textbf{Hyperbolic tangent.}
  \item \textbf{Sigmoid function.}
\end{itemize}

\begin{figure}[ht]
  \begin{subfigure}[b]{.475\textwidth}
    \centering
    \begin{tikzpicture}
      \begin{axis}[
        activationFunction,
        xmin = -1, xmax = 3,
        ymin = 0, ymax = 3]
        \addplot[myPlot] {max(0, x)};
        \addlegendentry{\(f(x) = \max\{0, x\}\)}
      \end{axis}
    \end{tikzpicture}
    \caption{Activation function.}
  \end{subfigure}\hfill
  \begin{subfigure}[b]{.475\textwidth}
    \centering
    \begin{tikzpicture}
      \begin{axis}[
        activationFunction,
        xmin = -1, xmax = 3,
        ymin = 0, ymax = 3]
        \addplot[myPlot, red, sharp plot, samples at={-5, -1e-6, 1e-6, 5}] {x>=0};
        \addlegendentry{\(f^\prime(x) = \mathbf{1}_{\R^+}(x)\)}
      \end{axis}
    \end{tikzpicture}
    \caption{Activation function derivative.}
  \end{subfigure}
  \caption[\glsentrylong{relu} and derivative]{\acf{relu} activation function.}
\end{figure}

\begin{figure}[ht]
  \begin{subfigure}[b]{.475\textwidth}
    \centering
    \begin{tikzpicture}
      \begin{axis}[
        activationFunction,
        xmin = -3, xmax = 3,
        ymin = -1, ymax = 1,
        ytick distance = .5]
        \addplot[myPlot] {tanh(x)};
        \addlegendentry{\(f(x) = \tanh x\)}
      \end{axis}
    \end{tikzpicture}
    \caption{Activation function.}
  \end{subfigure}\hfill
  \begin{subfigure}[b]{.475\textwidth}
    \centering
    \begin{tikzpicture}
      \begin{axis}[
        activationFunction,
        xmin = -3, xmax = 3,
        ymin = -1, ymax = 1,
        ytick distance = .5]
        \addplot[myPlot, red] {1 - tanh(x)^2};
        \addlegendentry{\(f^\prime(x) = 1 - \tanh^2 x\)}
      \end{axis}
    \end{tikzpicture}
    \caption{Activation function derivative.}
  \end{subfigure}
  \caption[Hyperbolic tangent and derivative]{Hyperbolic tangent and derivative.
  TODO.}
\end{figure}

\begin{figure}[ht]
  \begin{subfigure}[b]{.475\textwidth}
    \centering
    \begin{tikzpicture}
      \begin{axis}[
        activationFunction,
        xmin = -4, xmax = 4,
        ymin = 0, ymax = 1,
        xtick distance = 2,
        ytick distance = .5]
        \addplot[myPlot, domain=-8:8] {(1 + e^(-x))^(-1)};
        \addlegendentry{\(\sigma(x) = {(1 + e^{-x})}^{-1}\)}
      \end{axis}
    \end{tikzpicture}
    \caption{Activation function.}
  \end{subfigure}\hfill
  \begin{subfigure}[b]{.475\textwidth}
    \centering
    \begin{tikzpicture}
      \begin{axis}[
        activationFunction,
        xmin = -4, xmax = 4,
        ymin = 0, ymax = 1,
        xtick distance = 2,
        ytick distance = .5]
        \addplot[myPlot, domain=-8:8, red] {(1 + e^(-x))^(-1)*(1 - (1 + e^(-x))^(-1))};
        \addlegendentry{\(\sigma^\prime(x) = \sigma(x)(1 - \sigma(x))\)}
      \end{axis}
    \end{tikzpicture}
    \caption{Activation function derivative.}
  \end{subfigure}
  \caption[Sigmoid function and derivative]{Sigmoid function and derivative
    (also called logistic function and soft step). TODO.}
\end{figure}

\begin{figure}[ht]
  \begin{subfigure}[b]{.475\textwidth}
    \centering
    \begin{tikzpicture}
      \begin{axis}[
        activationFunction,
        xmin = -3, xmax = 3,
        ymin = -1, ymax = 1.5]
        \addplot[myPlot] {max(0, x)};
        \addlegendentry{\(f(x) = \max\{0, x\}\)}
        \addplot[myPlot, red] {tanh(x)};
        \addlegendentry{\(g(x) = \tanh x\)}
        \addplot[myPlot, green] {(1 + e^(-x))^(-1)};
        \addlegendentry{\(\sigma(x) = {(1 + e^{-x})}^{-1}\)}
      \end{axis}
    \end{tikzpicture}
    \caption{TODO.}
  \end{subfigure}\hfill
  \begin{subfigure}[b]{.475\textwidth}
    \centering
    \begin{tikzpicture}
      \begin{axis}[
        activationFunction,
        xmin = -3, xmax = 3,
        ymin = -1, ymax = 1.5]
        \addplot[myPlot, sharp plot, samples at={-5, -1e-6, 1e-6, 5}] {x>=0};
        \addlegendentry{\(f^\prime(x) = \mathbf{1}_{\R^+}(x)\)}
        \addplot[myPlot, red] {1 - tanh(x)^2};
        \addlegendentry{\(g^\prime(x) = 1 - \tanh^2 x\)}
        \addplot[myPlot, green] {(1 + e^(-x))^(-1)*(1 - (1 + e^(-x))^(-1))};
        \addlegendentry{\(\sigma^\prime(x) = \sigma(x)(1 - \sigma(x))\)}
      \end{axis}
    \end{tikzpicture}
    \caption{TODO.}
  \end{subfigure}
  \caption[TODO]{TODO.}
\end{figure}
