% -*- TeX-master: "../Thesis.tex" -*-


\chapter{File Structure}\label{cha:files}

\epigraphhead[75]{
  \epigraph{\itshape Organizing files is like organizing your room: \\
    it should be clean and easy to navigate through.}
  {---George \textsc{Sun}}
}


\lettrine{D}{uring the development} of this work, different code files have
been created and used in different programming languages. Establishing a
consistent, logical, clear and easy-to-navigate file structure has been a
critical element in facilitating the creation and editing of these files. In
addition, Git has been used as a version control system for tracking changes in
the text and code files. If the reader wishes, he can consult all the files in
the project
\fhref{https://gitlab.com/david-alvarez-rosa/bachelor-thesis}{repository}. Below
is the first level main structure.\medskip

\dirtree{%
  .1 \myDir{Bachelor's Thesis}.
  .2 \myDir{Code}.
  .2 \myDir{Datasets}.
  .2 \myDir{Thesis}.
  .2 \myDir{Website}.
  .2 \myGit{.gitignore}.
  .2 \myOrg{README}.
}\medskip

The total set of files is divided into 4 large groups that will be explored
individually: the code directory (see \vref{sec:code}), the datasets directory
(see \vref{sec:datasets-dir}), the directory containing the code of this thesis
(see \vref{sec:thesis}) and the website directory (see \vref{sec:website}).



\section{Code}\label{sec:code}

Inside this directory you will find all the code related to the implementation
of the model used, as well as files for the train and test of the model. Also,
here you will find the files for reading the dataset and those that are
executed in the backend of the server (they are called by the \gls{api} in
PHP). The files used for the iterations of the different models are also
collected here (some of these files should possibly be retrieved from the Git
history).\medskip

\dirtree{%
  .1 \myDir{Code}.
  .2 \myDir{.venv}.
  .3 {* Python virtual environment}.
  .2 \myDir{checkpoints}.
  .3 \myTorch{model.pth}.
  .3 {* Possibly other checkpoints}.
  .2 \myDir{lib}.
  .3 \myPy{\_utils}.
  .3 \myPy{deeplabv3\_emb}.
  .3 \myPy{segmentation}.
  .2 \myGit{.gitignore}.
  .2 \myPy{args}.
  .2 \myPy{compehend}.
  .2 \myPy{dataset}.
  .2 \myPy{model}.
  .2 \myOrg{README}.
  .2 \myPy{refer}.
  .2 \myPy{test}.
  .2 \myPy{train}.
  .2 \myPy{transforms}.
  .2 \myPy{utils}.
}



\section{Datasets}\label{sec:datasets-dir}

The datasets used in their original format will be stored in this
directory.\medskip

\dirtree{%
  .1 \myDir{Datasets}.
  .2 \myDir{refcoco}.
  .3 \myDir{images}.
  .3 \myJson{annotations}.
  .3 \myPy{refs(google).p}.
  .3 \myPy{refs(unc).p}.
  .2 \myDir{refcoco+}.
  .3 {* Similar to refcoco \(\uparrow\)}.
  .2 \myDir{refcocog}.
  .3 {* Similar to refcoco \(\uparrow\)}.
  .2 \myOrg{README}.
}



\section{Thesis}\label{sec:thesis}

This directory contains all the files that this thesis typed with \LaTeX{}
makes possible: source code, images, vector graphics, acronyms, references,
etc.\medskip

\dirtree{%
  .1 \myDir{Thesis}.
  .2 \myDir{Appendices}.
  .3 \myTex{Appendix\{A,B\}}.
  .2 \myDir{Build}.
  .3 {* Externalized build files}.
  .2 \myDir{Chapters}.
  .3 \myTex{Chapter\{1,2, \ldots,8\}}.
  .2 \myDir{Figures}.
  .3 \myDir{Images}.
  .4 {* Document images}.
  .3 \myDir{Logos}.
  .4 {* University logos}.
  .3 \myDir{Icons}.
  .4 {* Program icons}.
  .3 \myDir{SVGs}.
  .3 {* Document SVGs}.
  .3 \myDir{Tikz}.
  .4 {* Vector graphics}.
  .2 \myDir{FrontBack}.
  .3 \myTex{FrontMatter}.
  .3 \myTex{BackMatter}.
  .2 \myDir{Utils}.
  .3 \myTex{Acronyms}.
  .3 \myTex{EasterEgg}.
  .3 \myFile{Index.ist}.
  .3 \myTex{Preamble}.
  .3 \myBib{References}.
  .2 \myGit{.gitignore}.
  .2 \myOrg{README}.
  .2 \myTex{Thesis}.
}



\section{Utils}\label{sec:utils}

Directory containing different files useful for various functions. Among them
are the shell executables to synchronize files with the remote server for
training and with the web server. Slurm\footnote{Slurm is an free and
  open-source job scheduler that is used in the servers provided by
  VectorInstitute.} configurations files used are also present here.\medskip

\dirtree{%
  .1 \myDir{Utils}.
  .2 \myGit{.gitignore}.
  .2 \myPy{createSmall}.
  .2 \myPy{getImgNames}.
  .2 \mySlurm{launch}.
  .2 \myBash{newServer}.
  .2 \myBash{testAPI}.
  .2 \myBash{vector}.
  .2 \myBash{vectorInteractive}.
}



\section{Website}\label{sec:website}

This directory is an exact replica of the one on the web server, and contains
all the files necessary for its operation.\medskip

\dirtree{%
  .1 \myDir{Website}.
  .2 \myDir{api}.
  .3 \myDir{.cache}.
  .4 {* PyTorch cache files}.
  .3 \myDir{audio}.
  .4 {* Temp audio files}.
  .3 \myDir{results}.
  .4 {* Temp output images}.
  .3 \myLink{Code -> ../../Code}.
  .3 \myPhp{comprehend}.
  .3 \myPhp{upload\_wav}.
  .2 \myDir{css}.
  .3 \myCss{audio}.
  .3 \myCss{bootstrap.min}.
  .3 \myCss{main}.
  .2 \myDir{img}.
  .3 {* Website images and icons}.
  .2 \myDir{js}.
  .3 \myDir{recorderjs}.
  .4 \myJs{recorder}.
  .4 \myJs{recorderWorker}.
  .3 \myJs{audio}.
  .3 \myJs{audiodisplay}.
  .3 \myJs{bootstrap.bundle.min}.
  .3 \myJs{imgFileNames}.
  .3 \myJs{jquery-3.5.1.slim.min}.
  .3 \myJs{main}.
  .2 \myGit{.gitignore}.
  .2 \myApache{.htaccess}.
  .2 \myLink{datasets -> ../Datasets}.
  .2 \myImg{favicon.ico}.
  .2 \myHtml{index}.
  .2 \myOrg{README}.
  .2 \myFile{site.webmanifest}.
  .2 \myPdf{thesis}.
}\medskip

Note that there is a symbolic link here pointing to the datasets directory and
another pointing to the code directory. Therefore, all these directories are
actually part of the web server.
